% !TeX spellcheck = it_IT
\newpage
\section{Introduzione}

\subsection{Service-based economy}
L'economia basata sui servizi (\textit{Everything as a service}) si fonda sulla tendenza degli ultimi 30 anni di passare dai \textit{beni} ai \textit{servizi}.
\begin{example}[Bicicletta]
	Una signora compra una bicicletta dal venditore. Questa diventa di sua proprietà e si deve occupare della manutenzione. La bicicletta diventa un \textbf{servizio} (\textit{CicloPi}) e la signora paga per usare la bici che però non è più sua (e non deve più preoccuparsi di manutenzione e furto).
\end{example}
\noindent Esempi più vicini all'informatica sono il passaggio dai supporti fisici per la musica allo streaming o i dispositivi di memorizzazione passati ai Cloud Drive.
\subsection{Service contracts}
Quando usiamo un servizio non vogliamo sapere come viene implementato. L'unica cosa che ci interessa è cosa è specificato sul \textbf{contratto di utilizzo}. Nella maggior parte dei casi l'utente non lo legge.
\subsubsection{Quality of Service}
Ci sono più fornitori che ci danno lo stesso servizio ma con qualità del servizio diverse. Dobbiamo chiederci se il prezzo più basso vale la pena del sacrificio della qualità.
\subsubsection{Service Level Agreement}
Sono i contratti di servizio che includono anche il livello di \textbf{affidabilità di servizio}. In questa situazione abbiamo tre figure:
\begin{itemize}
	\item \textit{Programmatore}
	\item \textit{Business expert}: colui che sa il livello di affidabilità in base al mercato
	\item \textit{Legale}: colui che sa come scriverlo
\end{itemize}
\begin{example}[Google SLA]
	Google Compute Engine fornisce un \textbf{Service Level Objective} (SLO) del $99.95\%$. In caso di non raggiungimento del SLO si viene rimborsati con del credito in percentuale a quanto si è distanti dal target.\\
	\begin{center}
		\begin{tabular}{|c|c|}
			\hline
			\textbf{Montlhy Uptime Percentage} & \textbf{Rimborso} \\
			\hline
			$95.00\%-<99.95\%$ & $10\%$ \\
			\hline
			$90.00\%-<95.00\%$ & $25\%$ \\
			\hline
			$<90.00\%$ & $100\%$ \\
			\hline
		\end{tabular}
	\end{center}
	Se ad esempio l'1 e il 2 Aprile dalle 8am alle 5pm (orario di lavoro) non era disponibile il servizio, a quanto ammonta il rimborso in credito?\\
	Dobbiamo prima capire cos'è il \textbf{Montly Uptime Percentage} secondo l'SLA:
	\begin{definition}[Monthly Uptime Percentage]
		Total number of minutes in a month, minus the number of minutes of \textbf{Downtime} suffered from all \textbf{Downtime Period} in a month, divided by the total number of minutes in a month.
	\end{definition}
	\begin{definition}[Downtime Period]
		A period of one of more consecutive minutes of \textbf{Downtime}.
	\end{definition}
	\begin{definition}[Downtime]
		For virtual machines instances: loss of external connectivity or persistent disk accessfor the Single Instance or, with respect to Instances in Multiple Zones, all applicable running instances.
	\end{definition}
	Inoltre il cliente deve richiedere entro 60 giorni il rimborso fornendo anche dei \textbf{log file} che mostrino il \textbf{Downtime Period} assieme alla data e all'orario.
	\begin{equation*}
		a
	\end{equation*}
\end{example}

\begin{definition}[Legge del pesce rosso]
	Quando un fornitore di servizi non garantisce in alcun modo il prodotto.
	\begin{example}[Microsoft Service Agreement]
		Microsoft non è responsabile per alcuna perdita di dati o interruzione di servizi e non li garantisce in alcun modo (vedi art.6 e art.11).
	\end{example}
\end{definition}

\begin{example}
	Supponiamo di avere un servizio che si appoggia a due servizi cloud, ognuno disponibile in maniera indipendente il $90\%$ del tempo.
	Il servizio è probabile che non sarà disponibile per 4 ore e mezza al giorno:
	\begin{equation*}
		(1-(0.90 * 0.90))*24=4.56
	\end{equation*}
\end{example}

\subsection{Cloud}
Un primo fattore da tenere in considerazione è la \textbf{service demand}, che cambia nel tempo.\\
Prima del cloud c'era il problema di dimensionare il servizio in base a delle stime, che possono essere di due tipi:
\begin{itemize}
	\item \textbf{Overprovisioning}: utilizzare un infrastruttura che possa sopportare anche i carichi massimi previsti. In questo caso abbiamo un problema di \textbf{spreco di risorse}.
	\item \textbf{Underprovisioning}: quando l'infrastruttura non è sufficiente per tutti i momenti di carico. Questo causerebbe la perdita di clienti che dopo qualche tentativo fallito di utilizzo abbandonerebbero il servizio.
\end{itemize}
Il \textbf{cloud} ci fornisce risorse apparentemente infinite disponibili su richiesta.
\begin{definition}[Cloud]
	Secondo il NIST il Cloud Computing è un \textbf{modello} per consentire l'accesso ubiquo, conveniente e a richiesta di risorse computazionali.\\
	Le idee chiave sono le seguenti:
	\begin{itemize}
		\item Messa in comune di strutture autogestite, utilizzate come servizi
		\item Risorse virtuali \textbf{scalabili} disponibili attraverso internet a diversi clienti
		\item \textbf{Separazione} dei servizi dalla tecnologia alla base
	\end{itemize}
	Dal punto di vista economico sfrutta il principio \textbf{pay-per-use}, che permette all'utente di convertire i costi da Capital Expenses a Operation Expenses. Questo garantisce  elasticità e trasferimento dei rischi al fornitore dei servizi.
\end{definition}

\subsubsection{Service models}
Analizziamo ora i modelli di servizio, prendendo come esempio la pizza:
\begin{itemize}
	\item \textbf{IaaS}: fornisce server virtualizzati, archiviazione e rete e li gestisce. Il cliente è responsabile per tutti gli altri aspetti, quali OS e applicazioni. Nel nostro caso sarebbe comprare una pizza da cuocere.
	\item \textbf{PaaS}: fornisce l'intera piattaforma come servizio (VM, OS, servizi, SDKs) e gestisce l'infrastruttura, l'OS e l'abilitazione del software. Il cliente è responsabile dell'installazione e la gestione delle applicazioni. Nel nostro caso sarebbe una pizza d'asporto.
	\item \textbf{SaaS}: fornisce software a richiesta, disponibile tramite API. Viene gestita l'infrastruttura, l'OS e l'applicazione, quindi il cliente non è responsabile di nulla. Nel nostro caso sarebbe la pizzeria.
\end{itemize}

\subsubsection{Deployments models}
La scelta del deployment model può essere \textbf{pubblico}, \textbf{privato} o \textbf{ibrido}. Il modello privato garantisce un maggiore controllo sui dati mentre quello pubblico una maggiore scalabilità. La versione ibrida divide i dati tra pubblico e privato in base alle esigenze di sicurezza e scalabilità.
\subsubsection{Domande}
Ci si deve fare alcune domande sulla confidenzialità dei nostri dati:
\begin{itemize}
	\item Dove saranno fisicamente conservati i nostri dati?
	\item La privacy e l'integrità dei nostri dati sono garantite? Come?
	\item Come possiamo sapere se c'è stato un problema?
\end{itemize}
sulla disponibilità dei servizi: 
\begin{itemize}
	\item Cosa succede se il fornitore del servizio non lo fornisce?
	\item SPoF (Single Point of Failure): quando la nostra architettura sfrutta un solo servizio che potrebbe fallire
\end{itemize}
Le risposte le troviamo tutte nel SLA.

\subsubsection{Vendor Lock-In}
\begin{definition}[Vendor Lock-In]
	Il Vendor Lock-In è quando si rende un cliente dipendente da un venditore per prodotti o servizi, rendendogli impossibile il cambio ad un altro gestore senza affrontare costi esorbitanti.
\end{definition}
Cosa succede se si vuole cambiare provider? Il software continuerà a funzionare? Si potrà trasferire facilmente i propri dati? Serve sempre un \textbf{exit plan}.