\newpage
\section{Forze}
La legge oraria $\vec{F}(t)$ di un punto materiale di massa m è determinata salla soluzione di una equzione
del moto detta \textbf{seconda legge di newton}
$$m\ddot{\vec{r}} = \vec{F}_1 + \vec{F}_2 + \dots$$
$\vec{F}_1$ sono le \textbf{forze} $[kg \cdot m/s^2 \equiv N]$ (N è l'unità di misura, Newton) agenti sul punto meteriale: sono determinate empiricamente.
L'equazione differenziale è del \textbf{secondo ordine} (derivata seconda) quindi servono due \textbf{condizioni al bordo} (si chiamano così perché indicano le 
condizioni ai bordi del dominio), ad esempio $\vec{r}(t_0) = \vec{r}_0$ e $\dot{\vec{r}}(t_0) = \vec{v}_0$ (con questa cosa stiamo dicendo che per decsrivere un moto di un sistema
dobbiamo sapere in un tempo dove il sistema si trova e la sua velocità).\\
Questa è un equazione che va a descrivere fenomeni da dimensioni incredibilmente piccole a incredibilmente grandi.\\\\
Se la somma (detta \textbf{risultate} delle forze)
$$\vec{F}_1 + \vec{F}_2 + \cdots = 0 \hspace{10pt}\text{allora}\hspace{10pt} m\ddot{\vec{r}}(t) = 0 \Rightarrow \dot{\vec{v}}(t) \equiv \vec{v}_0$$
cioè il moto ha velocità costante (\textbf{rettilineo uniforme}). Questo è in particolare vero se tutte $\vec{F}_i = 0$ 
(\textbf{prima legge di Newton} o "principio di inserzia di Galileo"). Se un corpo non è soggetto a forze esterne mantiene il suo modo rettilineo uniforme.
Questa cosa collega la proprierà di simmetria degli oggetti alla traslazione dello spazio.

\subsection{Forza costante $\vec{F} = F_0\hat{x}$}
$$\vec{r}(t) = x(t)\hat{x} + y(t)\hat{y} + z(t)\hat{z} \hspace{15pt}\ddot(\vec{r})(t) = \ddot{x}(t)\hat{x} + \ddot{y}(t)\hat{y} + \ddot{z}(t)\hat{z}$$
$$m\ddot{\vec{r}}(t) = F_0 \hat{x} \Rightarrow 
\begin{cases}
    m\ddot{x}(t) = F_0\\
    m\ddot{y}(t) = 0\\
    m\ddot{t}(t) = 0 
\end{cases}
$$
Protietto su una base per ottenere 3 equazioni scalari. Mi servono $2 \times 3 = 6$ \textbf{condizioni al bordo} per risolvere.
Ad esempio codnizioni iniziali:
$$\vec{r}(0) = \vec{r_0} = (x_0, y_0, z_0) \hspace{15pt} \dot{\vec{r}}(0) = \vec{v}_0 = (v_{0x}, v_{0y}, v_{0z})$$
$$\Rightarrow 
\begin{cases}
    x(t) = v_0 + v_{0x}t + \frac{1}{2}\frac{F_0}{m}t^2\\
    y(t) = y_0 + v_{0y}t
    z(t) = z_0 + v_{0z}t
\end{cases}
$$
Con $t$ che rappresenta il \textbf{moto uniformamente accelerato}

\subsection{Forza peso $\vec{F} = -mg\hat{z}$}
Usata per esempio in prossimità della superficie terreste. Con $m$ massa del punto materiale dell'oggetto in cui si applica, mentre $\hat{z}$ ortogonale alla superficie. 
$g \equiv 9,8 m/s^2$, dipende da $M_T$, variazioni locali.
\begin{example}
    Grave che case da altezza $h$.
    $$m\ddot{\vec{r}}(r) = -mg\hat{z} \hspace{15pt} \text{con}\hspace{15pt} \vec{r}(t_0) = h \cdot \hat{z},, \dot{\vec{r}}(t_0) = 0 \text{(oggetto parte da fermo)}$$
    Proietto $m\ddot{z}(t)= -mg$ \hspace{10pt} $\dot{z}(t) = -g(t - t_0)$ \hspace{10pt} $z(t) = h - \frac{1}{2}g(t - t_0)^2$\\
    Si sostituisce le costanti della soluzione per verificare che siano verificate le condizioni ai bordi.
\end{example}

\begin{example}
    Problema del proiettile.
    $$m\ddot{\vec{r}}(t) = -mh\hat{z} \text{(equazione del moto)}\:\:\: \text{con}\vec{r}(t_0) = 0$$
    $$\text{e}\:\: \dot{\vec{r}}(t_0) = v_0 \cdot \cos\Theta\hat{x} + v_0 \cdot \sin\Theta\hat{z} \hspace{10pt}\text{ovver}\hspace{10pt}\vec{v}_0 = v_0(\cos\Theta, \sin\Theta) \:\: ||\vec{v}_0|| = v_0$$
    Proiezione lungo $\hat{y}$ banale: $\ddot{y}(t) \equiv 0, y(t) \equiv 0$
    $$
    \begin{cases}
        \ddot{x}(t) = 0& \hspace{20pt} \dot{x}(t) = v_0 \cos\Theta \\
        \dot{x}(t_0) = v_0 \cos\Theta & \hspace{20pt} x(t) = v_0\cos\Theta (t-t_0)\\
    \end{cases}
    $$
    $$
    \begin{cases}
        \ddot{z}(t) = -g & \hspace{20pt} \dot{t} = v_0 \sin\Theta - g(t - t_0)\\
        \dot{z}(t_0) = v_0 \sin\Theta & \hspace{20pt} z(t) = v_0\sin\Theta(t - t_0) - \frac{1}{2}g (t - t_0)^2
    \end{cases}
    $$
    Dalla legge oraria alla traiettoria
    $$
    \begin{cases}
        x(t) = v_0 \cos\Theta(t - t_0)\\
        z(t) = v_0\sin\Theta(t - t_0) - \frac{1}{2}g(t - t_0)^2
    \end{cases}
    $$
    $$t - t_0 = x(t) \: / \: (v_0\cos\Theta) \hspace{10pt} z = v_0\sin\Theta x\: / \: (v_0 \cos\Theta) - \frac{1}{2}g x^2 \: / \: (v_0\cos\Theta)^2 \hspace{10pt} z=v_0 tg\Theta - x^2 \frac{g}{2v_0^2}\frac{1}{\cos^2\Theta}$$
    \begin{observation}
        % Finisci
    \end{observation}
\end{example}

\subsection{Forza elastica $\vec{F} = -k (||\vec{r} - \vec{r}_v|| - l_0)\frac{\vec{r}-\vec{r}_v}{||\vec{r}-\vec{r}_v||}$}
Chiamata anche \textbf{legge di hooke}. $k$ è la costante elastica espressa in $[N/m]$ del materiale. $l_0 [m]$ lunghezza a riposo della "molla", 
dipende dal vettore posizione $\vec{r}$ ("posizionale"). Altro estremoi / vincolo $\vec{r}_v$.
$-\frac{\vec{r}-\vec{r}_v}{||\vec{r}-\vec{r}_v||}$ è il versone parallelo alla molla, cioò la distanza fra il punto di destinazione ed il vincolo.
La forza elastica è tanto piu intensa quanto è estesa la molla, e questa relazione fondamentale è lineare.
\begin{example}
    Oscillatore unidimensionale $\vec{r}_v = 0$.
    $$\vec{r}(t) = x(t)\vec{x}, x(t)\geq 0$$
    $$\vec{F}= -k(|x-0| - l_0)\frac{x - 0}{|x - 0|}\hat{x} \hspace{10pt} = \hspace{10pt} -k(|x| - l_0)\frac{x}{|x|}\hat{x} \Rightarrow F_x = -k(x-l_0)$$
    $$m\ddot{x}(t) = -k[x(t) - l_0]$$
    Soluzione generale (verifico per sostituzione)
    $$x(t) = l_0 + A \cdot \cos(\Omega t) + B \cdot \sin(\Omega t) \:\:\:\: \text{con} \Omega \equiv \sqrt{k/m}$$
    $$\Omega[rad/s] \text{la \textbf{frequenza angolare}} \hspace{15pt} \Omega/2\pi [\frac{1}{s} = Hz] \text{è la \textbf{frequenza}} \hspace{15pt} T = 2\pi/\Omega [s] \text{è il \textbf{periodo}, infatti } \Omega \cdot T = 2\pi$$ 
    Trovo $A$ e $B$ imponendo che la soluzione rispetti le condizoni al bordo, es: $x(0) = x_0, \dot{x}(0) = 0$. 
    Dalla soluzione generale ho 
    %aggiungi testo sopra freccia
    $$\dot{x}(t) = -\Omega A \sin(\Omega t) + B \Omega \cos(\Omega t) \Rightarrow 0 = -\Omega A \sin(\Omega \cdot 0) + B \Omega \cos(\Omega \cdot 0) \Rightarrow 0 = 0 + B\Omega \Rightarrow B = 0$$
    $$x(t) = l_0 + A \cdot \cos(\Omega t) \Rightarrow x_0 = l_0 + A \cdot \cos(\Omega \cdot 0) \Rightarrow x_0 = l_0 + A$$
    La soluzione completa è quindi
    $$x(t) = l_0 + (x_0 - l_0) \cos(\Omega t)$$
\end{example}

\subsection{Forza di attrito viscoso $\vec{F} = -\gamma\dot{\vec{r}}(t)$}
Modello approssimato per le basse velocità. Abbiamo che $-\gamma [N/(m/s)]$ è costante. Il meno è dato perché è una forza
che si oppone linearmente ad una velocità.
\begin{example}
    Proiettile in gel balistico.
    $$m\ddot{x}(t) = \gamma t \dot{x}(t) \:\: con \:\: \dot(0) = v_0$$
    Pongo poi $u(t) \equiv \dot{x}(t) \Rightarrow \dot{u} t = -\frac{1}{\tau}u(t)$ con $u(0) = v_0$ e $\frac{1}{\tau} = \frac{\gamma}{m}[\frac{1}{s}]$\\
    Soluzione generale $u(t) = A e^{-t/\tau} \Rightarrow v_0 = A \cdot e^0 \Rightarrow v_0 = A$.\\
    Quindi la soluzione completa è
    $$u(t) = v_0 e^{-t/\tau} \:\:\: \text{ovver} \:\:\: \dot{x}(t) = v_0 e^{-t/\tau}$$
    rallentamento esponenziale.
\end{example}

\subsection{Data la legge oraria, trovare la forza}
$r(t) = R, \Theta(t) = \Omega t$ \textbf{moto circolare uniforme}
$$\vec{r}(t) = r(t)\hat{r}\hspace{15pt}\hat{r} = \cos(\Theta(t))\hat{x} + \sin(\Theta(t))\hat{t}$$
$$\ddot{\vec{r}} (t) = [\ddot{r(t)} - r(t)\dot{\Theta}(t)^2]\hat{r} + [r(t)\ddot{\Theta}(t) + 2\dot{r}\dot{\Theta}(t)]\hat{\Theta}$$
Dalla legge oraria ho: $\dot(t) = 0, \ddot{r}(t) = 0, \dot{\Theta} = \Omega, \ddot{\Theta}(t) = 0 \Rightarrow$ in questo caso $\ddot{\vec{r}}(t) = -R\Omega^2\hat{r}$.\\
La risultate $\vec{F}$ delle forze deve essere tale che $m\ddot{\vec{r}}(t) = \vec{F} \Rightarrow -mR\Omega^2\hat{r} = \vec{F} \Rightarrow \vec{F} = -mR\Omega\vec{r}$.\\
La forza è costante e sempre diretta verso lo stesso punto (\textbf{forza centrale} o  \textbf{forza centripeda}). Ottengo $\vec{F}$ solo per questa legge oraria.