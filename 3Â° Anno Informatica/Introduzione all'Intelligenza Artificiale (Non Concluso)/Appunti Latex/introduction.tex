% !TeX spellcheck = it_IT
\newpage
\section{Introduzione}
\subsection{Obiettivi dell'IA}
\subsubsection{Modellare}
Modellare fedelmente l'essere umano:
\begin{itemize}
	\item \textbf{Agire umanamente}: Test di Turing\footnote{Ci sono due umani e una macchina. Tutti questi conversano tramite un computer. Se l'esaminatore non riesce a distinguere l'essere umano dalla macchina allora vince quest'ultima.}
	\item \textbf{Pensare umanamente}: modelli cognitivi per descrivere il funzionamento della mente umana
\end{itemize}
\subsubsection{Risultati}
Raggiungere i risultati ottimali:
\begin{itemize}
	\item Pensare razionalmente
	\item Agenti razionali: percepiscono l'ambiente, operano autonomamente e si adattano. Fanno la cosa giusta agendo in modo da ottenere il miglior risultato calcolando come agire in modo efficace e sicuro in una varietà di situazioni nuove. Ha alcuni vantaggi:
	\begin{enumerate}
		\item Estendibilità e generalità
		\item Misurabilità dei risultati rispetto all'obiettivo
	\end{enumerate}
	I limiti dipendono dai rischi, dall'etica e dalla complessità computazionale.
\end{itemize}
\subsection{Storia dell'IA}
Nasce sin dall'antichità con il desiderio dei filosofi di sollevare l'uomo dalle fatiche del lavoro. Dal 1940 c'è un esplosione di popolarità che però si alterna tra periodi di crisi e di grandi avanzamenti.
\begin{center}
\begin{tikzpicture}
	\draw (0,0) -- (15.5,0);
	
	\foreach \x in {1,3.5,9,15}
	\draw (\x cm,-0.1) -- (\x cm,0.1);
	
	\draw (1,0) node[below=3pt] {1943} node[above=3pt] {\textbf{Reti neurali}};
	\draw (3.5,0) node[below=3pt] {1947} node[above=3pt] {\textbf{Test di Turing}};
	\draw (9,0) node[below=3pt] {1956} node[above=3pt] {\parbox{20em}{\centering \textbf{Conferenza di Darthmouth} \\ Coniato il termine IA \\ Basata sulla congettura che l'intelligenza possa essere descritta precisamente e simulata}};
	\draw (15,0) node[below=3pt] {1957} node[above=3pt] {Ragionamento simbolico};
	
	\draw (15.5,0) -- (15.5,-3);
	\draw (0,-3) -- (15.5,-3);
	
	\draw (15.4cm,-1.5) -- (15.6cm, -1.5);
	
	\draw (15.5,-1.5) node[left=3pt] {1959} node[right=3pt] {\parbox{3em}{\centering\textbf{Giochi} \\ Dama}};
	
	\foreach \x in {2,8,12,15}
	\draw (\x cm,-3.1) -- (\x cm,-2.9);
	
	\draw (15,-3) node[above=3pt] {1962} node[below=3pt] {\parbox{8em}{\centering Unità neurali \\ con \textbf{apprendimento}}};
	\draw (12,-3) node[above=3pt] {'60} node[below=3pt] {\textbf{\textbf{Micromondi}}};
	\draw (8,-3) node[above=3pt] {1973} node[below=3pt] {\parbox{25em}{\centering \textbf{Rapporto di Lighthill} \\ Primo inverno IA\\ Manipolazione simbolica sintattica non adeguata \\  Limite di rappresentazione nelle singole unità neurali \\ Micromondi non scalari}};
	\draw (2,-3) node[above=3pt] {'80} node[below=3pt] {\parbox{10em}{\centering\textbf{Sistemi esperti} \\ Fifth Generation computer programme \\ Investimenti industriali}};
	
	\draw (0,-3) -- (0,-7.5);
	\draw [-{Latex[length=3mm, width=2mm]}](0,-7.5) -- (14,-7.5);
	
	\foreach \x in {3,10}
	\draw (\x cm,-7.4) -- (\x cm,-7.6);
	
	\draw (3,-7.5) node[below=3pt] {'90} node[above=3pt] {\parbox{15em}{\centering \textbf{Nuovo inverno IA} \\ Non c'è acquisizione di coscienza \\ No gestione incertezza \\ No buon senso}};
	\draw (10,-7.5) node[below=3pt] {2012} node[above=3pt] {\parbox{20em}{\centering \textbf{AI Revolution} \\ Machine learning \\ L'apprendimento diventa strategico}};
\end{tikzpicture}
\end{center}

\begin{example}[Scacchi]
	Un esempio propedeutico è quello dell'applicazione dell'IA al gioco degli scacchi, definita \textbf{IA debole}. Negli anni '60 c'erano principalmente due opinioni al riguardo:
	\begin{itemize}
		\item Newell e Simon sostenevano che in 10 anni le macchine sarebbero state campioni negli scacchi
		\item Dreyfus sosteneva che una macchina non sarebbe mai stata in grado di giocare a scacchi
	\end{itemize}
	Nel 1997 la macchina Deep Blue sconfigge il campione mondiale di scacchi Kasparov. Viene naturale farsi alcune domande...
	\begin{itemize}
		\item Ha avuto \textbf{fortuna}?
		\item Ha avuto un \textbf{vantaggio psicologico}? La macchina eseguiva le mosse immediatamente e Kasparov si sentiva come l'ultimo baluardo umano.
		\item \textbf{Forza bruta}? La macchina calcolava 36 miliardi di posizioni ogni 3 minuti
	\end{itemize}
	Oggi l'Intelligenza Artificiale eccelle in tutti i giochi. L'ultimo a "cadere" è stato il Go nel 2016. Allo stesso tempo però il livello delle persone è aumentato giocando contro le macchine.
\end{example}

\begin{definition}[IA debole]
	Al contrario dell’IA forte, non ha lo scopo di possedere abilità cognitive generali, ma piuttosto di essere in grado di	risolvere esattamente un singolo problema.
\end{definition}

\subsection{Reti neurali}
Le reti neurali sono caratterizzate da:
\begin{itemize}
	\item \textbf{Flessibilità}: capacità di acquisizione automatica di conoscenza e di adattamento automatico a contesti diversi e dinamici
	\item \textbf{Robustezza}: capacità di trattare incertezza e rumorosità del mondo reale
	\item Rappresentazione appresa dai dati in forma \textbf{sub-simbolica}
	\item Possibilità di usare più strati di reti neurali con diversi livelli di astrazione (\textbf{Deep Learning})
\end{itemize}

\subsubsection{Deep Learning}
Abbinando alla capacità dei modelli di machine learning una grande quantità di dati e degli High Performance Computer, si è favorito molto il deep learning.\\
Dal 2010 le reti neurali profonde hanno iniziato a diffondersi molto nelle grandi industrie, riscuotendo successo ad esempio:
\begin{itemize}
	\item \textbf{Computer vision}: ad esempio la classificazione del cancro della pelle
	\item \textbf{Natural Language Processing}: ad esempio IBM Watson o Google DeepL
\end{itemize}
Questa tecnologia ha raggiunto prestazioni a livello di quelle umane.