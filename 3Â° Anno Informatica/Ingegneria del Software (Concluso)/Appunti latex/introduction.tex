% !TeX spellcheck = it_IT
\newpage
\section{Introduzione}
L'ingegneria del software è modo in cui produciamo il software, dall'esplorazione del problema al ritiro del prodotto dal mercato. Riguarda tutti gli \textbf{strumenti}, le \textbf{tecniche} e i \textbf{professionisti} coinvolti nelle seguenti fasi:
\begin{enumerate}
	\item Analisi dei requisiti
	\item Specifica
	\item Progettazione
	\item Implementazione
	\item Integrazione
	\item Mantenimento
	\item Ritiro
\end{enumerate}
\begin{definition}[Processo software]
	È un approccio sistematico per \textbf{sviluppo}, \textbf{operatività}, \textbf{manutenzione} e \textbf{ritiro} del software.
\end{definition}
\subsubsection{Scopo}
Lo scopo è quello di produrre software che sia:
\begin{itemize}
	\item Fault free
	\item Consegnato in tempo
	\item Rispetti il budget
	\item Soddisfi le necessità
	\item Sia facile da modificare
\end{itemize}

\subsection{Casi di studio}
Richiamiamo alcune definizioni in ambito software:
\begin{definition}[Robustezza]
	Capacità di un software di mantenere il suo corretto funzionamento anche quando sottoposto a condizioni anomale (errori, eccezioni o input non validi). Contribuisce a garantire che il sistema sia affidabile e che possa continuare ad operare anche in situazioni critiche non previste.
\end{definition}
\begin{definition}[Fault tolerance]
	Capacità di un software di rilevare, gestire e riprendersi da errori o guasti senza causare interruzioni significative nei servizi o la perdita di dati.
\end{definition}
\subsubsection{Gemini V}
Missione nello spazio con equipaggio. Al suo rientro la navicella atterrò ad $80km$ dal punto previsto a causa di un \textbf{errore nel modello} (uno sviluppatore inserì la rotazione terrestre sbagliata).
\subsubsection{Denver Airport}
Il progetto prevedeva lo smistamento automatico dei bagagli con un sistema troppo complesso: i tempi di costruzione furono notevolmente allungati, i costi furono più del previsto e non c'era \textbf{fault tolerance} (il guasto di un singolo PC bloccava l'intero sistema). Alla fine venne abbandonato.
\subsubsection{THERAC-25}
Una macchina da radioterapia con un software progettato male e poco robusto: in caso di errore le emissioni di raggi non venivano sempre terminate ed era possibile da parte dell'operatore ignorarlo. Causò $3$ decessi su $6$ pazienti.
\subsubsection{Sistema anti-missile Patriot}
Un sistema poco robusto che non riuscì ad evitare un attacco in Arabia Saudita con conseguenti 28 morti. La causa fu l'uso oltre il tempo di progettazione: $100$ ore contro le $14$ previste.
\subsubsection{London ambulance service}
Un sistema per l'ottimizzazione delle ambulanze a Londra, migliorando i percorsi e istruendo vocalmente gli autisti. In questo caso era l'analisi del problema ad essere errata. Inoltre la UI era inadeguata, gli utenti poco addestrati e non era previsto un backup.
\subsubsection{Ariane V}
Un lanciatore per razzi che si è autodistrutto dopo $40$ secondi a causa di un errore di sviluppo e test inefficienti: veniva usato una variabile a $16bit$ per un valore a $64bit$, causando un overflow.
\subsubsection{Toyota}
Il software per le macchine Toyota tra il 2000 e il 2013 fu mal progettato e causava acceleramenti involontari del veicolo.
\subsubsection{METEOR}
Prima metro al mondo ad essere automatizzata, locata a Parigi. Fu un successo grazie alla sua ottima progettazione.

Questo ci porta a concludere che data la forte presenza del software nel mondo di oggi è importante utilizzare tecniche di ingegneria del software per renderlo affidabile, rapido da produrre e sostenibile.

\subsection{Storia}
Tra il 1963 e il 1964, durante lo sviluppo dei sistemi di guida e navigazione per la missione Apollo, viene coniato il termine software engineering da Margaret Hamilton. \\
Nel 1968 la NATO organizza una conferenza al riguardo in quanto la qualità del software era bassa ed era necessario decidere tecniche e paradigmi per la produzione di software.\\
Nel 1994 viene fatta un'analisi media del software prodotto che fa vedere come:
\begin{itemize}
	\item Il $16.2\%$ del software sia stato prodotto in tempo
	\item Il $52.7\%$ sia stato in ritardo, a causa di difficoltà iniziali, cambiamento della piattaforma e difetti finali
	\item Il $31.1\%$ non sia stato completato per obsolescenza prematura, incapacità e mancanza di fondi
\end{itemize}
In particolare le cause di abbandono evidenziate furono relativi a tre aspetti:
\begin{itemize}
	\item Aspettative incomplete, che cambiano nel tempo o che non sono chiare
	\item Mancanza di risorse e aspettative su funzionalità e tempi irrealistiche 
\end{itemize}

\subsection{Aspetti importanti}
Vediamo alcuni aspetti importanti dello sviluppo del software.
\subsubsection{Specificità}
Per natura, un software è diverso da altri prodotti ingegnerizzabili in quanto non è necessariamente vincolato da materiali e leggi fisiche. Inoltre non si consuma e non ha costi aggiuntivi per ogni unità prodotta.\\
Ad esempio la \textbf{fault tolerance} è una specificità, in quanto quando un software crasha lo si fa ripartire cercando di minimizzare gli effetti del problema.
\subsubsection{Economia}
L'ingegneria del software si occupa anche di cercare soluzioni vantaggiose dal punto di vista economico. In particolare è importante notare come il costo maggiore derivi sempre dalla \textbf{manutenzione} successiva alla produzione del software. Questa può essere di due tipi:
\begin{itemize}
	\item \textit{Correttiva}: rimuove gli errori lasciando invariata la specifica
	\item \textit{Migliorativa}: cambia le specifiche
	\begin{itemize}
		\item \textit{Perfettiva}: migliora la qualità del software, fornisce nuove funzionalità o ne migliora di esistenti
		\item \textit{Adattativa}: modifiche a seguito del cambiamento del contesto
	\end{itemize}
\end{itemize}
\subsubsection{Teamwork}
La maggior parte del software è prodotto in team. Questo porta a diversi problemi, come l'interfaccia tra le varie componenti del codice e la comunicazione tra i membri del team. L'ingegneria del software deve quindi gestire anche i rapporti umani e l'organizzazione di un team.