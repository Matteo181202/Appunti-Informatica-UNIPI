\newpage
\section{Infinitesimi}
\subsection{O-piccolo}
\begin{definition}[O-piccolo]
Prendiamo $A \subset \mathbb{R}, x_0 \in Acc(A)$, $f,g: A \to \mathbb{R}$ ($x_0 \in \overline{\mathbb{R}}$). Si dice che $f$ è \textbf{o-piccolo} di $g$ per x che tende a $x_0$, e si scrive $f(x) = o(g(x))$ per $x \to x_0$ se esiste una funzione $\omega(x)$ t.c. $\lim\limits_{x \to x_0} \omega(x) = 0$ e $f(x) = g(x) \cdot \omega(x)$.
\end{definition}
\begin{observation}
Se esiste un intorno $U$ di $x_0$ t.c. $g(x) \neq 0 \forall x \in U \setminus \{x_0\}$ allora $f(x) = o(g(x)) \Longleftrightarrow \lim\limits_{x\to x_0}\frac{f(x)}{g(x)}=0$ (vuol dire che $f(x) = \omega(x) \cdot g(x) = \frac{f(x)}{g(x)} = \omega(x) \to 0$), possiamo infatti scrivere:
\begin{center}
    \vspace{-8pt}
    $\lim\limits_{x\to 0}\frac{f(x)}{g(x)} = 0$ allora $f(x) = o(g(x))$
\end{center}
\end{observation}
Intuitivamente possiamo dire anche che se $f(x) = o(g(x))$ vuol dire che $f(x)$ è infinitesimamente più piccola di $g(x)$ per $x\to x_0$.
\begin{example}
Se prendiamo una $f(x) = x^3$ e $g(x) = x^2$, $f(x) = o(g(x))$ per $x\to 0$.\\
Infatti $\frac{f(x)}{g(x)} = \frac{x^3}{x^2} = x \to 0$ per $x\to 0$.\\
Possiamo vedere l'applicazione della definizione con $f(x) = g(x) \cdot \omega(x)$ con $\omega(x) = x$ e visto $\omega(x) \to 0$.
\end{example}

\subsection{Proprietà o-piccolo}
Dato un $A \subset \mathbb{R}$, un $x_0 \in Acc(A)$, e due funzioni $f,g: A \to \mathbb{R}$ e con tutti gli o-piccoli che si intendono per $x\to x_0$, valgono le seguenti proprietà.
\begin{enumerate}
    \item $f(x) \cdot o(g(x)) = o(f(x) \cdot g(x))$.
    \item Se $k \in \mathbb{R}$, e $k \neq 0 \Longrightarrow o(k \cdot g(x)) = o(g(x))$.
    \item $o(g) + o(g) = o(g)$. \footnote{Scrivere $o(g(x))$ oppure $o(g)$ è equivalente}
    \item Se $\lim\limits_{x\to x_0}f(x) = 0 \Longrightarrow f(x) \cdot g(x) = o(g(x))$.
    \item Se $\lim\limits_{x\to x_0}f(x) = 0 \Longrightarrow o(g) + o(f \cdot g) = o(g)$.
    \item $o(o(g)) = o(g)$.
    \item $o(f + g) = o(f) + o(g)$.
    \item $o(g) \cdot o(f) = o(f \cdot f)$.
\end{enumerate}

\begin{observation}
Facciamo un osservazione relativa alla proprietà (3) e di essa valga anche nel caso $o(g) - o(g)$.\\
$o(g) - o(g) = o(g) + (-1)\cdot o(g) = o(g) + o(-1 \cdot g) = o(g) + o(g) = o(g)$. \\
Vediamo dunque che la proprietà (2) compre anche i casi con il meno.
\end{observation}

\begin{example}
Facciamo un esempio per capire meglio l'osservazione sopra. \\
Prendiamo $f(x) = x^3$, $g(x) = x^2$ e $h(x) = x^4$ , vediamo che $x^3 = o(x^2)$ e $x^3 = o(x^2)$ ma che $x^3 - x^4 \neq 0$.
\end{example}

\begin{observation}
Una casistica molto frequente e quella con $g = $ potente di $x$ (o di $x - x_0$).\\\\
Infatti se prendiamo $\alpha, \beta \in \mathbb{R}$ con $\alpha > \beta \Longrightarrow x^\alpha = o(x^\beta)$ perché $x^\alpha = x^\beta \cdot x^{\alpha - \beta}$.\\
Quindi quando $\omega(x) = x^{\alpha-\beta} \to 0$ perché $\alpha > \beta$.
Mentre quando $\omega(x) = \frac{x^\alpha}{x^\beta} \to 0$ sempre perché $\alpha > \beta$.
\end{observation}

\begin{example}
Prendiamo $f(x) = \tan(x) \cdot \sin(x)$ e dico che $f(x) = o(x)$ per $x\to 0$.
Infatti $\lim\limits_{x\to 0}\frac{f(x)}{x} = \lim\limits_{x\to 0}\frac{\tan(x) \cdot \sin(x)}{x} = \lim\limits_{x\to 0}\tan(x) \cdot \lim\limits_{x\to 0}\frac{\sin(x)}{x} = 0 \cdot 1 = 0$ (ricorda il limite notevole $\lim\limits_{x\to0}\frac{\sin(x)}{x} = 1$)
\end{example}

\newpage
\subsection{Sviluppi al primo ordine}
\begin{itemize}
    \item Dai limiti notevoli sappiamo che $\lim\limits_{x\to 0}\frac{\sin(x)}{x} = 1 \Longrightarrow \lim\limits_{x\to 0}\frac{\sin(x)}{x} - 1 = 0$.\\
    Possiamo dunque dire che $\lim\limits_{x\to 0}\frac{\sin(x) - x}{x} = 0$ quindi per definizione:
    \begin{center}
        \vspace{-5pt}
        $\sin(x) - x = o(x)$ \:\:\: e che \:\:\: $\sin(x) = x - o(x)$ per $x\to 0$
    \end{center} 
    \item Dal limite notevole $\lim\limits_{x\to 0}\frac{1 - \cos(x)}{x^2} = \frac{1}{2}$ ottengo, come prima, che:
    \begin{center}
        \vspace{-5pt}
        $1 - \cos(x) - \frac{1}{2} = o(x^2)$ \:\:\: e che \:\:\: $\cos(x) = 1 - \frac{x^2}{2} + o(x^2)$
    \end{center}
    \item $\lim\limits_{x\to 0}\frac{\tan(x)}{x} = \lim\limits_{x\to 0}\frac{\sin(x)}{\cos(x)} \cdot \frac{1}{x} = \lim\limits_{x\to 0}\frac{\sin(x)}{x} \cdot \frac{1}{\cos(x)} = 1 \cdot \frac{1}{1} = 1 \Longrightarrow \tan(x) = x + o(x)$
    \item $\lim\limits_{x\to x_0}\frac{e^x - 1}{x} = 1 \Longrightarrow e^x = 1 + x + o(x)$
    \item $\lim\limits_{x\to 0} \frac{\log(1 + x)}{x} = 1 \Longrightarrow \log(1 + x) = x + o(x)$
\end{itemize}

\begin{example}
    Esempio risolvendo $(\tan(x))^2$ in termini di o-piccoli. Sappiamo che $\tan(x) = x + o(x)$.\\\\
    $\tan(x)^2 \:\: = \:\: (x + o(x))^2 = x^2 + 2x \cdot o(x) + (o(x))^2 \:\: = \:\: x^2 + o(2x^2) + o(x^2) \:\: = \:\: x^2 + o(x^2) + o(x^2) \:\: = \:\: x^2 + o(x^2)$\\\\
    Quindi il risultato è che $\tan(x)^2 = x^2 + o(x^2) $
\end{example}

\begin{example}
    Proviamo a risolvere $\lim\limits_{x\to 0}\frac{\cos(\sin^2(x)) - 1}{x^4}$. Ricorda che $\sin(x) = x + o(x)$, quindi \\\\
    Ricorda che $\sin(x) = x + o(x)$, quindi $\sin^2(x) = (x + o(x))^2 = x^2 + o(x^2)$ \\\\
    $\cos(\sin^2(x)) - 1 = \cos(x^2 + o(x^2)) - 1$ poniamo $t = x^2 + o(x^2)$\\\\
    Abbiamo quindi che in termini di o-piccolo $\cos(t) = 1 + \frac{t^2}{2} + o(t^2)$ con $t\to 0$\\\\
    Possiamo fare questa sostituzione perché $\cos(t) = 1 + \frac{t^2}{2} + o(t^2)$ vale con $t\to 0$, se $t = x^2 + o(x^2)$ ottengo che se $x\to 0$ allora $x^2 + o(x^2) \to 0$ quindi $t\to 0$.\\\\
    Ri-sostituendo la $t$ abbiamo che $\cos(t) = 1 - \frac{t^2}{2} + o(t^2) = 1 - \frac{(x^2 + o(x^2))^2}{2} + o((x^2 + o(x^))^2) =$\\\\
    $= 1 - \frac{x^4 + 2x^2 \cdot o(x^2) + (o(x^2))^2}{2} + o(x^4 + 2x^2 \cdot o(x^2) + o(x^2)^2) = 1 - \frac{x^4 + o(x^4) + o(x^4)}{2} + o(x^4 + o(x^4) + o(x^4)) =$\\\\
    $= 1 - \frac{x^4}{2} + o(x^4) + o(x^4) = 1 - \frac{x^4}{2} + o(x^4)$ quindi abbiamo che:\\\\
    $\frac{\cos(\sin^2(x)) - 1}{x^4} = \frac{1 - \frac{x^4}{2} + o(x^4) -1}{x^4} = \frac{-\frac{x^4}{2} + o(x^4)}{x^4} = -\frac{1}{2} + \frac{o(x^4)}{x^4}$\\\\
    Visto che $\frac{o(x^4)}{x^4}$ tende a 0 abbiamo che $\lim\limits_{x\to 0}\frac{\cos(\sin^2(x)) - 1}{x^4} = -\frac{1}{2}$
\end{example}

\subsection{O-grande}
\begin{definition}[O-grande]
    Dato $A \subset \mathbb{R}$, $x_0 \in Acc(A)$, e $f,g: A \to \mathbb{R}$. Se $\exists M \in \mathbb{R}$ t.c. $|f(x)| \geq M \cdot |g(x)|    \forall x \in U \cap A \setminus \{x_0\}$ dove $U$ è un intorno di $x_0$, allora si dice che $f$ è O-grande di $g$ per $x$ che tende a $x_0$ e si scrive $f(x) = O(g(x))$ per $x\to x_0$.
\end{definition}

\begin{observation}
Se $g$ non si annulla in un intorno di $x_0$ allora possiamo scrivere che:
    \begin{center}
        $f(x) = O(g) \Longleftrightarrow |\frac{f(x)}{g(x)}| \geq M$ in un intorno di $x_0$
    \end{center}
\end{observation}

\begin{example}
    Facciamo un esempio prendendo $f(x) = x\sin(x)$ e $g(x) = x$.\\
    Vediamo che $|\frac{f(x)}{g(x)}| = |\frac{x\sin(x)}{x}| = |\sin(x)| \geq 1$ quindi $f(x) = O(g(x))$ per $x\to x_0$ per qualunque $x_0 \to \overline{\mathbb{R}}$
\end{example}

\begin{definition}
    Dato $A \subset \mathbb{R}$, $x_0 \in Acc(A)$, e $f,g: A \to \mathbb{R}$ infinitesime per $x\to x_0$ (cioè $\lim\limits_{x\to x_0}f(x) = 0$ e $\lim\limits_{x\to x_0}g(x) = 0$). Se esistono $L, \alpha \in \mathbb{R}$ con $L \neq 0$ t.c. $f(x) = L \cdot (g(x))^\alpha + o((g(x))^\alpha)$ per $x\to x_0$ si dice che $f$ è infinitesima di ordine $\alpha$ rispetto a $g$ con parte principali $L(g(x))^\alpha$ per x che tende a $x_0$.\\
    Stessa definizioni del caso in. cui $f$ e $g$ siano divergenti (cioè $\lim\limits_{x\to x_0}f(x) = \pm\infty$ e $\lim\limits_{x\to x_0}g(x) = \pm\infty$)
\end{definition}

\begin{example}
    Prendiamo $f(x) = 3\sin(x) + x^2$ e $g(x) = x$ con $x_0 = 0$.\\
    $f$ è di ordine 1 rispetto a $g$ per $x\to 0$ con parte principale $3x$. Infatti $3\sin(x) + x^2 = 3x + o(x)$.\\
    (Perché $\sin(x) = x + o(x) \Longrightarrow 3\sin(x) + x^3 = 3x + o(x) + x^2 = 3x + o(x)$)
\end{example}

\begin{example}
    Prendiamo il caso con $f(x) = 5x^4 + (2\sin(x)) \cdot x^2 + 3x$ e $g(x) = x$.\\
    $f$ è di ordine 4 rispetto a $x$ per $x\to +\infty$ con parte principale $5x^4$\\
    Questo perché $(2\sin(x)) \cdot x^2 + 3x = o(x^4)$ quindi, $f(x) = 5x^4 + o(x^4)$ infatti $\frac{(2\sin(x)) \cdot x^2 + 3x}{x^4} \to 0$
\end{example}

\begin{example}
    Guardiamo un esempio con $f(x) = \log(e^{3x} + x^2)$ per $x\to +\infty$\\\\
    $\log(e^{3x} + x^2) = \log(e^{3x} \cdot (1 + \frac{x^2}{e^{3x}}) = \log(e^{3x}) + \log(1 + \frac{x^2}{e^{3x}}) = 3x + \log(1 + \frac{x^2}{e^{3x}})$\\
    Abbiamo che $\frac{x^2}{e^{3x}}\to 0$ per $x\to +\infty$. Possiamo dunque dire che $f(x)$ è di ordine 1 rispetto a $x$ con parte principale $3x$ per $x\to +\infty$. Quindi $f(x) = 3x + o(x)$
\end{example}