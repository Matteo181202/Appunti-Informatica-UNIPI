% !TeX spellcheck = it_IT
\newpage
\section{Dizionari}
\begin{definition}
	Un dizionario è una \textbf{struttura dati astratta} e consiste in una collezione di \textbf{coppie} della forma:
	\begin{itemize}
		\item \textbf{Chiave}: è il meccanismo di accesso all'elemento ed è \textbf{univoca} nella collazione
		\item \textbf{Elemento}:  è un qualsiasi tipo
	\end{itemize} 
\end{definition}
Le operazioni che possono essere eseguite sui dizionari sono \emph{inserimento}, \emph{ricerca} e \emph{cancellazione}.
\subsection{Indirizzamento diretto}
Ogni elemento del dizionario viene mappato con una chiave estratta da un determinato universo $U$ e che corrisponde alla posizione nell'array dove viene inserito. \\
La complessità di questa tecnica è $O(1)$ per tutte e tre le operazioni possibili sui dizionari. Il problema si presenta quando le chiavi usate $N$ sono molto minori dell'universo iniziale $U$ e abbiamo quindi che $\frac{N}{U} < 1$ con un conseguente \textbf{spreco di spazio di memoria} in quanto lo spazio nell'array dovrà comunque essere allocato.
\subsection{Chaining}
Un modo per evitare i problemi dell'\emph{indirizzamento diretto} è utilizzando un array più piccolo dell'universo possibile di chiavi. Questo ovviamente causa situazioni in cui c'è un \textbf{conflitto} perché due o più elementi voglio utilizzare la stessa chiave. Viene quindi creata una \textbf{lista} di elementi associata alla posizione in cui si è creato il conflitto e ogni nuovo elemento che vuole inserirsi lì verrà messo sulla testa della lista. \\
Per capire ogni elemento a quale chiave deve essere associato si utilizza una \textbf{funzione di hash} che deve essere definita in modo da ridurre il numero di collisioni e da generare sempre lo stesso indice per la stessa chiave.
%TODO Inserisci immagini
\subsection{Open addressing}
I conflitti di indirizzo vengono risolti cercando la prima posizione libera. Questo processo di ricerca viene chiamato \textbf{probing}. Questa operazione può avvenire in diversi modi:
\begin{itemize}
	\item \emph{Lineare}: vado alla posizione successiva
	\item \emph{Quadratico}: uso una funzione quadratica
	\item \emph{Doppio hash}: utilizzo la funzione hash per cercare la posizione successiva
\end{itemize}
\begin{example}
	Esempio di esecuzione di \emph{probing}:
	\begin{gather*}
		probe(0) \longrightarrow h(i) \\
		probe(1) \longrightarrow (h(i) + f(1))\\
		\vdots \\
		probe(j) \longrightarrow (h(i) + f(j))\\
		\vdots \\
		probe(m-1) \longrightarrow Fail\\
	\end{gather*}
\end{example}

\begin{example}
	Esempio di \emph{probing lineare}.
\end{example}
\begin{observation}
	Il problema che nasce dell'open addressing lineare è il \textbf{clustering primario}. Quando infatti si verifica una collisione, viene riempita una cella di memoria nelle vicinanze, aumentando così il rischio di collisione e aumentando il tempo necessario per la ricerca. \\ Questo avviene perché la \emph{probabilità} che uno slot preceduto da slot pieni sia riempito è pari a $\frac{i + 1}{m}$.
\end{observation}
\begin{observation}
	Il problema che deriva dall'utilizzo di open addressing quadratico è il \textbf{clustering secondario}. Infatti in questo caso se due chiavi hanno la stessa posizione iniziale la loro sequenza di tentativi sarà la stessa.
\end{observation}