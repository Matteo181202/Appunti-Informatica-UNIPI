% !TeX spellcheck = it_IT
\newpage
\section{Liste}
Una \textbf{struct} (struttura) \textbf{autoreferenziale} ha un membro puntatore che punta a una struttura dello stesso tipo, chiamato \textbf{link}, e che serve a creare una \emph{catena} (lista) di nodi collegati tra loro.
\begin{lstlisting}[language=C, caption=Esempio di una struct]
	struct nodo {
		int dato;
		struct nodo *nextPtr;
	} n1, n2, n3;
\end{lstlisting}

\begin{definition}[Struttura dinamica]
	È una struttura dati che può \textbf{variare} la sua dimensione a tempo di esecuzione, aumentando e diminuendo. Alcuni esempi sono proprio le \textbf{liste}, oltre che le pile, le code e gli alberi binari.
\end{definition}

\begin{note}
	L'ultimo nodo di una lista conterrà nel link \textbf{null}.
\end{note}

\subsection{Confronto tra liste e array}
\begin{table}[!h]
	\centering
	\begin{tabular}{|p{150px}|p{150px}|}
		\hline
		\textbf{Liste} & \textbf{Array} \\
		\hline
		Contiene sequenze di dati & Contiene sequenze di dati \\ 
		\hline
		Create dinamicamente a tempo di esecuzione e la dimensione non può essere prevista a tempo di compilazione & Creati staticamente a tempo di esecuzione e la dimensione deve essere calcolabile a tempo di compilazione \\
		\hline
		La dimensione è variabile &  La dimensione è costante e tutti gli elementi sono allocati a tempo di definizione\\
		\hline
		Diventa piena solo quando termina la memoria disponibile nell'heap & Diventa pieno quando sono pieni tutti gli elementi \\
		\hline
		Non sono memorizzate in celle contigue & Sono in celle contigue \\
		\hline
		Possono essere manipolate senza spostare elementi & Per manipolarli bisogna spostare gli elementi \\
		\hline
	\end{tabular}
\end{table}

\subsection{Operazioni sulle liste}
\subsubsection{Insert}
Viene codificata come segue:
\begin{lstlisting}[language=C, caption=Esempio di una struct]
	void insert(NodoPtr *lPtr, int val) {
		// Alloco nuovo nodo
		NodoPtr nuovoPtr = malloc(sizeof(Nodo));
		if (nuovoPtr != NULL) {
			// Inizializzo nodo
			nuovoPtr->dato = val;
			nuovoPtr->prossimoPtr = NULL;
			NodoPtr precedentePtr = NULL;
			NodoPtr correntePtr = *lPtr;
			while (correntePtr != NULL && val > correntePtr->dato) {
				precedentePtr = correntePtr;
				correntePtr = correntePtr->prossimoPtr;
			}
			if (precedentePtr == NULL) {
				// Inserimento all'inizio della lista
				nuovoPtr->prossimoPtr = *lPtr;
				*lPtr = nuovoPtr;
			}
			else {
				// Inserimento tra due nodi
				precedentePtr->prossimoPtr = nuovoPtr;
				nuovoPtr->prossimoPtr = correntePtr;
			}
		}
		else {
			puts("Memoria esaurita");
		}
	}
\end{lstlisting}