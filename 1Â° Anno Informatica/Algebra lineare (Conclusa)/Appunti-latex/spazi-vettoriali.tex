% !TeX spellcheck = it_IT
\newpage
\section{Spazi vettoriali}
\subsection{Spazio n-dimensionale}
\begin{definition}
	Uno spazio \textbf{n-dimensionale} standard su $\mathbf{R}$ si rappresenta come:
	\begin{equation*}	
		R^n = \Bigg\{ \begin{bmatrix}x_1\\x_2\\ \vdots \\ x_n\end{bmatrix} : x_i \in \mathbb{R} \Bigg\}
	\end{equation*}
\end{definition}
\noindent Geometricamente uno spazio n-dimensionale con $n=2$ sarà un punto sul piano cartesiano.\\
\begin{minipage}{.3\linewidth}
	\centering
	\[
	\begin{bmatrix}a_1\\a_2\end{bmatrix} \Longleftrightarrow Punto
	\] 
\end{minipage}
%TODO Inserire piano cartesiano con punto
\subsection{Operazioni}
Sugli spazi \emph{n-dimensionali} si possono effettuare alcune operazioni:
\begin{itemize}
    \item \textbf{Somma} $(x_1, x_2, x_3) + (x_1', x_2', x_3') = (x_1 + x_1', x_2 + x_2', x_3 + x_3')$.
    \item \textbf{Moltiplicazione} $\lambda(x_1, x_2, x_3) = (\lambda x_1, \lambda x_2, \lambda x_3)$.
\end{itemize}
\begin{figure}[h!]
    \vspace{-12pt}
    \centering
    \begin{minipage}{.3\linewidth}
    \centering
    \[
    \begin{bmatrix}x_1\\x_2\\ \vdots \\ x_n\end{bmatrix} + \begin{bmatrix}x_1'\\x_2'\\ \vdots \\ x_n'\end{bmatrix} = \begin{bmatrix}x_1 + x_1'\\x_2 + x_2'\\ \vdots \\ x_n + x_n'\end{bmatrix}
    \] 
    Somma
    \end{minipage}
    \begin{minipage}{.3\linewidth}
    \centering
    \[
    \lambda \cdot \begin{bmatrix}x_1\\x_2\\ \vdots \\ x_n\end{bmatrix} = \begin{bmatrix}\lambda x_1\\ \lambda x_2\\ \vdots \\ \lambda x_n\end{bmatrix}
    \]
    Moltiplicazione
    \end{minipage}
\end{figure}
%TODO Inserire rappresentazione geometrica di somma e prodotto

\subsection{Spazio vettoriale}
\begin{definition}[Spazio vettoriale]
Uno spazio vettoriale su $\mathbb{R}$ è un insieme V che ammette due tipi di operazioni:
\begin{itemize}
    \item Somma: dati $v_1, v_2 \in V \Longrightarrow v_1 + v_2 \in V$.
    \item Prodotto con $\lambda \in \mathbb{R}$: dato $v \in V \Longrightarrow \lambda \cdot v \in V$.
\end{itemize}
\end{definition}
\hspace{-15pt}Per queste operazioni esistono anche una serie di assiomi che devono essere rispettati:
\begin{table}[h!]
    \setlength{\tabcolsep}{5pt}
    \renewcommand{\arraystretch}{1.7}
    \centering
    \begin{tabular}{|c|c|}
        \hline
        Assiomi Somma & Assiomi Moltiplicazione\\
        \hline\hline
        $(v_1 + v_2) + v_3 = v_1 + (v_2 + v_3)$ & $(\lambda_1 + \lambda_2) v = \lambda_1 v + \lambda_2 v$\\
        $v_1 + v_2 = v_2 + v_1$ & $\lambda(v_1 + v_2) = \lambda v_1 + \lambda v_2$\\
        $\nexists\: 0 \in V \: : \: 0 + v = v + 0 = v \:\: \forall \:v$ & $(\lambda_1 \: \lambda_2)v = \lambda_1(\lambda_2 \: v)$\\
        $\forall \: v \nexists \: -v \in V \: : \: v + (-v) = (-v) + v = 0$ & $(1 \cdot v) = v$\\\hline
    \end{tabular}
    \caption{Assiomi somma e moltiplicazioni vettori}
\end{table}
\vspace{-10pt}
\begin{observation}
$R^n$ soddisfa tutti gli assiomi sopra scritti.
\end{observation}
\newpage
\begin{example}
Consideriamo una matrice $n \times m$ elementi reali $M_{n\times m}(\mathbb{R})$.
\begin{figure}[h!]
    \begin{minipage}{.2\linewidth}
    \vspace{-10pt}
    \centering
    \[
    \begin{bmatrix}
    a_{11} & \cdots & a_{1m}\\
    \vdots \\
    a_{n1} & \cdots & a_{nm}
    \end{bmatrix}
    \]
    \end{minipage}
    \begin{minipage}{.75\linewidth}
    $\mathbb{R}[x] = \{a_nx^n + a_{n-1}x^{n-1} + \cdots + a_0 \: : \: a_i \in \mathbb{R}, n \leq 0\}$\\\\
    Somma: se $A = [a_{ij}]$, $B = [b_{ij}] \in M_{n\times m}(\mathbb{R})$, $A + B = [a_{ij} + b_{ij}] \in M_{n\times m}(\mathbb{R})$\\\\
    Prodotto con $\lambda \in \mathbb{R}$: $\lambda A = [\lambda a_{ij}]$
    \end{minipage}
\end{figure}
\end{example}

\begin{example}
Prendiamo due funzioni continue $f, g: \mathbb{R}\to \mathbb{R}$. Possiamo effettuare le operazioni.
Somma: $(f_1 + f_2)(x) = f_1(x) + f_2(x)$ \hspace{.3cm} Prodotto con $\lambda$: $(\lambda f)(x) = \lambda \ldots f(x)$.
\end{example}
%TODO Ti sei perso, vedi di integrare

\subsection{Sottospazio}
Introduciamo ora il concetto di sottospazio vettoriale.
\begin{definition}[Sottospazio]
Sia V uno spazio vettoriale. Un \textbf{sottospazio} $W \subset V$ è un sottoinsieme tale che:
\begin{itemize}
    \item $v_1, v_2 \in W \Longrightarrow v_1 + v_2 \in W$.
    \item $v \in \mathbb{W} \Longrightarrow \lambda v \in W \:\forall \: \lambda$.
\end{itemize}
\end{definition}

\begin{proposition}
Un sottospazio $W \subset V$ è a sua volta uno spazio vettoriale.
\end{proposition}

\subsubsection{Interpretazione geometrica}
%TODO Sistema un po'
Un sottospazio vettoriale è una retta che passa per l'origine o un piano che passa per l'origine. IMPORTANTE: passa per l'origine

%TODO Cerca di renderlo più leggibile e di capirci qualcosa
\begin{example}
Dato uno spazio vettoriale:
\begin{equation*}
	V = \mathbb{R}^n = \Bigg \{\begin{bmatrix}t_1\\ t_2\end{bmatrix}: t_i \in \mathbb{R}\Bigg\}
\end{equation*}
Prendiamo un sottospazio vettoriale di $V$:
\begin{equation*}
	\Bigg \{\begin{bmatrix}t_1\\ t_2\end{bmatrix} \in \mathbb{R}^n : t_1 = 0\Bigg\} \subset \mathbb{R}^n
\end{equation*}
Un elemento generale di questo sottospazio (sottospazio con $n=2$) è: $\begin{bmatrix}0\\x_2\end{bmatrix}(x_2 \in \mathbb{R})$\\
Se prendiamo $\begin{bmatrix}0\\x_2\end{bmatrix} + \begin{bmatrix}0\\y_2\end{bmatrix} = \begin{bmatrix}0\\x_1 + x_2\end{bmatrix} \in W$ e $\lambda \cdot \begin{bmatrix}0\\x_2\end{bmatrix} = \begin{bmatrix}0\\\lambda \cdot x_2\end{bmatrix} \in W$\\\\
Similmente se prendiamo $\Bigg \{\begin{bmatrix}t_1\\ t_2\end{bmatrix} \in \mathbb{R}^n \: : \: t_2 = 0\Bigg\} \subset \mathbb{R}^n$ vettori di forma  $\begin{bmatrix}x_1\\0\end{bmatrix}$ che è un sottospazio.\\\\
Se prendiamo invece $\Bigg \{\begin{bmatrix}t_1\\ t_2\end{bmatrix} \in \mathbb{R}^n \: : \: t_1 = 1\Bigg\}$ questo non è un sottospazio perché se prendiamo il caso con $n=2$ $\begin{bmatrix}1\\x_2\end{bmatrix} + \begin{bmatrix}1\\y_2\end{bmatrix} = \begin{bmatrix}2\\x_1 + x_2\end{bmatrix}$ che non è un sottospazio.\\\\
Prendiamo ora $\Bigg \{\begin{bmatrix}t_1\\ t_2\end{bmatrix} \in \mathbb{R}^n \: : \: t_1 = t_2\Bigg\} \subset \mathbb{R}$ questo è un sottospazio perché:
se facciamo $\begin{bmatrix}x_1\\x_2\end{bmatrix} + \begin{bmatrix}x_2'\\x_1'\end{bmatrix} = \begin{bmatrix}x_1 \cdot x_2'\\x_2 + x_1'\end{bmatrix}$ e $\lambda \cdot\begin{bmatrix}x_1 \\ x_2\end{bmatrix} = \begin{bmatrix}\lambda x_1 \\ \lambda x_2\end{bmatrix}$ quindi è un sottospazio.
\end{example}

\begin{example}
Facciamo un esempio differente, prendiamo $\Bigg\{\begin{bmatrix}a & b \\ c & d\end{bmatrix} \in M_{2\times 2}(\mathbb{R})\: :\: a= 0 \Bigg\}\subset M_{2 \times 2}(\mathbb{R})$ è un sottospazio. Ma nel caso ci ci fosse stato $a=1$ non sarebbe stato un sottospazio, perché non sarebbe passato passato per (0,0).
\end{example}

\begin{example}
Facciamo alcuni esempi prendendo delle funzioni all'interno degli spazi vettoriali.
\begin{itemize}
    \item Dato $\{f \in \mathbb{R}[x] : deg(f) \leq d\} \subset \mathbb{R}[x]$ con $d$ fisso $\geq 0$. Questo è un sottospazio perché:
    \begin{itemize}
    	\item Se $deg(f_1) \leq d$, $deg(f_2) \leq d \Longrightarrow deg(f_1 + f_2) \leq d$
    	\item  Se $deg(f) \leq d \Longrightarrow deg(\lambda \cdot f) \leq d$, $\forall \lambda$
    \end{itemize}
    \item $\{f \in \mathbb{R}[x] : deg(f) = d\} \subset \mathbb{R}[x]$ non è un sottospazio per diverse ragioni:
    \begin{itemize}
    	\item Se $d>0$ allora $0 \notin W_d$
    	\item Se $d=2$ abbiamo $f = x^2 + 3 \in W$, $g = -x^2 + x + 1 \in W$ ma $f + g = x + 4 \notin W$.
    \end{itemize}
    \item $\{f \in \mathbb{R} \: : \: f(0) = d\} \subset \mathbb{R}[x]$ invece è un sottospazio perché $f(0) = 0$, $g(0)=0 \Longrightarrow (f + g)(0) = 0$ e anche $(\lambda f)(0) = 0$.
    \item $\{f \in \mathbb{R} \: : \: f(0) = 1\}$ non è un sottospazio perché non contiene 0.
    \item $\{f \in \mathbb{R} \: : \: f(2022) = 0\}$ è un sottospazio.
\end{itemize}
\end{example}

\begin{example}
Dati $a_1, a_2 \in \mathbb{R}$ fissi e dato il seguente insieme vettoriale\\\\
$\Bigg \{ \begin{bmatrix}x_1 \\ x_2\end{bmatrix}\in \mathbb{R}^2 \: : \: a_1x_1 + a_2x_2 = 0\Bigg\} \subset \mathbb{R}^n$ è un sottospazio. Perché preso
$\begin{cases}a_1x_1 + a_2x_2 \\a_1y_1 + a_2y_2 \end{cases}$\\\\
vediamo che la somma $a_1(x_1 + y_1) + a_2(x_2 + y_2) = 0$ ed anche il prodotto con $\lambda$ fa $a_1(\lambda\: x_1) + a_2(\lambda \: x_2)=0$.
\subsubsection{Generalizzazione}
Possiamo dire che, dato $a_1, a_2, \cdots, a_m \in \mathbb{R}$ fissi:\\\\
$\Bigg \{ \begin{bmatrix}x_1 \\ \vdots \\ x_m\end{bmatrix}\in \mathbb{R}^2 \: : \: a_1x_1 + a_2x_2 + \cdots + a_mx_m = 0\Bigg\} \subset \mathbb{R}^n$ è un sottospazio.\\\\Vediamo dunque che le soluzioni di un equazioni lineari omogenee a n variabili definiscono un sottospazio di $\mathbb{R}^n$. Possiamo generalizzare ulteriormente:\\\\
$\Bigg \{ \begin{bmatrix}x_1 \\ \vdots \\ x_m\end{bmatrix}\in \mathbb{R}^2 \: : \: \begin{array}{l}
    a_{11}x_1 + a_{12}x_2 + \cdots + a_{1m}x_m = 0\\
    a_{21}x_1 + a_{22}x_2 + \cdots + a_{2m}x_m = 0\\
    \cdots\\
    a_{n1}x_1 + a_{n2}x_2 + \cdots + a_{nm}x_m = 0\\
\end{array}\Bigg\} \subset \mathbb{R}^n$ Quindi è un sottospazio.\\\\
Dunque che la soluzione di un sistema di questioni lineare omogenee definisce un sottospazio $\mathbb{R}^m$.
\end{example}
\newpage
\subsection{Combinazioni lineari}
\begin{definition}[Combinazione lineare e banale]
Sia V uno spazio vettoriale e $v_1, v_2, \ldots, v_m$ vettori in $V$. Una \textbf{combinazione lineare} di $v_1, \ldots, v_m$ è una somma $\lambda v_1+ \lambda v_2 + \ldots + \lambda_m v_m \in V$, dove $\lambda_1, \lambda_2, \ldots, \lambda_m \in \mathbb{R}$. La combinazione lineare è detta \textbf{banale} se $\lambda_1 = \ldots = \lambda_m = 0$. In questo caso $\lambda_1 v_1 + \ldots + \lambda v_m = 0$.
\end{definition}

\noindent Nota che una combinazione lineare può essere $0$ ma non banale, per esempio:\\
$V = \mathbb{R}^2$, \hspace{.2cm} $v_1 = \begin{bmatrix}1\\1\end{bmatrix}$, $v_2 = \begin{bmatrix}2\\2\end{bmatrix}$, \hspace{.2cm}allora $-2v_1 + 1v_2 = 0$.

\begin{definition}[Sottospazio generato]
Siano $v_1, \ldots, v_m \in V$ vettori. Il \textbf{sottospazio generato} da $v_1, \ldots, v_m$ è $Span(v_1, v_2, \ldots, v_m) = \{\lambda_1 v_1 + \lambda_2 v_2 + \ldots + \lambda_n v_m : \lambda_{1}, \ldots, \lambda_m \in \mathbb{R}\}$. Questo rappresenta l'insieme delle combinazioni lineari.
\end{definition}

\begin{proposition}
$Span(v_1, \cdots, v_m) \subset V$ è un sottospazio.
\end{proposition}

\begin{demostration}
Bisogna verificare che $v, w \in span \Longrightarrow v + w \in span$ e $\lambda v \in span \: \forall \: \lambda$.
\end{demostration}

\begin{example}
Prendiamo $\mathbb{R}^2 = span\Big\{\begin{bmatrix}0\\1\end{bmatrix},\begin{bmatrix}1\\0\end{bmatrix}\Big\}$. $span\Big\{\begin{bmatrix}0\\1\end{bmatrix}\Big\}$ e $span\Big\{\begin{bmatrix}1\\0\end{bmatrix}\Big\}$ sono due rette.\\
Se facciamo $span\Big\{\lambda_1 \begin{bmatrix}1\\0\end{bmatrix} + \lambda_2 \begin{bmatrix}0\\1\end{bmatrix}\Big\} = \Big\{\begin{bmatrix}\lambda_1\\\lambda_2\end{bmatrix}\Big\} = \mathbb{R}^2$
\end{example}

\begin{example}
Sia $W = \Big\{\begin{bmatrix}x_1\\x_2\\x_3\end{bmatrix} \in \mathbb{R}^3 \: : \: x_1 = 0\Big\}$ abbiamo allora che:
$W = span\Big\{\begin{bmatrix}0\\1\\0\end{bmatrix}, \begin{bmatrix}0\\0\\1\end{bmatrix} \Big\} \subset \mathbb{R}^3$ quindi: $\Big \{\lambda_1 \begin{bmatrix}0\\1\\0\end{bmatrix} + \lambda_2 \begin{bmatrix}0\\0\\1\end{bmatrix}\} = \{\begin{bmatrix}0\\\lambda_1 \\ \lambda_2\end{bmatrix}\} = \{\begin{bmatrix}x_1\\x_2\\x_3\end{bmatrix}\: : \: x_1=0\}$ è un sottospazio di $\mathbb{R}^3$ ma non è uguale a $\mathbb{R}^3$, ma è più piccolo essendo un piano attraverso l'origine.
\end{example}

\newpage
\subsection{Vettori lineamenti indipendenti}
\begin{definition}
I vettori $v_1, v_2, \cdots, v_m \in V$ sono \textbf{linearmente indipendenti} se $\lambda_1 v_1 + \lambda_2 v_2 + \cdots + \lambda_m v_m = 0$ vale solo per $\lambda_1 = \cdots = \lambda_m = 0$.
\end{definition}
\hspace{-15pt}Questo vuol dire che se una combinazione lineare dei V è uguale a zero $\Longrightarrow$ la combinazione è banale.\\
Se $v_1, \cdots, v_n$ non sono indipendenti allora sono \textbf{linearmente dipendenti}.\\
$v_1, v_2, \cdots, v_m$ sono linearmente dipendenti $\Longleftrightarrow \: \exists \lambda_1, \lambda_2, \cdots, \lambda_m \in \mathbb{R}$ non tutti uguali a 0 tale che $\lambda_1 v_1 + \lambda v_2 + \cdots + \lambda_2 v_2 = 0$.

\begin{proposition}
$v_1, v_2, \cdots, v_m$ sono linearmente dipendenti $\Longleftrightarrow \: \exists \: 1 \leq i \leq n$ tale che $v_i$ è combinazione lineare dei $v_j$ per $j\neq i$.
\end{proposition}

\begin{demostration}
Se $v_1, \cdots, v_m$ sono dipendenti allora $\exists \: \lambda_1, \cdots, \lambda_m$ non tutti ugualia 0 tale che $\lambda_1 v_1 + \cdots + \lambda_n v_m = 0$. $\exists \: i \: : \: \lambda_i \neq 0$ che possiamo usare come dividendo: $\frac{\lambda_1}{\lambda_1} v_1 + \cdots + 1v_i + \cdots + \frac{\lambda_n}{\lambda_i} v_m = 0$
(mando tutti a destra) $v_i = -\frac{\lambda_1}{\lambda_i} v_1 + \cdots - \frac{\lambda_{i-1}}{\lambda_i}v_{i-1} - \frac{\lambda_{i+1}}{\lambda_i}v_{i+1} + -\frac{\lambda_n}{\lambda_i} v_m$.
Se $v_i = \lambda_1 v_1 + \cdots + \lambda_{i-1}v_{i-1} + \lambda_{i+1}v_{i+1} + \cdots + \lambda_m v_m$ allora $ \lambda_1 v_1 + \cdots + \lambda_{i-1}v_{i-1} - v_i + \lambda_{i+1}v_{i+1} + \cdots + \lambda_m v_m = 0$
\end{demostration}
\hspace{-15pt}In pratica per vedere se m vettori $v_1, \cdots, v_m \in \mathbb{R}^n$ sono linearmente indipendenti prendiamo innanzitutto m vettori:\\
$v_1 = \begin{bmatrix}a_11 \\ a_21 \\ \vdots \\ a_{n1}\end{bmatrix}$, $v_2 = \begin{bmatrix}a_{12}\\a_{22}\\\vdots\\ a_{n2}\end{bmatrix}$, $\cdots, v_m = \begin{bmatrix}a_{1m}\\a_{2m}\\\vdots\\ a_{nm}\end{bmatrix}$, questi sono vettori di $\mathbb{R}^m$.\\\\
Questi vettori sono linearmente indipendente, quindi l'equazione $\lambda_1v_1 + \lambda_2 v_2 + \cdots + \lambda_mv_m = 0$ vale, se e solo se ($\lambda_1, \cdots, \lambda_m$) è soluzione del sistema:
\[
\begin{array}{l}
     a_{11}x_1 + a_{12}x_2 + \cdots + a_{1m}x_m = 0\\
     a_{21}x_1 + a_{22}x_2 + \cdots + a_{2m}x_m = 0\\
     \cdots\\
     a_{n1}x_1 + a_{n2}x_2 + \cdots + a_{nm}x_m = 0\\
\end{array}
\hspace{.5cm}
\parbox{6cm}{Quindi $v_1, \cdots, v_m$ sono lin. indipendenti $\Longrightarrow$ il sistema sopra ammette solo la soluzione banale (0,$\cdots$,0)}
\]
\subsection{Interpretazioni geometrica}
Facciamo un interpretazione geometrica di quello visto sopra ponendo $n=2$. $V = \mathbb{R}^2$, $v_1, v_2 \in \mathbb{R}^2$ sono linearmente dipendenti $v_1, v_2 \neq 0$ oppure $\exists \: \lambda_1, \lambda_2 \: : \: \lambda_1 v_1 + \lambda_2 v_2 = 0$. Ad esempio $\lambda \neq 0$, $v_1 = -\frac{\lambda_1}{\lambda_2}v_2$ e se $v_2 = \begin{bmatrix}1\\0\end{bmatrix} \Longrightarrow v_1 = \begin{bmatrix}-\lambda_2\setminus\lambda_1\\0\end{bmatrix}$ e corrisponde un punto della retta $x_2 = 0$. \\
In generale se $v_1 = \begin{bmatrix}x_1\\x_2\end{bmatrix}$, $v_2$ deve essere $\begin{bmatrix}\lambda x_1 \\ \lambda x_2\end{bmatrix}$ quindi $v_1, v_2$ sono lin. dipendenti $\Longleftrightarrow$ i punti corrispondenti sono sulla tessa retta attraverso (0,0).
\begin{example}
Si decida se i seguenti vettori di $\mathbb{R}^2$ sono linearmente indipendenti:\\
$v_1 = \begin{bmatrix}1\\2\\3\end{bmatrix}$\hspace{.5cm}$v_2 = \begin{bmatrix}1\\0\\1\end{bmatrix}$\hspace{.5cm}$v_3 = \begin{bmatrix}0\\0\\1\end{bmatrix}$\hspace{.5cm}$v_4 = \begin{bmatrix}2\\2\\4\end{bmatrix}$.\\\\
Per faro dobbiamo cercare le soluzioni del sistema lineare omogeneo con la matrice associata.
\[
\begin{array}{l}
    x_1 + x_2 + 2x_4 = 0\\
    2x_1 + 2x_4 = 0\\
    3x_1 + x_2 + x_3 + 4x_4
\end{array}
\Rightarrow
\begin{bmatrix}
1 & 1 & 0 & 2\\
2 & 0 & 0 & 2\\
3 & 1 & 1 & 4
\end{bmatrix}
\begin{array}{l}
\text{Algoritmo di gauss}\\
R_2 = R_2 - 2R_1\\
R_3 = R_3 - 3R_1
\end{array}
\Rightarrow
\begin{bmatrix}
1 & 1 & 0 & 2\\
0 & -2 & 0 & -2\\
0 & -2 & 1 & -2
\end{bmatrix}
R_3 = R_3 - R_2
\]
\[
\Rightarrow
\begin{bmatrix}
1 & 1 & 0 & 2\\
0 & -2 & 0 & -2\\
0 & 0 & 1 & 0
\end{bmatrix}
\parbox{6cm}{In questo caso ci sono 3 pivot, una variabile libera $\Longrightarrow \infty$ soluzioni}
\]
Quindi il sistema ammette soluzioni non banali $\Longrightarrow$ i vettori sono lim. dipendenti.\\ 
Se si guardasse solo $v_1, v_2, v_3$ quello che risulterebbe sarebbe una matrice $3 \times 3$ con 3 pivot, in questo caso allora ci sarebbe solo la soluzione banale ed allora $v_1, v_2, v_3$ sarebbero lin. indipendenti.
\end{example}

\begin{proposition}
Se $v_1, v_2, \cdots, v_n \in V$ sono vettori tali che $v_n$ è combinazione lineare di allora: $span(v_1, v_2, ..., v_n) = span(v_1, v_2, \cdots, v_{n-1})$.
\end{proposition}

\subsection{Base di un sistema lineare}
\begin{definition}
Un sistema $v_i, \cdots, v_n$ di vettori è una \textbf{base} di V se i vettori $v_i, \cdots, v_n$:
\begin{itemize}
    \item Sono linearmente indipendenti.
    \item Lo $span(v_1, v_2, \cdots, v_n) = V$
\end{itemize}
\end{definition}

\begin{corollary}
Se $span(v_1, \cdots, v_n) ) C$ si può scegliere una base di V fra i $v_1, \cdots, v_n$.
\end{corollary}

\begin{example}
Vogliamo trovare la base standard di $\mathbb{R}^n$.\\
$e_1 = \begin{bmatrix}1\\0\\\vdots\\0\end{bmatrix}, e_2 = \begin{bmatrix}0\\1\\0\\\vdots\\0\end{bmatrix}, \cdots, e_n = \begin{bmatrix}0\\0\\\vdots\\0\\1\end{bmatrix}$
Possiamo osservare che $\begin{bmatrix}\lambda_1\\\lambda_2\\\vdots\\\lambda_n\end{bmatrix} = \lambda_1 e_1 + \lambda_2 \_2 + \cdots + \lambda_ne_n$,\\\\
dunque $span(e_1, \cdots, e_n) = \mathbb{R}^n$ e $\lambda_1e_1 + \cdots + \lambda_ne_n = 0$ se e solo se $\lambda_1= \cdots = \lambda_n = 0$. Ma questa non è l'unica base, c'è ne sono tante, ad esempio se prendiamo $n=2$.\\\\
$\begin{bmatrix}1\\0\end{bmatrix}, \begin{bmatrix}1\\1\end{bmatrix}$ è una base perché $\lambda_1\begin{bmatrix}1\\0\end{bmatrix} + \lambda_2\begin{bmatrix}1\\1\end{bmatrix} = \begin{bmatrix}\lambda_1 + \lambda_2\\\lambda_2\end{bmatrix} = \begin{bmatrix}0\\0\end{bmatrix}$ se e solo se $\lambda_1 + \lambda_2 = \lambda_2 = 0 \Longleftrightarrow \lambda_1 = \lambda_2 = 0$
\end{example}

\begin{example}
Troviamo la base standard di $M_{2\times2}(\mathbb{R})$.
\[\begin{bmatrix}1&0\\0&0\end{bmatrix}, \hspace{.5cm} \begin{bmatrix}0&1\\0&0\end{bmatrix}, \hspace{.5cm} \begin{bmatrix}0&0\\1&0\end{bmatrix}, \hspace{.5cm} \begin{bmatrix}0&0\\0&1\end{bmatrix}\]
Si applica lo stesso ragionamento visto sopra con $\mathbb{R}^n$.
\end{example}

\begin{example}
Base standard di $\mathbb{R}[x]_{\leq d} = \{f \in \mathbb{R}[x] \::\: deg(x) \leq d\}$ sarebbe $1, x, x^2, \cdots, x^s$. Infatti, $a_dx^d + a_{d-1}x^{d-1} + \cdots + a_0 = a_0 \cdot 1 + a_1 \cdot x + \cdots + a_d \cdot x^d$ è il sistema indipendente
\end{example}
\begin{example}
Prendiamo $\mathbb{R}[x]$ che non ammette di base finita. Infatti $\nexists f_1, \cdots, f_n \in \mathbb{R}[x] \::\: span(f_1, \cdots, f_n = \mathbb{R}[x]$ perché se $f \in span(f_1, \cdots, f_n)$ allora $deg(f) \leq max(deg(f_1), \cdots, deg(f_n))$. (Comunque è vero: $span(1,x,x^2, x^2, \cdots) = \mathbb{R}[x]$ e ogni sottoinsieme finito di $1,x, x^2, \cdots$ è lin. indipendente)
\end{example}

La dimensione di uno spazio V sarà definita come il numero degli elementi di una base. Per questo bisogna sapere: questo numero è lo stesso per ogni base.
\begin{proposition}
Sia V uno spazio vettoriale che ammette una base $e_1, e_2, \cdots, e_n$. Se $v_1, v_2, \cdots, v_n \in V$ e $r > n \Longrightarrow v_1, v_2, \cdots, v_n$ sono linearmente dipendenti.
\end{proposition}

\begin{demostration}
Per $n=2$, la prima osservazione è che se la proposizione vale per $r =2$ vale per ogni $r > 2$. Infatti se $\lambda_1v_2 + \lambda_2v_2 + \lambda_3v_3 = 0$ è una combinazione lineare non banale allora $\lambda_1v_2 + \lambda_2v_2 + \lambda_3v_3 + 0 \cdot v_4 + 0 \cdot v_5 + \cdots + 0 \cdots v_r = 0$ è una combinazione non banale (perché $\lambda_1, \lambda_2$ o $\lambda_3$ è diverso da 0). Quindi siano $n=2, r=3$ e $e_1, e_2$ una base di V. Come $V=span(e_1,e_2)$, $v_1, v_2, v_3 \in span(e_1, e_2)$. Quindi:\\\\
$\begin{array}{l}
v_1 = a_{11}e_1 + a_{12}e_2\\
v_2 = a_{21}e_1 + a_{22}e_2\\
v_3 = a_{31}e_1 + a_{32}e_2
\end{array}
\hspace{.5cm}
\parbox{8cm}{Dobbiamo trovare $\lambda_1, \lambda_2, \lambda_3$ non tutti $=0$ tali che $\lambda_1v_1 + \lambda_2v_2 + \lambda_3v_3 = 0$. Facciamo la sostituzione con il sistema a fianco.}$\\\\
$\lambda_1(a_{11}e_1 + a_{12}e_2) + \lambda_2(a_{21}e_1+a_{22}e_2) + \lambda_3(a_{31}e_1 + a_{32}e_2) = 0$ che diventa $(\lambda_1a_{11} + \lambda_1a_{12})e_2 + (\lambda_2a_{21} + \lambda_2a_{22})e_2 + (\lambda_3a_{31} + \lambda_3a_{32})e_2 = 0$. Ma $e_1, e_2$ sono linearmente indipendenti, quindi:\\\\
$\begin{array}{l}
\lambda_1a_{11} + \lambda_2a_{21} + \lambda_3 a_{31} = 0\\
\lambda_1a_{12} + \lambda_2a_{22} + \lambda_3 a_{32} = 0
\end{array}
\hspace{.3cm}
\parbox{5cm}{Questo è un sistema omogeneo di equazioni per $\lambda_1, \lambda_2, \lambda_3$ con matrice di coefficienti}\hspace{.3cm}
\begin{bmatrix}
a_{11}&a_{21}&a_{31}\\
a_{12}&a_{22}&a_{32}
\end{bmatrix}$\\
Se facciamo l'algoritmo di Gauss, ottengo un numero di pivot minore o uguale a 2 (perché ci sono solo due righe), allora ci sarà $\geq 1$ colonne senza pivot ed allora il sistema avrà $\infty$ soluzioni ed allora ci sarà una soluzione non banale $\lambda_1, \lambda_2, \lambda_2$. Ma se il sistema sopra ha una soluzione non banale ($\lambda_1, \lambda_2, \lambda_2$) allora anche $\lambda_1v_1 + \lambda_2 v_2 + \lambda_3 v_3 = 0$ sarà una combinazione non banale e quindi ci siamo.\\\\
La dimostrazione per $n,r$ generale è la stessa, infatti alla fine ottengo un sistema lineare di n equazioni in $r > n$ variabili ed allora c'è sempre una soluzione non banale. $\blacksquare$
\end{demostration}

\begin{corollary}
Se $v_1, \cdots, v_r$ ed $e_1, \cdots, e_n$ sono due basi di V allora $r=n$.
\end{corollary}

\begin{demostration}
Se $r>n$, $v_1, \cdots, v_r$ è linearmente dipendente dopo la proposizione se $e_1, \cdots, e_n$ è una base, quindi $r\leq n$. Se $r<n$ e $v_1, \cdots, v_n$ è una base allora $e_1, \cdots, e_n$ è linearmente dipendete e questa è una contraddizione. Dunque $r=n$. $\blacksquare$
\end{demostration}

\subsection{Dimensione spazio vettoriale}
\begin{definition}[Dimensione di un V]
Se V ammette una base $e_1, \cdots, e_n$ n è la dimensione di V. La dimensione di V si indica come $dim V = n$.
\end{definition}

\begin{corollary}
Se la dimensione di V è n e $v_1, \cdots, v_m$ sono vettori lin. indipendenti con $m<n \Longrightarrow \:\exists\: w_{m+1}, w_{m+2}, \cdots, w_n \::\: v_1, \cdots, v_n, w_{m+1}, \cdots, w_n$ sono una base di V. 
\end{corollary}

\begin{demostration}
Dobbiamo verificare che $span(v_1, \cdots, v_m) = V$. Sappiamo che $span(v_1, \cdots, v_m)$ non può essere V, allora $\exists\: v \in V$ tale che $v \notin span(v_1, \cdots, v_m)$ ma allora basta vedere che $v_1, v_2, \cdots, v_m$ sono linearmente indipendenti ed allora abbiamo una contraddizione.\\
Sia $\lambda_1v_1 + \lambda_2v_2 + \cdots + \lambda_mv_m + \lambda_0v_0 = 0$ una combinazione lineare se $\lambda=0 \Longrightarrow \lambda_1 = \lambda_2 = \cdots = \lambda_n = 0$ perché $v_1, v_n \cdots, v_m$ lin. indipendenti allora $v_1, v_2, \cdots, v_m$ lin. indipendenti. Se prendiamo un $\lambda_{m+1}\neq 0$ possiamo fare $-\frac{\lambda_1}{\lambda}v_1 - \frac{\lambda_2}{\lambda}v_2 - \cdots - \frac{\lambda_n}{\lambda}v_n = V \in span(v_1, \cdots, v_n)$, ma questa è una contraddizione $v \notin span$. $\blacksquare$
\end{demostration}

\begin{example}
Decidiamo se $\begin{bmatrix}1\\0\\1\end{bmatrix}, \begin{bmatrix}2\\0\\0\end{bmatrix}, \begin{bmatrix}3\\2\\1\end{bmatrix}$ sono una base $\mathbb{R}^3$. \\Per faro dobbiamo solo decidere se sono indipendenti o meno, e per farlo usiamo gauss.
\[
\begin{bmatrix}
1 & 2 & 3\\
0 & 0 & 1\\
1 & 0 & 1
\end{bmatrix}
R_3 - R_2 \Rightarrow
\begin{bmatrix}
1 & 2 & 3\\
0 & 0 & 1\\
0 & -2 & 1
\end{bmatrix}
\xRightarrow[]{\text{Scambio} R_2, R_3}
\begin{bmatrix}
1 & 2 & 3\\
0 & -2 & 1\\
0 & 0 & 1
\end{bmatrix}
\]
Vediamo dunque che ci sono 3 pivot e quindi i vettori sono lineamenti indipendenti e di conseguenza abbiamo una base.
\end{example}

\begin{example}
Prendiamo $V = \mathbb{R}[x]_{\leq 2}$. Abbiamo visto che $1, x, x^2$ sono una base questo vuol dire allora che $\mathbb{R}[x]_{< 2} = 2$. Vediamo se $1, 1+x, (1+x)^2$ forma una base. Per fare questo bisogna vedere se sono linearmente indipendenti. Supponiamo che:
$\lambda_11 + \lambda_2(1+x) + \lambda_3(1+x)^2 = 0 \Rightarrow \lambda_1 + \lambda_2 + \lambda_2x + \lambda_3x^2 + 2x\lambda_3 + \lambda_3 = 0 \Rightarrow (\lambda_1 + \lambda_2 + \lambda_3)1 + (\lambda_2 +\lambda_3)x + \lambda_3 + x^2 = 0$.\\
Visto che $1, x, x^2$ sono lin. indipendenti allora $\lambda_1 + \lambda_2 + \lambda_3 = 0$, $\lambda_2 + 2\lambda_3 = 0$, $\lambda_3 = 0$ sostituendo viene che $\lambda_2 = 0, \lambda_1 = 0$ allora $1, 1+x, (1+x)^2$ sono lin. indipendenti ed allora sono una base di $\mathbb{R}(x)_{<2}$.
\end{example}

\begin{proposition}
Sia $v_1, \cdots, v_n$ una base di V, e $v \in V$ un vettore. Allora $\exists ! \: \alpha_1, \cdots, \alpha_n \in \mathbb{R} \::\: v = \alpha_1v_1 + \alpha_2v_2 + \cdots + \alpha_nv_n$. (Ogni vettore si scrive in modo unico come combinazione lineare degli elementi della base)
\end{proposition}

\begin{demostration}
Scriviamo come $V = span(v_1, \cdots, v_n)$, l'esistenza degli $\alpha_i$ è chiaro. Se adesso $v = \alpha_1v_i + \cdots + \alpha_nv_n = \beta_1v_1 + \cdots + \beta_nv_n$ allora $0 = (\alpha_1 - \beta_1)v_1 + \cdots + (\alpha_n - \beta)v_n$ allora $\alpha_1 = \beta_1, \cdots, \alpha_1 = \beta_n$ perché i $v_i$ sono lin. indipendenti.
\end{demostration}

\begin{example}
$W' = \{f \in \mathbb{R}[x]_{\leq 2} \::\: f(1) = f(2) = 0\}$, $W' \subset W$ (per esempio visto prima) e $dim(W) = 3 \Longrightarrow dim(W') \leq 2$. Ci sono due vettori indipendenti in $W$, $(x-1)(x-2), (x-1)(x-2)^2$ di gradi diversi $\Longrightarrow dim(W') = 2$.\\
$W' \subset W$ in base $W'$ è $(x-1)(x-2), (x-1)(x-2)^2$ e completiamo in una base di $W$ $(x-1), (x-1)(x-2), (x-1)(x-2)^2 \in W \setminus W'$ è una base di $W$, perché sono indipendenti:\\
$W \subset V$, $dim(V) = 4$, $dim(W) = 4$. $1, x-1, (x-1)(x-2), (x-1)(x-2)^2$ è una base di $V \setminus W$.
\end{example}

\begin{example}
$V = M_{3\times3}(\mathbb{R})$, la dimensione è $dim(V) = 9$. Mentre $W \subset \{ A \in M_{3\times 3}(\mathbb{R}): \text{ la somma di ogni riga è 0}\}$. Supponiamo $dim(W) < 9$ elementi linearmente indipendenti di W.
\[
\begin{bmatrix}
1 & -1 & 0\\
0 & 0 & 0\\
0 & 0 & 0
\end{bmatrix}
\begin{bmatrix}
1 & 0 & -1\\
0 & 0 & 0\\
0 & 0 & 0
\end{bmatrix}
\begin{bmatrix}
0 & 0 & 0\\
1 & -1 & 0\\
0 & 0 & 0
\end{bmatrix}
\begin{bmatrix}
0 & 0 & 0\\
1 & 0 & -1\\
0 & 0 & 0
\end{bmatrix}
\begin{bmatrix}
0 & 0 & 0\\
0 & 0 & 0\\
1 & -1 & 0
\end{bmatrix}
\begin{bmatrix}
0 & 0 & 0\\
0 & 0 & 0\\
1 & 0 & -1
\end{bmatrix}
\]
Essendo linearmente indipendente allora $dim(W) \geq 6$. Proviamo a dire che $dim(W) = 6$. L'idea:\\
$W_1 = \{A \in M_{3 \times 3}(\mathbb{R}) : \text{ la somma della prima riga = 0}\}$,\\ $W_1 = \{A \in M_{3 \times 3}(\mathbb{R}) : \text{ la somma della prima e della seconda riga = 0}\}$.\\
$W \subset W_2 \subset W_1 \subset M_{3\times3}(\mathbb{R})$, sapendo $dim(M_{3\times3}(\mathbb{R})) = 9, dim(W_1) = 6, dim(W_2) = 7, dim(W) = 7$.
\end{example}

\begin{example}
Sappiamo già: $v_1 = \begin{bmatrix}1\\2\\3\end{bmatrix}, v_2 = \begin{bmatrix}1\\0\\1\end{bmatrix}, v_3 = \begin{bmatrix}0\\0\\1\end{bmatrix}$ che sono una base di $\mathbb{R}^3$.
Troviamo le coordinate di $\begin{bmatrix}0\\2\\1\end{bmatrix}$ rispetto a queste basi.
$\alpha_1\begin{bmatrix}1\\2\\3\end{bmatrix} + \alpha_2\begin{bmatrix}1\\0\\1\end{bmatrix} + \alpha_3\begin{bmatrix}0\\0\\1\end{bmatrix} = \begin{bmatrix}0\\2\\1\end{bmatrix}$. Ed usiamo Gauss-Jordan.
\[
\begin{bmatrix}
1 & 1 & 0 & 0\\
2 & 0 & 0 & 2\\
3 & 1 & 1 & 1
\end{bmatrix}
\begin{array}{l}
    R_2 - 2R_1\\
    R_3 - 3R_1
\end{array}
\Rightarrow
\begin{bmatrix}
1 & 1 & 0 & 0\\
0 & -2 & 0 & 2\\
0 & -2 & 1 & 1
\end{bmatrix}
\begin{array}{l}
    R_3 - R_2\\
    R_2 = \frac{1}{2}R_1
\end{array}
\Rightarrow
\begin{bmatrix}
1 & 1 & 0 & 0\\
0 & 1 & 0 & -1\\
0 & 0 & 1 & -1
\end{bmatrix}
R_1 - R_2
\Rightarrow
\begin{bmatrix}
1 & 0 & 0 & 1\\
0 & 1 & 0 & -1\\
0 & 0 & 1 & -1
\end{bmatrix}
\]
Abbiamo quindi $\alpha_1 = 1, \alpha_2 = -1, \alpha_3 = -1$.
\end{example}

\begin{example}
Vedere se $\begin{bmatrix}1\\0\\1\end{bmatrix}, \begin{bmatrix}2\\0\\0\end{bmatrix}, \begin{bmatrix}3\\1\\1\end{bmatrix}$ è una base $\mathbb{R}^3$
e calcolare coordinate $\begin{bmatrix}2\\0\\4\end{bmatrix}$ rispetto a base. \\
Per calcolare le coordinate dobbiamo risolvere il sistema lineare che si crea con le 3 matrici:
\[
\begin{array}{l}
x_1 + 2x_2 + 3x_3 = 2\\
x_3 = 0\\
x_1 + x_3 = 4
\end{array}
\hspace{.3cm}
\begin{array}{r}
x_3 = 0\\
x_1 = 4\\
x_2 = -1
\end{array}
\hspace{.5cm}
\parbox{8cm}{Usiamo Gauss per verificare l'indipendenza perché se questi vettori sono indipendenti allora i coefficienti $4, -1, 0$ saranno le coordinate.}
\]
\[
\begin{bmatrix}
1 & 2 & 3\\
0 & 0 & 1\\
1 & 0 & 1
\end{bmatrix}
R_3 - R_1
\begin{bmatrix}
1 & 2 & 3\\
0 & 0 & 1\\
0 & -2 & -2
\end{bmatrix}
\text{Inverto } R_2, R_3
\begin{bmatrix}
1 & 2 & 3\\
0 & -2 & -2\\
0 & 0 & 1
\end{bmatrix}
\]
Torna $x_1 = 4, x_2 = -1, x_3 = 0$, inoltre abbiamo una forma a scalini con 3 pivot allora i vettori sono indipendenti e quindi sono una base.
\end{example}

\begin{proposition}
Se abbiamo uno spazio vettoriale $dim(V) = n$ ed abbiamo $v_1, v_2, \cdots, v_m$ vettori linearmente indipendenti di V con $m < n$ allora $\exists \: w_{m+1}, w_{m+2}, \cdots, w_n \::\: v_1, \cdots, v_m, w_{m+1}, \cdots, w_n$ sono una base di V.
\end{proposition}

\begin{demostration}
$spa(v_1 \cdots, v_m)$ non può essere V, perché se $span(v_1, \cdots, v_m) = V \Longrightarrow v_1, \cdots, v_m$ è una base ma $m<n$ è $dim(V) = n$ e questa è una contraddizione. Quindi $span(v_1,\cdots,v_m) \neq V \Longrightarrow \exists w_{m+1} \in V \::\: w_{m+1} \notin span(v_1, \cdots, v_m)$. Ma allora $v_1, \cdots, v_{m}, w_{m+1}$ sono linearmente indipendenti tale che se $\lambda_1v_1 + \cdots + \lambda_{m}v_m + \lambda_{m+1}w_{m+1} = 0$, $\lambda_{m+1} = 0 $ allora $\lambda_1 = \cdots = \lambda_m  = 0$. Se $\lambda_{m+1} \neq 0$ allora $v_{m+1} = (-\frac{\lambda_1}{\lambda_{m+1}})v_1 + \cdots + (\frac{-\lambda_m}{\lambda_{m+1}})w_m \in span(w_1, \cdots, v_m)$ che è una contraddizione.
\end{demostration}

\hspace{-15pt}Per ricapitolare se la $dim(V) = n$ e $v_1, \cdots, v_n$ sono vettori di V possiamo dire che:
\begin{itemize}
    \item Se $m > n$ allora i vettori sono linearmente dipendenti.
    \item Se $m = n$ e i vettori sono indipendenti allora si forma una base.
    \item Se $m < n$ e i vettori sono indipendenti allora si completa in una base di V.
\end{itemize} 

\newpage
\begin{example}
Facciamo un esercizio che sarà suddiviso in due parti.
\begin{itemize}
    \item Decidiamo se i vettori $\begin{bmatrix}1\\-1\\1\end{bmatrix}, \begin{bmatrix}-1\\1\\-1\end{bmatrix}, \begin{bmatrix}-1\\-1\\1\end{bmatrix}$ sono una base di $\mathbb{R}^3$. $dim(\mathbb{R}^3) \Longrightarrow$ se sono indipendenti sono allora una base. Usiamo gauss.
    \[
    \begin{bmatrix}
    1 & -1 & -1\\
    -1 & 1 & -1 \\
    1 & -1 & 1
    \end{bmatrix}
    \begin{array}{l}
        R_2 + R_1\\
        R_3 - R_1
    \end{array}
    \Rightarrow
    \begin{bmatrix}
    1 & -1 & -1\\
    0 & 0 & -2 \\
    0 & 0 & 2
    \end{bmatrix}
    R_2 + R_3
    \begin{bmatrix}
    1 & -1 & -1\\
    0 & 0 & -2 \\
    0 & 0 & 0
    \end{bmatrix}
    \]
    Risulta avere 2 pivot e quindi i vettori sono dipendenti. I pivot però sono delle colonne 1 e 3 e quindi se escludiamo la colonna centrale abbiamo come risultato due vettori indipendenti che chiamiamo $v_1, v_2$.
    \item Ora come secondo punto dobbiamo completare $v_1, v_2$ in una base di $\mathbb{R}^3$. Per fare questo dobbiamo trovare un terzo vettore non contenente in $span(v_1, v_2)$.\\
    L'idea qui è che so che $v_1 = \begin{bmatrix}1\\0\\0\end{bmatrix}, e_2 = \begin{bmatrix}0\\1\\0\end{bmatrix}, e_3 = \begin{bmatrix}0\\0\\1\end{bmatrix}$ è la base standard. So anche che almeno uno di questi 3 vettori non è contenuto in $span(v_1, v_2)$ perché se $e_1, e_2, e_3 \in span(v_1, v_2) \Longrightarrow span(e_1, e_2, e_3) \subset span(v_1, v_2)$ ma $span(e_1, e_2, e_3) = \mathbb{R}^3$ e quindi abbiamo una contraddizione. A questo punto devo trovare quale dei 3 vettori non è in $span(v_1, v_2)$. Lo facciamo provando i vari vettori e trovano quello che utilizzando Guass faccia venire 3 pivots.
    \[
    \begin{bmatrix}
        1 & -1 & 0\\
        -1 & -1 & 0\\
        1 & 1 & 1
    \end{bmatrix}
    \begin{array}{l}
        R_2 + R_1\\
        R_3 - R_1
    \end{array}
    \Rightarrow
    \begin{bmatrix}
        1 & -1 & 1\\
        0 & -2 & 0\\
        0 & 2 & 1
    \end{bmatrix}
    R_3 - 2R_2
    \Rightarrow
    \begin{bmatrix}
        1 & -1 & 1\\
        0 & -2 & 1\\
        0 & 0 & 1
    \end{bmatrix}
    \]
    Il risultato in questo caso è 3 pivot quindi i vettori sono linearmente indipendenti e quindi è una base di $\mathbb{R}^3$.
\end{itemize}
\end{example}

\begin{proposition}
Sia $W \subset V$ un sottospazio. Allora:
\begin{enumerate}
    \item Abbiamo che $dim(W) \leq dim(v)$.
    \item E se $W \neq V$ allora $dim(W) < dim(v)$.
\end{enumerate}
\end{proposition}

\begin{demostration}
Per dimostrare questa proposizione bisogna andare a dimostrare i due punti separatamente.
\begin{enumerate}
    \item Se $r = dim(W)$ e $w_1, \cdots, w_r$ è una base di W allora se $r > n$ allora per una proposizione vista precedentemente $w_1, \cdots, w_r$ sarebbero linearmente dipendenti e questa è una contraddizione quindi $r\leq n$.
    \item se $r = n$, $w_1, \cdots, w_r$ sono $n = r$ vettori lineamenti indipendenti di V ed allora sono una base di V quindi $span(w_1, \cdots, w_r) = V \Longrightarrow V = W$.
\end{enumerate}
\end{demostration}

\begin{example}
Sia $V = M_{2 \times 2}(\mathbb{R}), W = \{\begin{bmatrix}a & b \\ c & d\end{bmatrix} \in M_{2 \times 2}(\mathbb{R}) \::\: b = c\}$ (questa è definita anche matrice simmetrica).\\
Si calcoli da dimensione di W, $dim(W)$. Partiamo dal fatto che la $dim(M_{2 \times 2}(\mathbb{R})) = 4$ (basi standard). Mentre $V \neq M_{2 \times 2}(\mathbb{R})  \Longrightarrow dim(V) \leq 3$. Vediamo però che:
\[
\begin{bmatrix}1 & 0 \\ 0 & 0\end{bmatrix}
\begin{bmatrix}0 & 0 \\ 0 & 1\end{bmatrix}
\begin{bmatrix}0 & 1 \\ 1 & 0\end{bmatrix}
\hspace{.3cm}
\parbox{6cm}{Sono linearmente indipendenti quindi devono essere una base di W e quindi dim(W) = 3}
\]
\end{example}

\begin{example}
Sia $W = \{f \in \mathbb{R}[x\ \::\: deg(f) \leq 3, f(1) = 0\}$ sottospazio di $V = \mathbb{R}[x]_{\leq 3}$. Sappiamo che la $dim(V) = 4$ (1, x, $x^2, x_3$). La proposizione mi dice che $dim(W) \leq 3$ e se trovo 3 vettori indipendenti allora $dim(W) = 3$. Possiamo vedere che $x-1, x^2-1, x_3-1$ sono lin. indipendenti quindi concludiamo che $dim(W) = 3$.
\end{example}

\begin{observation}
Se V è un spazio, $V_1, V_2 \subset V$ sottospazi allora anche $V_1 \cup V_2$ è un sottospazio. Infatti se $v \in V_1 \cup V_2$ e $w \in V_1 \cap V_2 \Longrightarrow v+w V_1 \cap V_2$ perché $V_1$ sottospazio ed allora $v + w \in V_1$ ed allora in modo simile per $V_2 \Longrightarrow v + w \in V_2$ ed in modo simile $\lambda v \in V_1, \lambda v \in V_2 \Longrightarrow \lambda v \in V_1 \cap V_2 \forall \: \lambda \in \mathbb{R}$.
\end{observation}

\begin{example}
Sia $W_i \subset \mathbb{R}^n$ il sottospazio delle soluzioni dell'equazione omogenea $E_i \:: a_{i1}x_1 + \cdots + a_{in}x_n = 0$. Allora $W_1 \cap W_2 \cap \cdots \cap W_r$ è il sottospazio delle soluzioni comuni di $E_1, E_2, \cdots, E_r$.
\end{example}

\subsection{Formula di Grassman}
\begin{definition}[Somma fra sottospazi]
Siano $V_1, V_2 \subset V$ due sottospazi. La loro somma di $V_1, V_2$ è definita come:
\[V_1 + V_2 = \{v_1 + v_2 : v_1 \in V_1, v_2 \in V_2\}\]
\end{definition}

\begin{observation}
Si osservi che $V_1 + V_2 \subset V$ è un sottospazio a sua volta.
\end{observation}

\begin{demostration}
Questa definizione si somma fra sottospazi è vera perché se $v, w \in V_1 + V_2$ allora:
\[
\begin{rcases}
v = v_1 + v_2 & \text{ con } (v_i \in V_i)\\
w = w_1 + w_2 & \text{ con } (w_i \in V_i)
\end{rcases}
\Rightarrow v+w = (v_1 + w_1) + (v_2 + w_2) \in V_1 + V_2 \text{ perché }(v_1 + w_1) \in V_1, (v_2 + w_2) \in V_2
\]
Se $\lambda \in \mathbb{R}$, $\lambda v = \lambda(v_1 + v_2) = \lambda v_1 + \lambda v_2 \in V_1 + V_2$ con $\lambda v_1 \in V_1$ e $\lambda v_2 \in V_2$. $\blacksquare$
\end{demostration}

\begin{proposition}
Se $v_1, v_2, \cdots, v_n \in V_1 + V_2 \Longrightarrow span(v_1, \hdots, v_n) \subset V_1 + V_2$.
\end{proposition}

\begin{demostration}
La dimostrazione è abbastanza veloce, infatti basta vedere che se $V_1 + V_2$ è un sottospazio che contiene $v_1, \cdots, v_n$ allora contiene le loro combinazioni. lineari.
\end{demostration}

\begin{example}
Dato un $V_1 = \{\begin{bmatrix}a\\0\end{bmatrix}\::\: a \in \mathbb{R}\}, V_2 = \{\begin{bmatrix}0\\b\end{bmatrix}\} \::\: b \in \mathbb{R} \subset \mathbb{R}^2$ sottospazi.\\
Allora $V_1 + V_2 = \{\begin{bmatrix}a\\b\end{bmatrix}\::\: a,b \in \mathbb{R}\} = \mathbb{R}^2$
\end{example}

\begin{example}
Dati $V_1 = \{\begin{bmatrix}0\\a_2\\a_3\end{bmatrix}\::\: a_2, a_3 \in \mathbb{R}\}$, $V_2 = \{\begin{bmatrix}a_1\\a_2\\0\end{bmatrix}\::\: a_1, a_2 \in \mathbb{R}\} \subset \mathbb{R}^3$.\\
Allora $V_1 + V_2 = \{\begin{bmatrix}a_1\\a_2 + a_2\\a_3\end{bmatrix} \::\: a_1, a_2, a_3 \in \mathbb{R}\} = \mathbb{R^3}$ ma anche $V_1 \cap V_2 = \{\begin{bmatrix}0\\a_2\\0\end{bmatrix} \::\: a_2 \in \mathbb{R}\}$
\end{example}

\begin{theorem}[Formula di Grassman]
Sia $dim(V) < \infty$, $V_1, V_2 \subset V$ sottospazi allora:
\[dim(V_1 + V_2) = dim(V_1) + dim(V_2) - dim(V_1 \cap V_2)\]
\end{theorem}

\begin{demostration}
Per dimostrare questa formula sia $e_1, \cdots, e_4$ una base di $V_1 \cap V_2$. Si completa in una base $e_1, \cdots, e_r, v_{r+1}, \cdots, v_n$ di $V_1$ e $e_1, \cdots, e_r, w_{r+1}, \cdots, w_m$ di $V_2$. \\
Quindi abbiamo he $dim(V_1) = n, dim(V_2) = m$ e che $V_1 \cap V_2 = r$. A questo punto verifichiamo che $e_1, \cdots, e_r, v_{r+1}, \cdots, v_n, w_{r+1}, \cdots, w_m$ è una base di $V_1 + V_2$. Se fosse una base allora $dim(V_1 + V_2) = n + m - r$.
Per verificare se è una base verifichiamo se è lin indipendente, sia:\\
$\lambda_1 e_1 + \cdots + \lambda_r e_r + \mu_2 v_{r+1} + \cdots + \mu_{n-r}v_n + \nu_1 w_{r+1} + \cdots + \nu_{m-r}w_m = 0$. Tutti i coefficienti $\lambda_i, \mu_i, \nu_i \in \mathbb{R}$, dobbiamo ora vedere se sono tutti uguali a 0.\\\\
$\lambda_1 e_1 + \cdots + \lambda_r e_r + \mu_2 v_{r+1} + \cdots + \mu_{n-r}v_n = -\nu_1 w_{r+1} - \cdots - \nu_{m-r}w_m$. Vediamo che la parte $\lambda_1 e_1 + \cdots + \lambda_r e_r + \mu_2 v_{r+1} + \cdots + \mu_{n-r}v_n \in V_1$ mentre $-\nu_1 w_{r+1} - \cdots - \nu_{m-r}w_m \in V_2$, quindi $-\nu_1 w_{r+1} - \cdots - \nu_{m-r}w_m \in V_1 \cap V_2 \Longrightarrow$ come $e_1, \cdots, e_r$ è una base di $V_1 \cap V_2, \exists \: \alpha_1, \cdots, \alpha_r \::\: \alpha_1 e_1 + \cdots + \alpha_r e_r = -\nu w_{r+1} - \cdots - \nu_{m-r}w_m$. \\\\
Ma $e_1, \cdots, e_r, w_{r+1}, \cdots, w_m$ è base di $V_2$ ed allora è linearmente indipendente ed allora $\alpha_1 = \cdots = \alpha_r = \nu_1 = \cdots = \nu_{m-r} = 0$, ma allora $\lambda_1 e_1 + \cdots + \lambda_r e_r + \mu_1 v_{r+1} + \cdots + \nu_{n-r}v_n = 0 \Longrightarrow \lambda_1 = \cdots = \lambda_r = \mu_1 = \cdots = \mu_{n-r} = 0$ perché $e_1, \cdots, e_r, v_{r+1}, \cdots, v_n$ è una base di $V_1$.\\
Vediamo dunque che $span(e_1, \cdots, e_r, v_1, \cdots, v_{n-r}, w_1, \cdots, w_{m-r}) = V_1 + V_2$ se $v \in V_1 + V_2$, $v = v^1, v^2 \::\: v^1 \in V_1, v^2 \in V_2$. Ma allora $\exists \alpha_1, \cdots, \alpha_m, \beta_1, \cdots, \beta_m \in \mathbb{R} \::$\\
$v^2 = \alpha_1 e_1 + \cdots + \alpha_r e_r + \alpha_{r+1}v_1 + \cdots + \alpha_n v_{n-r}$ e $v_2 = \beta_1 e_1 + \cdots + \beta_r e_r + \beta_{r+1}w_1 + \cdots + \beta_m w_{m-r}$ perché $e_1, \cdots, e_r, v_1, \cdots, v_{n-r}$ è un base di $V_1$ e $e_1, \cdots, e_r, w_1, \cdots, w_{m-r}$ è una base di $V_2$. \\
Detto ciò allora abbiamo che $v_1 = v^1 + v^2 = (\alpha_1 + \beta_1)e_1 + \cdots + (\alpha_r + \beta_r)e_r + \alpha_{r+1}v_{r+1} + \cdots + \alpha_n v_n + \beta_{r+1}w_1 + \cdots + \beta_m w_m$. $\blacksquare$
\end{demostration}

\begin{example}
Consideriamo i due sottospazi in $\mathbb{R}^4$ seguenti:\\
$V= \Big\{\text{soluzioni di } \begin{array}{l}x_1 + 2x_2 + x_3 = 0\\-x_1 -x_2 + 3x_4 = 0\end{array}\Big\}, W = span\Bigg(w_1 = \begin{bmatrix}2\\0\\1\\1\end{bmatrix}, w_2 = \begin{bmatrix}3\\-2\\-2\\0\end{bmatrix}\Bigg)$\\
Calcoliamo $dim(V \cap W), dim(V + W)$. Sappiamo che $dim(W)=2$ perché $w_1 $ e $w_2$ sono \textbf{linearmente indipendenti}. Questo è ovvio in quanto abbiamo solo due vettori che non sono uno il multiplo dell'altro.\\
Bisogna dunque calcolare la $dim(V)$ tramite Gauss-Jordan:
\[
\begin{bmatrix}
1 & 2 & 1 & 0\\
-1 & -1 & 0 & 3
\end{bmatrix}
\xrightarrow{R_2 + R_1}
\begin{bmatrix}
1 & 2 & 1 & 0\\
0 & 1 & 1 & 3
\end{bmatrix}
\xrightarrow{R_1 - 2R_2}
\begin{bmatrix}
	1 & 0 & -1 & -6\\
	0 & 1 & 1 & 3
\end{bmatrix}
\]
Abbiamo dunque $x_3, x_4$ come variabili libere che fa si che $x_1 = x_3 + 6 x_4$ e $x_2 = -x_3 - 3x_4$, la soluzione generale è dunque:\\\\
$\begin{bmatrix}
	x_1 \\ x_2 \\ x_3 \\ x_4
\end{bmatrix}
= x_3
\begin{bmatrix}
	1\\-1\\1\\0
\end{bmatrix}
 + x_4
\begin{bmatrix}
	6\\-3\\0\\1
\end{bmatrix}$
\hspace{.5cm} con $v_1 = \begin{bmatrix}1\\-1\\1\\0\end{bmatrix}$ e $v_2 = \begin{bmatrix}6\\-3\\0\\1\end{bmatrix}$ \\\\
Quindi $dim(V) = 2$ e $v_1, v_2$ è una base. Cerchiamo ora $dim(V + W)$.
\[
\begin{bmatrix}
1 & 6 & 2 & 3\\
-1 & -3 & 0 & -2\\
1 & 0 & 1 & -2\\
0 & 1 & 1 & 0
\end{bmatrix}
\xrightarrow[R_2 + R_1]{R_3 - R_1}
\begin{bmatrix}
1 & 6 & 2 & 3\\
0 & 3 & 2 & 1\\
0 & -6 & -1 & -5\\
0 & 1 & 1 & 0
\end{bmatrix}
\xrightarrow{R_3 + 3R_1}
\begin{bmatrix}
1 & 6 & 2 & 3\\
0 & 3 & 2 & 1\\
0 & 0 & 3 & -3\\
0 & 1 & 1 & 0
\end{bmatrix}
\xrightarrow{\text{Inverto } R_2, R_4}
\begin{bmatrix}
1 & 6 & 2 & 3\\
0 & 1 & 1 & 0\\
0 & 0 & 3 & -3\\
0 & 3 & 2 & 1
\end{bmatrix}
\]
\[
\xrightarrow{R_4 - R_2}
\begin{bmatrix}
1 & 6 & 2 & 3\\
0 & 1 & 1 & 0\\
0 & 0 & 3 & -3\\
0 & 0 & -1 & 1
\end{bmatrix}
\xrightarrow{R_4 + \frac{1}{3}R_3}
\begin{bmatrix}
1 & 6 & 2 & 3\\
0 & 1 & 1 & 0\\
0 & 0 & 3 & -3\\
0 & 0 & 0 & 0
\end{bmatrix}
\]
Abbiamo dunque 3 pivots ed allora le prime 3 colonne sono indipendenti ma $v_1, v_2, w_1, w_2$ sono dipendenti. Questo fa si che $dim(V+W) =$3.\\
Utilizzando poi Grassman: $dim(V \cap W) = dim(V) + dim(W) - dim(V+W) = 2 + 2 - 3 = 1$
\end{example}

\begin{example}
Siano $V = span\Bigg(\begin{bmatrix}1\\1\\1\\1\end{bmatrix}, \begin{bmatrix}1\\-1\\1\\1\end{bmatrix}\Bigg), W = span\Bigg(\begin{bmatrix}1\\0\\1\\0\end{bmatrix}\begin{bmatrix}1\\2\\0\\2\end{bmatrix}\Bigg)$.\\\\
Chiamiamo i due vettori in V $v_1, v_2$ mentre i due in $W$ $w_1, w_2$. Trovare basi di $V + W, V \cap W$. Per $V+W$ facciamo:
\[
\begin{bmatrix}
1 & 1 & 1 & 1\\
1 & -1 & 0 & 2\\
1 & 1 & 1 & 0\\
1 & -1 & 0 & 2
\end{bmatrix}
\begin{array}{l}
    R_2 - R_1\\
    R_3 - R_1\\
    R_3 - R_1
\end{array}
\Rightarrow
\begin{bmatrix}
1 & 1 & 1 & 1\\
0 & -2 & -1 & 1\\
0 & 0 & 0 & -1\\
0 & -2 & -1 & 1
\end{bmatrix}
R_4 - R_2
\Rightarrow
\begin{bmatrix}
1 & 1 & 1 & 1\\
0 & -2 & -1 & 1\\
0 & 0 & 0 & -1\\
0 & 0 & 0 & 0
\end{bmatrix}
\]
Abbiamo dunque 3 pivtos ed allora $dim(V+W) = 3$, perché $v_1, v_2, w_2$ sono lin. indipendenti e quindi sono una base. Utilizzando allora Grassmann: $dim(V \cap W) = dim(V) + dim(W) - dim(V+W) = 2 + 2 - 3 = 1$.\\
In uno spazio di $dim = 1$ ogni vettore diverso da 0 è base, per trovarlo facciamo:\\
\[\begin{array}{l}
    x_1 + x_2 + x_3 + x_4 = 0\\
    -x_2 - x_3 - x_4 = 0\\
    -x_4 = 0
\end{array}
\hspace{.3cm}
\parbox{7cm}{Abbiamo dunque che $x_4 = 0, x_3$ = t sono variabili libere. Quindi $x_2 = -\frac{7}{2}, x_1 = -\frac{t}{2}$}
\]
La soluzione generale è dunque $(-\frac{t}{2}, -\frac{t}{2}, t, 0)$ e con $t=1$ abbiamo $(\frac{1}{2}, \frac{1}{2}, -1, 0)$ che fa si abbiamo $\frac{1}{2}v_1 + \frac{1}{2}v_2 - w_1 = 0$. Dunque $w_1 = \frac{1}{2}v_1 + \frac{1}{2}v_2 \in V \cap W$.\\
Quindi $w_1$ è una base di $V \cap W$ e questo perché so che questo spazio ha $dim = 1$ grazie a Grassman. Ogni volta che $V \cap W$ è della forma $t \cdot w_1$ allora ri dimostra che $dim(V \cap W) = 1$.
\end{example}
\begin{example}
	Dati $V = M_{3 \times 3}(R)$, $V_1 = \{\text{matrici diagonali}\}$ e $V_2 = \{\text{matrici dove la 1° riga = 2° riga}\}$.
	\[
	\text{Base di $V_1$: }
	\begin{bmatrix}
		1 & 0 & 0 \\
		0 & 0 & 0 \\
		0 & 0 & 0 
	\end{bmatrix}
	\text{,}
	\begin{bmatrix}
		0 & 0 & 0 \\
		0 & 1 & 0 \\
		0 & 0 & 0 
	\end{bmatrix}
	\text{,}
	\begin{bmatrix}
		0 & 0 & 0 \\
		0 & 0 & 0 \\
		0 & 0 & 1
	\end{bmatrix}
	\Rightarrow dim(V_1) = 3
	\]
	\[
	\text{Base di $V_2$: }
	\begin{bmatrix}
		1 & 0 & 0 \\
		1 & 0 & 0 \\
		0 & 0 & 0 
	\end{bmatrix}
	\text{,}
	\begin{bmatrix}
		0 & 1 & 0 \\
		0 & 1 & 0 \\
		0 & 0 & 0 
	\end{bmatrix}
	\text{,}
	\begin{bmatrix}
		0 & 0 & 1 \\
		0 & 0 & 1 \\
		0 & 0 & 0
	\end{bmatrix}
	\text{,}
	\begin{bmatrix}
		0 & 0 & 0 \\
		0 & 0 & 0 \\
		1 & 0 & 0
	\end{bmatrix}
	\text{,}
	\begin{bmatrix}
		0 & 0 & 0 \\
		0 & 0 & 0 \\
		0& 1 & 0
	\end{bmatrix}
	\text{,}
	\begin{bmatrix}
		0 & 0 & 0 \\
		0 & 0 & 0 \\
		0 & 0 & 1
	\end{bmatrix}
	\Rightarrow dim(V_1) = 6
	\]
	\[
	\text{Elemento generale di $V_1 \cap V_2$: }
	\begin{bmatrix}
		0 & 0 & 0 \\
		0 & 0 & 0 \\
		0 & 0 & 1
	\end{bmatrix}
	\Rightarrow dim(V_1 \cap V_2) = 1
	\]
	Grassmann: $dim(V_1 + V_2) = 3 + 6 - 1 = 8$
\end{example}