% !TeX spellcheck = it_IT
\newpage
\section{Algoritmo di Gauss}
Utilizzando le proprietà (1), (2) e (3) viste sopra si ottiene un algoritmo per semplificare $(E)$.
In un primo momento si considera $b_1 = \ldots = b_n = 0$ e mettiamo i coefficienti in una matrice $n \times m$, [$a_{ij}$]:
\[
\begin{bmatrix}
a_{11} & a_{12} & \ldots & a_{1m}\\
a_{21} & a_{22} & \ldots & a_{2m}\\
& \vdots & & \\
a_{n1} & a_{n2} & \ldots & a_{nm}
\end{bmatrix}
\xRightarrow[]{\text{(Con operazioni)}}
\begin{bmatrix}
\ldots \\
a_{j1} + \lambda a_{i1} + a_{j2} + \lambda a_{i2} + \ldots + a_{jm} + \lambda a_{im}\\
\ldots
\end{bmatrix}
\]
Le operazioni di prima si traducono come:
\begin{enumerate}
    \item Scambiare due righe fra di loro.
    \item Sostituire la riga $R_j$ con la riga $R_j + \lambda R_i$.
    \item Moltiplicare una riga per $\lambda \neq 0$.
\end{enumerate}
Partendo da una matrice l'algoritmo produce, utilizzando le 2 operazioni una matrice detta \textbf{a forma di scalini}.

\begin{definition}[Matrice a forma a scalini]
Una matrice è a forma a scalini (per righe) se:
\begin{itemize}
    \item Le righe $(0,\ldots,0)$ sono "in fondo" alla matrice (partendo da sinistra).
    \item Il primo elemento di ogni riga (se esiste) è a destra del primo elemento diverso da 0 della riga precedente. Tale elemento si dice pivot.
\end{itemize}
\end{definition}

\begin{definition}[Pivot]
Il primo elemento diverso a 0 d ogni riga di una matrice (nella forma a scalini) si chiama pivot.
\end{definition}

\begin{example}
Esempio di matrici in forma a scalini e non:
\vspace{-10pt}
\begin{figure}[h!]
    \centering
    \begin{minipage}{.3\linewidth}
        \centering
        \[
            \begin{bmatrix}
            1 & 1 & 1\\
            1 & 1 & 0\\
            0 & 0 & 1
            \end{bmatrix}
        \]
        NO
    \end{minipage}
    \begin{minipage}{.3\linewidth}
        \centering
        \[
            \begin{bmatrix}
            1 & 1 & 1\\
            0 & 1 & 1\\
            0 & 0 & 1
            \end{bmatrix}
        \]
        SI
    \end{minipage}
    \begin{minipage}{.3\linewidth}
        \centering
        \[
            \begin{bmatrix}
            0 & 1 & 1\\
            1 & 1 & 0\\
            0 & 0 & 1
            \end{bmatrix}
        \]
        NO
     \end{minipage}
\end{figure}
\end{example}

\begin{definition}[Algoritmo di Gauss]
Ogni matrice $n \times m$ si mette in forma a scalini (per righe) con operazioni del tipo (A) e (B).
\begin{enumerate}
    \item Se la matrice è già in forma a scalini abbiamo finito.
    \item Si cerca il primo elemento diverso da 0 della prima colonna diversa da 0.
    \item Cambiamo righe si può supporre che questo elemento è il pivot della prima riga. Notiamo che se siamo in forma a scalini abbiamo finito sennò procediamo.
    \item Si annullano tutti gli elementi della colonna del pivot sotto il pivot  con operazioni del tipo (B). Se siamo in forma a scalini finiamo sennò continuiamo.
    \item Non consideriamo la prima riga (la cancelliamo) e ricominciamo con l'algoritmo.
\end{enumerate}
\end{definition}

\newpage
\begin{example}
Prendiamo il seguente sistema di equazioni e scriviamolo su una matrice:
\[
\begin{array}{l}
x_1 + x_2 + 3x_4 = 0\\
3x_1 - x_2 + x_3 + 10x_4 = 0\\
x_1 + 5x_2 + 2x_3 + x_4 = 0
\end{array}
\hspace{.2cm}
\Rightarrow
\hspace{.2cm}
\begin{bmatrix}
1 & -1 & 0 & 3\\
3 & -1 & 1 & 10\\
1 & 5 & 2 & 1
\end{bmatrix}
\:\:
\parbox{4cm}{Da qui iniziamo ad applicare l'algoritmo}
\]
\end{example}
\begin{figure}[h!]
    \vspace{-20pt}
    \centering
    \begin{minipage}{.3\linewidth}
        \centering
        \[
            \begin{bmatrix}
            \underline{1} & -1 & 0 & 3\\
            3 & -1 & 1 & 10\\
            1 & 5 & 2 & 1
            \end{bmatrix}
        \]
        Applichiamo il passo (1) e troviamo 1 come pivot
    \end{minipage}
    \begin{minipage}{.3\linewidth}
        \centering
        \[
            \begin{bmatrix}
            1 & -1 & 0 & 3\\
            0 & 3 & 1 & 1\\
            0 & 6 & 2 & -1
            \end{bmatrix}
        \]
        Applichiamo il passo (3) e calcoliamo $R_2 = R_2 - 3R_1$ e $R_3 = R_3 - R_1$
    \end{minipage}
    \begin{minipage}{.3\linewidth}
        \centering
        \[
            \begin{bmatrix}
            0 & 3 & 1 & 1\\
            0 & 0 & -1 & -5
            \end{bmatrix}
        \]
        Applichiamo (4) e non consideriamo la riga $R_1$ per poi ripetere l'operazione (3) facendo $R_3 = R_3 - 3R_2$
    \end{minipage}
\end{figure}
\hspace{-15pt}Vediamo così che la matrice finale è ha forma di scalini. Possiamo ora prendendo i numeri nella matrice andare a riscrivere il sistema di equazioni associato. Il sistema è il seguente:
\[
\begin{array}{l}
x_1 + x_2 + 3x_4 = 0\\
2x_2 + x_3 + x_4 = 0\\
-x_3 + 5x_4 = 0
\end{array}
\hspace{.2cm}
\xRightarrow[]{\text{(E la matrice)}}
\hspace{.2cm}
\begin{bmatrix}
1 & -1 & 0 & 3\\
0 & 3 & 1 & 1\\
0 & 0 & -1 & -5
\end{bmatrix}
\]
A questo punto se consideriamo $x_4 = t$ abbiamo che $x_3 = -5t$ e di conseguenza $2x_2 - 5t + t = 0$ e quindi $x_2 = 2t$ ed ancora abbiamo $x_1 = 2t -3t -t$. Possiamo dunque dire che in questo esempio $x_4$ essendo una colonna senza pivot è una "variabile libera" e quindi ci sono più soluzioni.\\
Potrebbe esserci anche il caso in cui la colonna contenga un pivot e quinci ci sarebbe un unica soluzione.\\\\
Se consideriamo invece un sistema non omogeneo formato aggiungendo una colonna con $b_1, ..., b_n$, è possibile utilizzare ugualmente l'algoritmo di guass aggiungendo una colonna alla matrice.

\begin{example}
Prendiamo il seguente sistema di equazioni e mettiamolo su una matrice:
\end{example}
\begin{figure}[h!]
    \vspace{-20pt}
    \centering
    \begin{minipage}{1\linewidth}
        \centering
        \[
            \begin{array}{l}
            x_1 + x_2 + x_3 + 2x_4 = 9\\
            x_1 + x_2 + 2x_3 + x_4 = 8\\
            x_1 + 2x_2 + x_3 + x_4 = 7\\
            2x_1 + x_2 + x_3 + x_4 = 6
            \end{array}
            \hspace{.2cm}
            \Rightarrow
            \hspace{.2cm}
            \begin{bmatrix}
            1 & 1 & 1 & 2 & 9\\
            1 & 1 & 2 & 1 & 8\\
            1 & 2 & 1 & 1 & 7\\
            2 & 1 & 1 & 1 & 6
            \end{bmatrix}
            \xRightarrow[]{\text{Uso algoritmo}}
            \begin{bmatrix}
            1 & 1 & 1 & 2 & 9\\
            1 & 1 & 2 & 1 & 8\\
            1 & 2 & 1 & 1 & 7\\
            2 & 1 & 1 & 1 & 6
            \end{bmatrix}
            \begin{array}{r}
            R_2 = R_2 - R_1\\
            R_3 = R_3 - R_1\\
            R_4 = R_4 - 2R_1
            \end{array}
        \]
        Il pivot è in $R_1$ ed è 1, da qui applichiamo (3) e poi (4).
    \end{minipage}
\end{figure}
\vspace{-15pt}
\begin{figure}[h!]
    \begin{minipage}{1\linewidth}
        \centering
        \[
            \begin{bmatrix}
            0 & 0 & 1 & 2 & -1\\
            0 & 1 & 0 & -1 & -2\\
            0 & -1 & -1 & -3 & -12
            \end{bmatrix}
            \Rightarrow
            \begin{bmatrix}
            0 & 1 & 0 & -1 & -2\\
            0 & 0 & 1 & -1 & -1\\
            0 & -1 & -1 & -3 & -12
            \end{bmatrix}
            \Rightarrow
            \begin{bmatrix}
            0 & 1 & 0 & -1 & -2\\
            0 & 0 & 1 & -1 & -1\\
            0 & 0 & -1 & -4 & -14
            \end{bmatrix}
        \]
        Il pivot ora è in $R_3$ e quindi applichiamo (2) per scambiare $R_2$ con $R_3$ e poi facciamo (3) con $R_4 = R_4 - R_2$.
    \end{minipage}
\end{figure}
\vspace{-15pt}
\begin{figure}[h!]
    \begin{minipage}{.45\linewidth}
        \centering
        \[
            \begin{bmatrix}
            0 & 0 & 1 & -1 & -1\\
            0 & 0 & -1 & -4 & -14
            \end{bmatrix}
            \Rightarrow
            \begin{bmatrix}
            0 & 0 & 1 & -1 & -1\\
            0 & 0 & 0 & -5 & -15
            \end{bmatrix}
        \]
        Usiamo (4) per eliminare $R_2$ e troviamo il pivot in $R_3$, applichiamo poi (4) con $R_4 = R_4 + R_4$
    \end{minipage}
    \hspace{1cm}
    $\Longrightarrow$
    \hspace{-1cm}
    \begin{minipage}{.45\linewidth}
        \centering
        \[
            \begin{bmatrix}
            1 & 1 & 1 & 2 & 9\\
            0 & 1 & 0 & -1 & -2\\
            0 & 0 & 1 & -1 & -1\\
            0 & 0 & 0 & -5 & -15
            \end{bmatrix}
        \]
        Abbiamo finito perché il risultato è a scalini    
\end{minipage}
\end{figure}
Il risultato finale corrisponde al seguente sistema:
\[
\begin{array}{l}
x_1 + x_2 + x_3 + 2x_4 = 9\\
x_2 - x_4 = -2\\
x_3 - x_4 = -1\\
-5x_4 = -15
\end{array}
\hspace{.5cm}
\parbox{5cm}{Se sostituiamo: \\$x_4 = 4, x_3 = 2, x_2 = 1, x_1 = 0$\\Quindi abbiamo un sistema con un unica soluzione}
\]
Da questo esempio possiamo vedere una caratteristica comune per questa tipologia di esercizi, cioè che il sistema di equazione ha un unica soluzione se ogni colonna contiene un pivot.

\newpage
\begin{example}
Altro esempio di sistema di applicazione dell'algoritmo di gauss.
\[
    \begin{array}{l}
    x_1 + x_2 - 4x_3 = 1\\
    2x_1 + 3x_2 - 10x_3 = 2\\
    5x_1 + 3x_2 - 4x_3 = 5
    \end{array}
    \Longrightarrow
    \begin{bmatrix}
    1 & 1 & -4 & 1\\
    2 & 3 & -10 & 2\\
    5 & -3 & -4 & 1
    \end{bmatrix}
    \begin{array}{r}
    R_2 = R_2 - 2R_1\\
    R_3 = R_3 - 5R_1
    \end{array}
    \Longrightarrow
    \begin{bmatrix}
    0 & 1 & -2 & 0\\
    0 & -8 & 16 & 0
    \end{bmatrix}
    R_3 = R_3 + 8R_2
\]
\[
    \Longrightarrow
    \begin{bmatrix}
    1 & 1 & -4 & 1\\
    0 & 1 & -2 & 0\\
    0 & 0 & 0 & 0
    \end{bmatrix}
    \hspace{.3cm}
    \parbox{5cm}{Questa è una matrice in forma a scalini e la sua trasposizione in sistema di equazione è:}
    \hspace{.3cm}
    \begin{array}{l}
    x_1 + x_2 - 4x_3 = 1\\
    x_2 - 2x_3 = 0
    \end{array}
\]
Abbiamo quindi che $x_3$ è una variabile libera e quindi se poniamo $x_3 = t$ abbiamo $x_2 = 2t$ e $x_1 = 1+2t$.\\
Soluzione particolare: (1,0,0). Soluzione generale: $(1+2t, 2t, t)$. Sol. Omogenea: $(2t, 2t, t)$.
\end{example}
\hspace{-15pt}Da questo esempio vediamo invece che se c'è una colonna senza pivot all'ora esiste almeno una variabile libera e quindi ci sono $\infty$ soluzioni.

\begin{example}
Facciamo un ultimo esempio per vedere un ulteriore casistica per l'algoritmo di gauss.
\[
    \begin{array}{l}
    x_1 - x_2 - x_3 = 3\\
    3x_1 - 2 x_2 - 4x_3 = 3\\
    4x_1 + x_2 - 9x_3 = 7
    \end{array}
    \Longrightarrow
    \begin{bmatrix}
    1 & -1 & -1 & 3\\
    3 & -2 & -4 & 3\\
    4 & 1 & -9 & 7
    \end{bmatrix}
    \begin{array}{r}
    R_2 = R_2 - 3R_1\\
    R_3 = R_3 - 4R_1
    \end{array}
    \Longrightarrow
    \begin{bmatrix}
    0 & 1 & -1 & -6\\
    0 & 5 & -5 & -5
    \end{bmatrix}
    R_3 = R_3 -5R_2
\]
\[
    \Longrightarrow
    \begin{bmatrix}
    1 & -1 & -1 & 3\\
    0 & 1 & -1 & -6\\
    0 & 0 & 0 & 25
    \end{bmatrix}
    \hspace{.3cm}
    \parbox{5cm}{Questa è una matrice in forma a scalini e la sua trasposizione in sistema di equazione è:}
    \hspace{.3cm}
    \begin{array}{l}
    x_1 - x_2 - x_3 = 3\\
    x_2 - x_3 = -6
    0 = 25
    \end{array}
\]
Possiamo notare che l'equazione $0=25$ non ha senso quindi non c'è nessuna soluzione.
\end{example}
\hspace{-15pt}Anche in questo caso possiamo estendere l'esempio in un caso generale dicendo che se c'è un pivot nell'ultima colonna allora non esistono soluzioni particolari (il sistema omogeneo però ammette $\infty$ soluzioni).\\
In sintesi possiamo riassumere i 3 casi visti in questi esempi come di seguito:
\begin{itemize}
    \item Ogni colonna "non aggiunta" ha un pivot $\Longleftrightarrow$ unica soluzione.
    \item C'è un pivot nell'ultima colonna $\Longleftrightarrow \: \nexists$ soluzione.
    \item C'è una colonna "non aggiunta" senza pivot e l'ultima colonna non ne ha $\Longleftrightarrow \: \infty$ soluzioni.
\end{itemize}

\subsection{Algoritmo di Gauss-Jordan}
Questo algoritmo estendo l'algoritmo di gauss producendo una matrice ridotta a scalini.
\begin{definition}[Matrice ridotta]
Una matrice è in forma \textbf{ridotta} a scalini se:
\begin{itemize}
    \item E' in forma a scalini.
    \item Ogni pivot è uguale a 1.
    \item Ogni pivot è l'unico elemento $\neq 0$ nella sua colonna.
\end{itemize}
\end{definition}

\begin{example}
Esempio di matrici in forma a scalini ridotta e non.
\vspace{-10pt}
\begin{figure}[h!]
    \centering
    \begin{minipage}{.4\linewidth}
        \centering
        \[
            \begin{bmatrix}
            1 & 2 & 0 & 0\\
            0 & 0 & 1 & 0\\
            0 & 0 & 0 & 1\\
            \end{bmatrix}
        \]
        SI, è in forma a scalini ridotta
    \end{minipage}
    \hspace{.3cm}
    \begin{minipage}{.4\linewidth}
        \centering
        \[
            \begin{bmatrix}
            1 & 2 & 3 & 4\\
            0 & 0 & 1 & 2\\
            0 & 0 & 0 & 1
            \end{bmatrix}
        \]
        NO, questa è in forma scalini ma non ridotta.
    \end{minipage}
\end{figure}
\end{example}
\newpage
\begin{definition}[Algoritmo di Gauss-Jordan]
L'algoritmo sfrutta le 2 operazioni viste per l'algoritmo di gauss (A) e (B) ma aggiungendo anche l'operazione (C). Questo algoritmo, partendo da una matrice a scalini genera una matrice ridotta a scalini.
\begin{enumerate}
    \item Con l'algoritmo di Gauss portiamo la matrice in forma a scalini.
    \item In ogni riga si cerca il pivot (se esiste). Poi se il pivot è $\lambda \neq 1$, moltiplicare la riga per $\frac{1}{\lambda}$ (operazione (C)).
    \item Nella colonna dei pivot gli elementi sotto (e nella riga a sinistra) sono già uguali a 0. Annullare gli elementi sopra della colonna con operazioni del tipo (B).\\
    Questa operazione non cambia gli altri pivot perché sono o a sinistra o sotto.
\end{enumerate}
\end{definition}

\begin{example}
Proviamo ad applicare l'algoritmo di gauss-jordan con il seguente sistema di euqzioni
\[
    \begin{array}{l}
    2x_1 + x_2 - x_3 = -1\\
    3x_1 + 2x_2 - x_3 = 0\\
    4x_1 - 3x_2 + x_3 = -1\\
    5x_1 - 2x_2 + 2x_3 = 2
    \end{array}
    \Longrightarrow
    \begin{bmatrix}
    2 & 1 & -1 & -1\\
    3 & 2 & -1 & 0\\
    4 & -3 & 1 & -1\\
    5 & -2 & 2 & 2
    \end{bmatrix}
    \hspace{.3cm}
    \parbox{5cm}{Per semplificare la matrice usiamo l'operazione (C) e moltiplichiamo una riga per una costante:}
    \begin{array}{l}
    R_2 = 2R_2\\
    R_4 = 2R_4
    \end{array}
\]
\[          
    \Rightarrow
    \parbox{2cm}{Applichiamo l'algoritmo di gauss}
    \begin{bmatrix}
    2 & 1 & -1 & -1\\
    6 & 4 & -2 & 0\\
    4 & -3 & 1 & -1\\
    10 & -4 & 4 & 4
    \end{bmatrix}
    \begin{array}{l}
    R_2 = R_2 - 3R_1\\
    R_3 = R_3 - 2R_1\\
    R_4 = R_4 - 5R_1
    \end{array}
    \Rightarrow
    \begin{bmatrix}
    2 & 1 & -1 & -1\\
    0 & 1 & 1 & 3\\
    0 & -5 & 3 & 1\\
    0 & -9 & 9 & 9
    \end{bmatrix}
    \begin{array}{l}
    R_3 = R_3 + 5R_2\\
    R_4 = R_4 + 9R_2
    \end{array}
\]
\[
    \Rightarrow
    \begin{bmatrix}
    2 & 1 & -1 & -1\\
    0 & 1 & 1 & 3\\
    0 & 0 & 8 & 16\\
    0 & 0 & 18 & 36
    \end{bmatrix}
    \hspace{.3cm}
    \begin{array}{l}
    R_3 = \frac{1}{8}R_3\\
    R_4 = R_4 - \frac{18}{8}R_3
    \end{array}
    \hspace{.3cm}
    \parbox{4cm}{La matrice è in forma scalini quindi applichiamo il punto (2) dell'algoritmo di guass-jordan}
    \Rightarrow
    \begin{bmatrix}
    2 & 1 & -1 & -1\\
    0 & 1 & 1 & 3\\
    0 & 0 & 1 & 2\\
    0 & 0 & 0 & 0
    \end{bmatrix}
    R_1 = R_1 - R_2
\]
\[
\Rightarrow
    \begin{bmatrix}
    2 & 0 & -2 & -4\\
    0 & 1 & 1 & 3\\
    0 & 0 & 1 & 2\\
    0 & 0 & 0 & 0
    \end{bmatrix}
    \begin{array}{l}
    R_1 = R_1 + 2R_3\\
    R_2 = R_2 - R_3
    \end{array}
    \Rightarrow
    \begin{bmatrix}
    2 & 0 & 0 & 0\\
    0 & 1 & 0 & 1\\
    0 & 0 & 1 & 2\\
    0 & 0 & 0 & 0
    \end{bmatrix}
    R_1 = \frac{1}{2}R_1
    \Rightarrow
    \begin{bmatrix}
    1 & 0 & 0 & 0\\
    0 & 1 & 0 & 1\\
    0 & 0 & 1 & 2\\
    0 & 0 & 0 & 0
    \end{bmatrix}
    \begin{array}{l}
    x_1 = 0\\
    x_2 = 1\\
    x_3 = 2
    \end{array}
\]
\end{example}

\begin{example}
Vediamo ora un secondo esempio di questo algoritmo.
\[
    \begin{array}{l}
    2x_1 - 4x_2 + 3x_3 - x_4 = 3\\
    3x_1 - 6x_2 + x_3 + 9x_4 = 3\\
    4x_1 - 8x_2 + 5x_3 + x_4 = 7
    \end{array}
    \Rightarrow
    \begin{bmatrix}
    2 & -4 & 3 & -1 & 3\\
    3 & -6 & 1 & 9 & 8\\
    4 & -8 & 5 & 1 & 7
    \end{bmatrix}
    \hspace{.3cm}
    \parbox{4cm}{Anche in questo caso semplifichiamo la seconda riga con un operazione (C)}
    \hspace{.3cm}
    R_2 = 2R_2
\]
\[
    \Rightarrow
    \begin{bmatrix}
    2 & -4 & 3 & -1 & 3\\
    6 & -12 & 2 & 18 & 16\\
    4 & -8 & 5 & 1 & 7
    \end{bmatrix}
    \begin{array}{l}
    R_2 = R_2 - 3R_1\\
    R_3 = R_3 - 2R_1
    \end{array}
    \Rightarrow
    \begin{bmatrix}
    2 & -4 & 3 & -1 & 3\\
    0 & 0 & -7 & 21 & 7\\
    0 & 0 & -1 & 3 & 1
    \end{bmatrix}
    \begin{array}{l}
    R_2 = -\frac{1}{7}R_2\\
    R_3 = -R_3
    \end{array}
\]
\[
    \Rightarrow
    \begin{bmatrix}
    2 & -4 & 3 & -1 & 3\\
    0 & 0 & 1 & -3 & -1\\
    0 & 0 & 1 & -3 & -1
    \end{bmatrix}
    R_3 = R_3 + R_2
    \begin{bmatrix}
    2 & -4 & 3 & -1 & 3\\
    0 & 0 & 1 & -3 & -1\\
    0 & 0 & 0 & 0 & 0
    \end{bmatrix}
    R_1 = R_1 - 3R_2
\]
\[
    \Rightarrow
    \begin{bmatrix}
    2 & -4 & 0 & 8 & 6\\
    0 & 0 & 1 & -3 & -1\\
    0 & 0 & 0 & 0 & 0
    \end{bmatrix}
    R_1 = \frac{1}{2}R_1
    \begin{bmatrix}
    1 & -2 & 0 & 4 & 3\\
    0 & 0 & 1 & -3 & -1\\
    0 & 0 & 0 & 0 & 0
    \end{bmatrix}
    \begin{array}{l}
    x_2 = s\\
    x_4 = t
    \end{array}
    \begin{array}{l}
    x_1 = 3 + 2s - 4t\\
    x_3 = -1 + 3t
    \end{array}
\]
\end{example}