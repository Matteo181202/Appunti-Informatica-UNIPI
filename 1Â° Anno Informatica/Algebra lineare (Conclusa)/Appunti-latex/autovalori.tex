% !TeX spellcheck = it_IT
\newpage
\section{Autovalori}
\begin{definition}[Autovalori]
	Sia $V$ uno spazio vettoriale su $\mathbb{R}$ ($dim(V) < \infty$) e $\phi: V \to V$. $\lambda \in \mathbb{R}$ è \textbf{autovalore} di $\phi$ se $\exists v \neq 0$ in $V$ tale che $\phi(V) = \lambda \cdot v$. In questo caso $v$ è \textbf{autovettore} di $\phi$ (associato a $\lambda$).
\end{definition}
\begin{observation}
	Alcune osservazioni su questa definizione:
	\begin{enumerate}
		\item $v$ può essere autovettore per un solo $\lambda$. Infatti, se $\begin{cases}
			\phi(v)=\lambda_{1} \cdot v \\
			\phi(v)=\lambda_{2} \cdot v
		\end{cases} \Longrightarrow (\lambda_{1} - \lambda_{2}) \cdot v = 0 \Longrightarrow \lambda_{1} = \lambda_{2}$
		\item In generale ci sono molti autovettori associati allo stesso $\lambda$
	\end{enumerate}
\end{observation}

\begin{definition}[Diagonalizzabile]
	$\phi$ è \textbf{diagonalizzabile} se $\exists$ base $B$ tale che $[\phi]^B_B$ è una matrice \textbf{diagonale}.
\end{definition}

\begin{proposition}
	$\phi$ è diagonalizzabile se e solo se $V$ ammette una base costituita da autovettori di $\phi$. \\
\end{proposition}

\begin{example}
	Se $\phi: \mathbb{R}^2 \to \mathbb{R}^2$, $v \mapsto A \cdot v$  dove $A = \begin{bmatrix}
		1 & 1 \\
		0  &1
	\end{bmatrix}$ $\Longrightarrow$ $\phi$ non è diagonalizzabile.  
\end{example}

\subsection{Come trovare gli autovalori?}
$\lambda$ è autovalore per $\phi$ se e solo se $\phi(v) = \lambda \cdot v$ per un $v \neq 0$. Quindi $\phi(v) - \lambda \cdot v = 0 \Longrightarrow (\phi - \lambda \cdot id) \cdot v = 0$.
%TODO Ti sei perso