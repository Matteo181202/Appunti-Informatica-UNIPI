% !TeX spellcheck = it_IT
\newpage
\section{Applicazione lineare}
\begin{definition}[Applicazione lineare]
Siano $V_1, V_2$ spazi vettoriali su $\mathbb{R}$. Un'applicazione lineare (o mappa lineare) è una mappa $\varphi: V_1 \to V_2$ tale che:
\begin{enumerate}
    \item $\varphi(v_1 + v_2) = \varphi(v_1) + \varphi(v_2)$, $\forall v_1, v_2 \in V_1$.
    \item $\lambda \varphi(v) = \varphi(\lambda v)$, $\forall  v \in V_1$.
\end{enumerate}
\end{definition}

\begin{example}
Alcuni esempi di applicazioni lineari:
\begin{itemize}
    \item $V_1 = \mathbb{R}^n$, $V_2 = \mathbb{R}$, $\varphi\Big(\begin{bmatrix}a_1\\\vdots\\a_n\end{bmatrix}\Big) = \lambda_1a_1 + \cdots + \lambda_n a_n$ con $\lambda_1 \cdots \lambda_n$ fisso.
    \item $V_1 = \mathbb{R}^n$, $V_2 = \mathbb{R}^2$, $\varphi\Bigg(\begin{bmatrix}a_1 \\ \vdots \\ a_n\end{bmatrix}\Bigg) = \begin{bmatrix}{c}\lambda_1 a_1 + \cdots + \lambda_n a_n \\ \mu_1 a_1 + \cdots + \mu_n a_n\end{bmatrix}$ con $\lambda_1, \cdots, \lambda_n$ e $\mu_1, \cdots, \mu_n$ fissi.
    \item $V_1 = \mathbb{R}^n, V_2 = \mathbb{R}^{n-1}$, $\varphi\Bigg(\begin{bmatrix}a_1 \\ \vdots \\ a_n\end{bmatrix}\Bigg) = \begin{bmatrix}a_1 \\ \vdots \\ a_{n-1}\end{bmatrix}$
    \item $V_1 = \mathbb{R}[x]_{\leq d}, V_2 = \mathbb{R}[x]_{\leq d-1}$ quindi è come scrivere $\varphi(f) = f'$. \\
    E questo va bene perché sono rispettate le proprietà (a) e (b) della definizione sopra.
    \item $V_1 = \{f:[0,1]\to \mathbb{R} \::\: f \text{ continua }, \int_0^1 f<\infty\}$, $V_2 = \mathbb{R}$. Vediamo che $\varphi (f) = \int_0^1 f$.\\
    Infatti, anche in questo caso, le proprietà (a) e (b) della definizione sono rispettate.
\end{itemize}
\end{example}

\hspace{-15pt}Sia $\varphi: V_1 \to V_2$ un'applicazione lineare e sia $e_1, \cdots, e_n$ una base di $V_1$ allora sia $v \in V_1$, $v = \lambda_1 e_1 + \lambda_2 e_2 + \cdots + \lambda_n e_n$. $\varphi(v) = \varphi(\lambda_1 e_1) + \varphi(\lambda_2 e_2) + \cdots + \varphi(\lambda_n e_n) = \lambda_1 \varphi(e_1) + \lambda_2 \varphi(e_2) + \cdots + \lambda_n \varphi(e_n)$.\\
In conclusione, conoscere $\varphi(v) \Longleftrightarrow$ conoscere $\varphi (e_1), \cdots, \varphi (e_n)$ e lineare coordinate di v rispetto $e_1, \cdots, e_n$. Viceversa se faccio $\varphi (e_1) = v_1, \varphi(e_2) = v_2, \cdots, \varphi(e_n) = v_n$ allora $\exists !$ applicazione lineare $\varphi v_1 - \varphi v_2$ con queste proprietà.

\begin{example}
$V_1 = \mathbb{R}^n$, $V_2 = \mathbb{R}^2$ e le basi standard sono $\begin{bmatrix}1\\0\end{bmatrix}, \begin{bmatrix}0\\1\end{bmatrix}$. Esiste una sola $\varphi: \mathbb{R}^2 \to \mathbb{R}^2$ tale che $\varphi \Big(\begin{bmatrix}1\\0\end{bmatrix}\Big) = \Big(\begin{bmatrix}0\\0\end{bmatrix}\Big)$, $\varphi \Big(\begin{bmatrix}0\\1\end{bmatrix}\Big) = \Big(\begin{bmatrix}0\\1\end{bmatrix}\Big)$. Infatti tale $\varphi$ è dato da $\varphi \Big(\begin{bmatrix}x_1\\x_2\end{bmatrix}\Big) = \Big(\begin{bmatrix}0\\x_2\end{bmatrix}\Big)$
\end{example}

\subsection{Nucleo e immagine}
\begin{definition}
Sia $\varphi: V_1 \to V_2$ un'applicazione lineare possiamo definire di $\varphi$:
\begin{itemize}
    \item \textbf{Il nucleo}: $Ker(\varphi) = \{v \in V_1 \::\: \varphi(v) = 0\} \subset V_1$ sottospazio. 
    \item \textbf{L'immagine}: $Im(\varphi) = \{v_2 \in V_2 \::\: \exists v_1 \in V_1 \::\: \varphi(v_1) = v_2\} \subset V_2$ sottospazio.
\end{itemize}
\end{definition}

\begin{example}
Alcuni esempi di nucleo ed immagine di un'applicazione lineare.
\begin{enumerate}
    \item Per $\varphi: \mathbb{R}^2 \to \mathbb{R}^2$, $\varphi\Bigg(\begin{bmatrix}x_1\\x_2\end{bmatrix}\Bigg) = \begin{bmatrix}0\\x_2\end{bmatrix}$.\\
    $Ker(\varphi) = \{\begin{bmatrix}x_1\\0\end{bmatrix} \::\: x_1 \in \mathbb{R}\} = span\Bigg(\begin{bmatrix}1\\0\end{bmatrix}\Bigg)$ \hspace{.3cm} $Im(\varphi) = \{\begin{bmatrix}0\\x_2\end{bmatrix} \::\: x_2 \in \mathbb{R}\} = span\Bigg(\begin{bmatrix}0\\1\end{bmatrix}\Bigg)$
    \item $\varphi: \mathbb{R}^2 \to \mathbb{R}^2$, $\varphi\Bigg(\begin{bmatrix}x_1\\x_2\end{bmatrix}\Bigg) = \begin{bmatrix}x_1\\0\end{bmatrix}$.\\
    $Ker(\varphi) = \{\begin{bmatrix}x_1\\0\end{bmatrix} \::\: x_1 \in \mathbb{R}\} = span\Bigg(\begin{bmatrix}1\\0\end{bmatrix}\Bigg)$ \hspace{.3cm} $Im(\varphi) = \{\begin{bmatrix}x_2\\0\end{bmatrix} \::\: x_2 \in \mathbb{R}\} = span\Bigg(\begin{bmatrix}1\\0\end{bmatrix}\Bigg)$
    \item $\varphi: \mathbb{R}[x]_{\leq d} \to \mathbb{R}[x]_{\leq d-1}$, $Ker(\varphi) = \{$ polinomi costanti $\} = span(1)$ \hspace{.3cm} $Im(\varphi) = \mathbb{R}[x]_{d-1}$
\end{enumerate}
\end{example}

\begin{theorem}
Sia $dim(V_1) < \infty$ e sia $\varphi: V_1 \to V_2$ un'applicazione lineare, allora vale che:
\[dim\: Ker(\varphi) + dim \: Im(\varphi) = dim \: V_1\]
\end{theorem}

\begin{demostration}
Per dimostrare il teorema sopra partiamo prendendo $v_1, \cdots, v_r$ una base di $Ker(\varphi)$ (quindi $dim\:Ker(\varphi) = r$), e $w_1, \cdots, w_s$ una base di $Im(\varphi)$ (quindi $dim\:Im(\varphi) = s$). \\\\
Siano poi $\overline{v_1}, \cdots, \overline{v_s} \in V_1$ tali che $\varphi(\overline{v_1}) = w_1, \cdots, \varphi(\overline{v_s}) = w_s$. Noi dobbiamo dimostrare che $v_1, \cdots, v_r, \overline{v_1}, \cdots, \overline{v_2}$ è una base di $V_1$ (in questo modo dimostriamo che $dim\:V_1 = r + s$ ed il teorema è verificato).\\\\
Verifichiamo l'indipendenza lineare. Supponiamo che $\lambda_1 v_1 + \cdots + \lambda_r v_r + \lambda_{r+1}\overline{v_1} + \cdots + \lambda_{r+s}\overline{v_s} = 0$. Applichiamo $\varphi$: $\lambda_1 v_1 + \cdots + \lambda_r v_r + \varphi(\lambda_{r+1}\overline{v_1} + \cdots + \lambda_{r+2}\overline{v_s}) = 0$ ($\varphi(v_1) = 0 \:\forall : i$). quindi $\lambda_{r+1}\varphi(\overline{v_1}) + \cdots + \lambda_{r+2}\varphi(\overline{v_s}) = 0$ che è come scrivere $\lambda_{r+1}w_1 + \cdots + \lambda_{r+s}w_s = 0 \Longrightarrow \lambda_{r+1} = \cdots = \lambda_{r+s} = 0$ perché $w_1, \cdots, w_s$ base.\\\\
Quindi $\lambda_1 v_1 + \cdots + \lambda_r v_r = 0$ ed allora $\lambda_1 = \cdots = \lambda_r = 0$ perché $v_1, \cdots, v_r$ è una base di $Ker(\varphi)$.\\
In fine $\lambda_1 = \cdots = \lambda_r = \lambda_{r+1} = \cdots = \lambda_{r+s} = 0$.\\
$span(v_1, \cdots, v_r, \overline{v_1}, \cdots, \overline{v_s}) = v_1$ tale che sia $v \i V_1$. $\varphi(v) \in Im(\varphi) \Longrightarrow \exists \: \overline{\lambda_1}, \cdots, \overline{\lambda_s}$ tale che $\varphi(v) = \overline{\lambda_1}w_1 + \cdots + \overline{\lambda_s}w_s$. Ma allora $\varphi(v - \overline{\lambda_1}\overline{v_1} - \cdots - \overline{\lambda_s}\overline{v_s}) = \varphi(v) - \overline{\lambda_1}\varphi(\overline{v_1}) - \cdots - \overline{\lambda_s}\varphi(\overline{v_s}) = 0$. Quindi $v - \overline{\lambda_1}\overline{v_1} - \cdots - \overline{\lambda_s}\overline{v_s} \in Ker(\varphi)$, ma allora $v - \overline{\lambda_1}\overline{v_1} - \cdots - \overline{\lambda_s}\overline{v_s} = \lambda_1 v_1 + \cdots + \lambda_r v_r$ $\forall \: \lambda_{1}, \cdots, \lambda_r$ perché $v_1, \cdots, v_r$ base di $Ker(\varphi)$. In somma $v = \lambda_1 v_1 + \cdots + \lambda_r v_r + \overline{\lambda_1}\overline{v_1} + \cdots + \overline{\lambda_s}\overline{v_s}$
\end{demostration}
\begin{observation}
	Supponiamo che $v_1, \hdots, v_n$ sia una base di $V_1$. Allora $\varphi$ è unicamente determinata dai valori $\varphi(v_1), \hdots, \varphi(v_n)$. \\
	Infatti $\forall v \in V_1 $ si scrive in modo unico come $v=\lambda_1v_1 + \hdots + \lambda_n v_n$. Ma allora
	\begin{equation*}
		\varphi(v)  = (\lambda_1v_1 + \hdots + \lambda_n v_n) = \varphi(\lambda_{1}v_1) + \varphi(\lambda_nv_n) = \lambda_1\varphi(v_1) + \hdots + \lambda_n\varphi(v_n)
	\end{equation*}
	Viceversa, dati vettori $w_1, \hdots, w_n \in V_2$, $\exists \varphi: V_1 \rightarrow V_2$ lineare tale che $\varphi(v_1)=w_1, \hdots, \varphi(v_n) = w_n$ perché $\varphi(v)$ deve essere $v=\lambda_1v_1 + \hdots + \lambda_n v_n = \lambda_{1}w_1 + \hdots + \lambda_nw_n$.
\end{observation}
\begin{example}
	Troviamo $\varphi:\mathbf{R}^2\rightarrow\mathbf{R}^2$ tale che:
	\begin{equation*}
		\varphi \bigg(
		\begin{bmatrix}
			1 \\ 0
		\end{bmatrix}
		\bigg) = 
		\begin{bmatrix}
			0 \\ 1
		\end{bmatrix}
		\text{ e }
		\varphi \bigg(
		\begin{bmatrix}
			0 \\ 1
		\end{bmatrix}
		\bigg) = 
		\begin{bmatrix}
			1 \\ 0
		\end{bmatrix}
	\end{equation*}
	Un elemento generico di $\mathbf{R}^2$ è:
	\begin{equation*}
		\begin{bmatrix}
			a_1 \\ a_2
		\end{bmatrix}
		= a_1
		\begin{bmatrix}
			1 \\ 0
		\end{bmatrix}
		+ a_2
		\begin{bmatrix}
			0 \\ 1
		\end{bmatrix}
	\end{equation*}
	Allora abbiamo che:
	%TODO Sei rimasto indietro
	
\end{example}
\begin{example}
	Troviamo $\varphi:\mathbf{R}^2\rightarrow\mathbf{R}^2$ tale che:
	\begin{equation*}
		\varphi \bigg(
		\begin{bmatrix}
			1 \\ 0
		\end{bmatrix}
		\bigg) = 
		\begin{bmatrix}
			0 \\ 0
		\end{bmatrix}
		\text{ e }
		\varphi \bigg(
		\begin{bmatrix}
			0 \\ 1
		\end{bmatrix}
		\bigg) = 
		\begin{bmatrix}
			0 \\ 1
		\end{bmatrix}
	\end{equation*}
	Usiamo l'elemento generico di $\mathbf{R}^2$ dell'esempio precedente e abbiamo che:
	%TODO Sei rimasto indietro
\end{example}
\begin{observation}
	Sappiamo che $dim(V_1) = dim(V_2) = n$. Sia $a_1, \hdots, a_n$ base di $V_1$ e $a'_1, \hdots, a'_n$ base di $V_2$. \\
	Sappiamo che $\exists \varphi: V_1 \rightarrow V_2$ tale che $\varphi(a_1) = a'_1, \hdots, \varphi(a_n)=a'_n$ e $\exists \psi: V_2 \rightarrow V_1$ tale che $\varphi(a'_1) = a_1, \hdots, \varphi(a'_n)=a_n$ %TODO Probabilmente hai sbagliato a copiare e sei pure rimasto indietro
\end{observation}
\begin{example}
	Dati $V_1 = M_{2 \times 2}(\mathbf{R})$ e $V_2 = \mathbf{R}^4$ e le loro basi:
	\begin{equation*}
		\text{Base di }V_1:
		\begin{bmatrix}
			1 & 0 \\
			0 & 0
		\end{bmatrix},
		\begin{bmatrix}
			0 & 1 \\
			0 & 0
		\end{bmatrix},
		\begin{bmatrix}
			0 & 0 \\
			1 & 0
		\end{bmatrix},
		\begin{bmatrix}
			0 & 0 \\
			0 & 1
		\end{bmatrix}
		\text{Base di }V_2:
		\begin{bmatrix}
			1 \\ 0 \\ 0 \\ 0
		\end{bmatrix},
		\begin{bmatrix}
			0 \\ 1 \\ 0 \\ 0
		\end{bmatrix},
		\begin{bmatrix}
			0 \\ 0 \\ 1 \\ 0
		\end{bmatrix},
		\begin{bmatrix}
			0 \\ 0 \\ 0 \\ 1
		\end{bmatrix}
	\end{equation*}
	Abbiamo quindi che:
	\begin{equation*}
		\varphi \bigg(
		\begin{bmatrix}
			1 & 0 \\
			0 & 0
		\end{bmatrix}
		\bigg)
		=
		\begin{bmatrix}
			1 \\ 0 \\ 0 \\ 0
		\end{bmatrix},
		\varphi \bigg(
		\begin{bmatrix}
			0 & 1 \\
			0 & 0
		\end{bmatrix}
		\bigg)
		=
		\begin{bmatrix}
			0 \\ 1 \\ 0 \\ 0
		\end{bmatrix},
		\varphi \bigg(
		\begin{bmatrix}
			0 & 0 \\
			1 & 0
		\end{bmatrix}
		\bigg)
		=
		\begin{bmatrix}
			0 \\ 0 \\ 1 \\ 0
		\end{bmatrix},
		\varphi \bigg(
		\begin{bmatrix}
			0 & 0 \\
			0 & 1
		\end{bmatrix}
		\bigg)
		=
		\begin{bmatrix}
			0 \\ 0 \\ 0 \\ 1
		\end{bmatrix}
	\end{equation*}
	\begin{equation*}
		\varphi \bigg(
		\begin{bmatrix}
			a_{11} & a_{12} \\
			a_{21} & a_{22}
		\end{bmatrix}
		\bigg)
		=
		\begin{bmatrix}
			a_{11} \\ a_{12} \\
			a_{21} \\ a_{22}
		\end{bmatrix}
		\text{ Mentre l'inversa è: }
		\varphi \bigg(
		\begin{bmatrix}
			a_{11} \\ a_{12} \\
			a_{21} \\ a_{22}
		\end{bmatrix}
		\bigg)
		=
		\begin{bmatrix}
			a_{11} & a_{12} \\
			a_{21} & a_{22}
		\end{bmatrix}
	\end{equation*}
\end{example}
%TODO Da fare generalizzazione
\begin{definition}
	Se $dim(V)=n$, esiste un \textbf{isomorfismo} $\varphi: V \xrightarrow{\sim} \mathbf{R}^n$. Se $a_1, \hdots, a_n$ è una base di $V$, poniamo
	\begin{equation*}
		\varphi(a_1)=
		\begin{bmatrix}
			1 \\ 0 \\ \vdots \\ 0
		\end{bmatrix},
		\varphi(a_2)=
		\begin{bmatrix}
			0 \\ 1 \\ \vdots \\ 0
		\end{bmatrix},
		\hdots,
		\varphi(a_n)=
		\begin{bmatrix}
			0 \\ 0 \\ \vdots \\ 1
		\end{bmatrix}
	\end{equation*}
	Otteniamo quindi $\forall v \in V_1$:
	\begin{equation*}
		v = \lambda_{1}a_1 + \ldots + \lambda_n a_n \leadsto \varphi(v) = \begin{bmatrix}
			\lambda_{1} \\
			\lambda_{2} \\
			\vdots \\
			\lambda_{n}
		\end{bmatrix}
		\text{ Inversa: } \psi : 
		\begin{bmatrix}
			\lambda_{1} \\
			\vdots
			\lambda_{n}
		\end{bmatrix}
		\mapsto
		\lambda_{1}a_1 + \ldots + \lambda_{n} a_n.
	\end{equation*}
\end{definition}

%TODO Problema
\begin{example}
	Sia $\varphi: \mathbf{R}^n \rightarrow \mathbb{R}^m$ un'applicazione lineare. Conoscendo i valori di $\varphi$ sulla base standard, come si calcola $\varphi(v)$ per $v \in \mathbf{R}^n$ generale?\\
	Ipotizziamo che $n = m = 2$. Conosciamo
	\begin{equation*}
		\varphi \bigg(
		\begin{bmatrix}
			1 \\ 0
		\end{bmatrix}
		\bigg) = 
		\begin{bmatrix}
			a_{11} \\
			a_{21}
		\end{bmatrix}
		\varphi \bigg(
		\begin{bmatrix}
			0 \\ 1
		\end{bmatrix}
		\bigg) = 
		\begin{bmatrix}
			a_{12} \\
			a_{22}
		\end{bmatrix}
	\end{equation*}
	Se abbiamo un vettore generale
	\begin{equation*}
		v = \begin{bmatrix}
			b_1 \\ b_2
		\end{bmatrix} \in \mathbf{R}^2
	\end{equation*}
	allora
	\begin{equation*}
		\begin{bmatrix}
			b_1 \\ b_2
		\end{bmatrix} = 
		b_1 \begin{bmatrix}
			1 \\ 0
		\end{bmatrix} + 
		b_2 \begin{bmatrix}
			0 \\ 1
		\end{bmatrix} = 
		b_1
		\varphi \bigg(
		\begin{bmatrix}
			1 \\ 0
		\end{bmatrix} \bigg) +
		b_2 \varphi \bigg( \begin{bmatrix}
			0 \\ 1
		\end{bmatrix}
		\bigg) =
		b1 \begin{bmatrix}
			a_{11} \\ a_{21}
		\end{bmatrix} +
		b2 \begin{bmatrix}
			a_{12} \\ a_{22}
		\end{bmatrix} =
		\begin{bmatrix}
			a_{11}b_1 + a_{12}b_2 \\
			a_{21}b_1 + a_{22}b_2 
		\end{bmatrix}
	\end{equation*}
\end{example}
\begin{definition}[Prodotto di una matrice e un vettore colonna]
	Sia $A \in M_{m \times m}(\mathbf{R})$, $v \in \mathbb{R}^n$, il loro prodotto è
	\begin{equation*}
		\begin{bmatrix}
			a_{11}, \ldots, a_{1n} \\
			\vdots \\
			a_{m1}, \ldots, a_{mn}
		\end{bmatrix}
		\begin{bmatrix}
			b_1 \\
			\vdots \\
			b_n
		\end{bmatrix}
		=
		\begin{bmatrix}
			a_{11}b_{1} + a_{12}b_{2} + \ldots a_{1n}b_n\\
			a_{21}b_{1} + a_{22}b_{2} + \ldots a_{2n}b_n\\
			\vdots \\
			a_{m1}b_{1} + a_{m2}b_{2} + \ldots a_{mn}b_n\\
		\end{bmatrix} \in \mathbf{R}^m
	\end{equation*}
\end{definition}
\begin{example}
	Dati $\varphi: \mathbb{R}^3 \rightarrow \mathbf{R}$, $\varphi \bigg(\begin{bmatrix}
		x \\ y \\ z
	\end{bmatrix}\bigg) = x + 2y + 3z$. Troviamo $\varphi \bigg(\begin{bmatrix}
1 \\ -1 \\ 1
\end{bmatrix}\bigg) = 1 + 2 \cdot (-1) + 3 \cdot 1$.
Ma anche:
\begin{equation*}
	\varphi\bigg(
	\begin{bmatrix}
		1 \\ 0 \\ 0
	\end{bmatrix}
	\bigg) = 1,
	\varphi\bigg(
	\begin{bmatrix}
		0 \\ 1 \\ 0
	\end{bmatrix}
	\bigg) = 2,
	\varphi\bigg(
	\begin{bmatrix}
		10 \\ 0 \\ 1
	\end{bmatrix}
	\bigg) = 3
\end{equation*}
Matrice di $\varphi$: $A \in M_{1 \times 3} (\mathbf{R})$, $A = [1, 2, 3]$
\begin{equation*}
	A \cdot \begin{bmatrix}
		1 \\ -1 \\ 1
	\end{bmatrix} = [1, 2, 3] \cdot \begin{bmatrix}
	1 \\ -1 \\ 1
\end{bmatrix} = 1 \cdot 1 + 2 \cdot (-1) + 3 \cdot 1 = 2
\end{equation*}
\end{example}
\begin{example}
	Dati $\varphi: \mathbf{R}^2 \rightarrow \mathbf{R}^2$, $\varphi\bigg(\begin{bmatrix}
		x \\ y
	\end{bmatrix}\bigg) = \begin{bmatrix}
	x+y \\ x - y
\end{bmatrix}$, $\varphi\bigg(\begin{bmatrix}
1 \\ 1
\end{bmatrix}\bigg) = \begin{bmatrix}
2 \\ 0
\end{bmatrix}$\\
Matrice di $\varphi$:
%TODO Ti sei perso di nuovo
\end{example}
%TODO Sarebbe una generalizzazione
\begin{definition}
	Sia $\varphi: V \rightarrow W$ un'applicazione lineare dove $dim(V) = n$ e $dim(W)=m$.\\
	Sia $B = \{a_1, \ldots, a_n\}$ base di $V$ e $B' = \{a'_1, \ldots, a'_m\}$ base di $W$.
	Scriviamo
	\begin{equation*}
		\varphi(a_1) = a_{11}a'_1 + a_{21}a'_2 + \ldots + a_{m1}a'_m
		\varphi(a_2) = a_{12}a'_1 + a_{22}a'_2 + \ldots + a_{m2}a'_m
		\vdots
		\varphi(a_n) = a_{1n}a'_1 + a_{2n}a'_2 + \ldots + a_{mn}a'_m
	\end{equation*}
	La matrice di $\varphi$ rispetto alle basi $B$, $B'$ è:
	\begin{equation*}
		A = \begin{bmatrix}
			a_{11} & a_{12} & \ldots & a_{1n} \\
			a_{21} & a_{22} & \ldots & a_{2n} \\
			\vdots \\
			a_{m1} & a_{m2} & \ldots & a_{mn}
		\end{bmatrix} = \begin{bmatrix}
		\varphi(a_1) & \vdots & \varphi(a_2) & \vdots & \ldots & \vdots & \varphi(a_n)
	\end{bmatrix}
	\end{equation*}
\end{definition}
\begin{theorem}
	Se $v=b_1a_1 + \ldots + b_na_n$ è un vettore di $V$. Le coordinate di $\varphi(v)$ rispetto alla base $B'$ sono date dal vettore
	\begin{equation*}
		A: \begin{bmatrix}
			b_1 \\
			\vdots \\
			b_n
		\end{bmatrix} \in \mathbf{R}^m
	\end{equation*}
\end{theorem}
%TODO Tutti i mathbf sono bb
\begin{example}
	Dati $\varphi: \mathbb{R}^2 \rightarrow \mathbb{R}$ e $\varphi \bigg( \begin{bmatrix}
		x \\ y
	\end{bmatrix}\bigg) = x + 2y$. \\
La matrice di $\varphi$ rispetto alla base standard di $\mathbb{R}^2$ in partenza è:
\begin{equation*}
	\varphi\bigg(\begin{bmatrix}
		1 \\ 0
	\end{bmatrix}\bigg) = 1
	\varphi\bigg(\begin{bmatrix}
		0 \\ 1
	\end{bmatrix}\bigg) = 2
	A = [1, 2] \in M_{1 \times 2}(\mathbb{R})
\end{equation*}
Se considero la base $\begin{bmatrix}
	1 \\ 0
\end{bmatrix}$, $\begin{bmatrix}
1 \\ 1
\end{bmatrix}$ di $\mathbb{R}^2$ in partenza:
\begin{equation*}
	\varphi\bigg(\begin{bmatrix}
		1 \\ 0
	\end{bmatrix}\bigg) = 1,
	\varphi\bigg(\begin{bmatrix}
		1 \\ 1
	\end{bmatrix}\bigg) = 3
	
\end{equation*}
\end{example}