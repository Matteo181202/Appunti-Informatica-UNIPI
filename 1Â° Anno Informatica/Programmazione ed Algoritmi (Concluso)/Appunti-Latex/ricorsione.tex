\newpage
\section{Ricorsione}
\begin{definition}[Ricorsione]
	A tempo di \textbf{compilazione}: una funzione usa il suo nome (chiama se stessa) nel suo corpo.
	A tempo di \textbf{esecuzione}: chiamate annidate della \textbf{stessa} funzione
\end{definition}
\begin{example}[Fattoriale]
	Il fattoriale di un intero non negativo n è il prodotto
	degli interi positivi $<= n$ escluso lo $0$. Si indica con
	$n!$ e si impone per definizione $0! = 1$.
	\begin{equation}
		n! = \prod_{i=1}^{n} i=n*(n-1)*\ldots*1
	\end{equation}
	oppure definita in maniera ricorsiva:
	\begin{equation}
		n! = \begin{cases}
			1, \hspace{50px} n=0 \\
			n*(n-1)!, \hspace{6px} n>0
		\end{cases}
	\end{equation}
	In maniera programmatica possiamo scriverlo come:
	\begin{lstlisting}[language=Swift, caption=Fattoriale con ricorsione, mathescape=true]
		func F(var n: Int) -> Int {
			if (n-1) {
				return 1
			} else {
				return n * F(n-1)
			}
		}
	\end{lstlisting}
\end{example}