\newpage
\section{Induzione}
L'induzione è un metodo formale che permette di definire in modo rigoroso insiemi, funzioni. Serve anche per dimostrare che una proprietà è vera per tutti gli elementi di un determinato insieme. Diventa sopratutto utile quando queste definizioni sono insiemi, funzioni di cardinalità o di lunghezza notevolmente grande (ma sempre finiti).

\subsection{Sommatorie, produzioni}
\subsubsection{Sommatoria}
\begin{definition}
	Sia $a:\mathbf{N}^+ \rightarrow \mathbf{N}$ una successione di anturali $a_1, a_2, \ldots, a_n$. La sommatoria degli $a_i$ per $i$ che va da $1$ a $n$ è:
	\begin{equation}
		\sum\limits_{i=1}^n a_i = a_1 + a_2 + \ldots + a_n
	\end{equation}
\end{definition}
Possiamo definirla come di seguito in maniera induttiva:
\begin{itemize}
	\item Clausola base: $\sum\limits_{i=1}^0 a_i = 0$
	\item Clausola induttiva: $\sum\limits_{i=1}^{n+1} a_i = \bigg(\sum\limits_{i=1}^n a_i) + a_{n+1}$
\end{itemize}

\subsubsection{Sommatoria da \emph{k}}
\begin{definition}
	Sia $a:\mathbf{N}^+ \rightarrow \mathbf{N}$ una successione di naturali $a_1, a_2, \ldots, a_n$. Sia $k \in \mathbf{N}^+$. La sommatoria degli $a_i$ per $i$ che va da $k$ a $n$ è:
	\begin{equation}
		\sum\limits_{i=k}^n a_i = a_1 + a_2 + \ldots + a_n
	\end{equation}
\end{definition}
Possiamo definirla come di seguito in maniera induttiva:
\begin{itemize}
	\item Clausola base: $\sum\limits_{i=k}^0 a_i = 0$
	\item Clausola induttiva:
	$ \sum\limits_{i=k}^{n+1} a_i = 
	\begin{cases}
		0 & \text{se } k > n+1 \\
		\bigg(\sum\limits_{i=1}^n a_i) + a_{n+1} & \text{se } k \leq n+1
	\end{cases}$
\end{itemize}

\subsubsection{Produttoria da \emph{k}}
\begin{definition}
	Sia $a:\mathbf{N}^+ \rightarrow \mathbf{N}$ una successione di naturali $a_1, a_2, \ldots, a_n$. Sia $k \in \mathbf{N}^+$. La produttoria degli $a_i$ per $i$ che va da $k$ a $n$ è:
	\begin{equation}
		\prod\limits_{i=k}^n a_i = a_1 \cdot a_2 \cdot \ldots \cdot a_n
	\end{equation}
\end{definition}
Possiamo definirla come di seguito in maniera induttiva:
\begin{itemize}
	\item Clausola base: $\sum\limits_{i=k}^0 a_i = 0$
	\item Clausola induttiva: $\sum\limits_{i=k}^{n+1} a_i = \begin{cases}
		0 & \text{se } k>n+1 \\
		\bigg(\sum\limits_{i=1}^n a_i) \cdot a_{n+1} & \text{se } k \leq n+1
	\end{cases}$
\end{itemize}

%TODO Inserire de morgan n-ario e intersezione n-aria

\subsection{Schema generale induttivo}
La definizione per induzione di un insieme $A$ sfrutta uno schema generale che permette di rappresentare infinite soluzioni:

\begin{enumerate}
    \item \textbf{Clausola base}, o caso base, che stabilisce che certi oggetti appartengono all'insieme. Questi elementi costituiscono i mattoncini per costruire altri elementi dell’insieme. Nel nostro caso andiamo ad elencare un numero finito di elementi che appartengono ad $A$.
    \item \textbf{Clausola induttiva} o passo induttivo, che descrive in che modo gli elementi dell'insieme possono essere usati per produrre altri elementi dell'insieme. Nel nostro caso usiamo elementi di $A$ per costruirne o definirne altri.
    \item \textbf{Clausola terminale}, che stabilisce che l'insieme che si sta definendo non contiene altri elementi oltre a quelli ottenuti dalle due clausole precedenti. Quindi l'insieme definito è il più piccolo insieme che soddisfa la clausola base e quella induttiva. Nel nostro caso definiamo che gli unici elementi che soddisfano A sono quelli definiti nelle condizioni $1$ e $2$. \footnote{La clausola terminale di solito non viene specificata essendo sotto intesa}
\end{enumerate}
\begin{example}
    Riscrivendo per intero in nostro esempio con A:\\
    \underline{Clausola base:} $A_1 \times A_2$ è una n-upla\\
    \underline{Clausola induttiva:} $A_1 \times A_2 \times A_3$ è una n-upla\\ \\
    Tramite queste due clausole possiamo andare a ricreare ogni n-upla (2-upla, 3-upla, ecc..) andando semplicemente a ripetere la clausola induttiva partendo da quella base. Infatti una 3-upla è clausola base per 1 volta clausola induttiva, mentre una 4-upla è la clausola base più due ripetizioni della clausola induttiva: \\
    3-upla = $A_1 \times A_2 \times A_3$ \hspace{.5cm} 4-upla = $A_1 \times A_2 \times A_3 \times A_4$ \hspace{.5cm} e così per ogni n-upla
\end{example}
\begin{example}
    Altri esempi particolari di casi induttivi:
    \begin{itemize}
        \item \textbf{Chiusura di kleene} o stella di kleene: $A^{*} = \bigcup\limits_{n=0}^{\infty} A^n$ \hspace{.3cm} $A^n = A \times A \times A \ldots$ ($n$ volte). Parte da $0$ perché si considera quella che in informatica sarebbe la stringa vuota, cioè $A^0 = \{\}$, che si rappresentata con $\epsilon$.
        \item \textbf{Chiusura positiva}: $A^{+} = \bigcup\limits_{n=1}^{\infty} A^n$ \hspace{.3cm} Uguale alla stella di kleene ma senza insieme vuoto.
    \end{itemize}
    Entrambi questi casi sono operazioni molto consuete nel mondo dell'informatica, essendo che possiamo vedere $A$ come un insieme di caratteri per esempio $A = unicode$, e tramite queste operazioni si vanno a creare tutte le possibili sequenze o stringhe con quei caratteri.
\end{example}

\subsection{Definizione induttiva dell'insieme $\mathbb{N}$}
\begin{definition}[Definizione induttiva insieme $\mathbb{N}$]\label{definizioni-induttiva-insieme-N}
    L’insieme N dei numeri naturali è l’insieme di numeri che soddisfa le seguenti clausole:
    \begin{enumerate}
        \item \underline{Clausola base:} $0 \in \mathbb{N}$.
        \item \underline{Clausola induttiva:} se $n \in \mathbb{N}$ allora $(n + 1) \in \mathbb{N}$.
    \end{enumerate}
\end{definition}
Definendo le tre clausole è immediato capire la definizione dell'insieme $\mathbb{N}$ per induzione. Infatti possiamo ricavare qualsiasi numero dell'insieme semplicemente partendo da 0 (che già sappiamo per clausola base appartenere all'insieme) e ripetendo la clausola un numero illimitato \footnote{Illimitato è diverso da infinito in quanto quest'ultimo non è un numero mentre il primo si} di volte. 

\begin{example}
    Altri esempi analoghi alla definizione dell'insieme $\mathbb{N}$:
    \begin{enumerate}
        \item Definizione \textbf{N. pari}: \hspace{.3cm} \underline{Base:} $2 \in \mathbb{N}^{pari}$ \hspace{.3cm} \underline{Induttiva:} se $n \in \mathbb{N}^{pari}$ allora $n+2 \in \mathbb{N}^{pari}$
        \item Definizione \textbf{N. dispari}: \hspace{.1cm} \underline{Base:} $1 \in \mathbb{N}^{dispari}$ \hspace{.2cm} \underline{Induttiva:} se $n \in \mathbb{N}^{dispari}$ allora $n+2 \in \mathbb{N}^{dispari}$
        \item Definizione \textbf{Potenze 2}: \hspace{.2cm} \underline{Base:} $1 \in$ P \hspace{.2cm} \underline{Induttiva:} se $p \in$ P allora $2*p \in$ P
        
        È possibile scrivere anche come: \hspace{.2cm} \underline{Base:} $2^0 \in P$ \hspace{.2cm} \underline{Induttiva:} se $2^n \in P$ allora $2^n+1 \in P$
    \end{enumerate}
\end{example}

\subsection{Definizione induttiva di funzioni}
Per andare ad effettuare la definizione induttiva di una funzione bisogna andare (1) a stabilire il valore delle funzioni per gli elementi appartenenti alla clausola base e (2) una regola per andare a calcolare il valore della funzione sugli elementi che vi appartengono, stabiliti dalla clausola induttiva. Successivamente tramite te la clausola terminale definiamo che i punti (1) e (2) sono sufficienti a definire la funzione per tutti gli elementi dell'insieme. Quindi, prendendo una $f: A \rightarrow B$:
\begin{itemize}
    \item La \textbf{clausola base} sarà il valore di f(a) per alcuni $a \in A$.
    \item La \textbf{clausola induttiva} invece indicherà il valore di f(a) utilizzando valori di f già definiti in precedenza.
\end{itemize}
\begin{example}
    Facciamo un primo esempio definendo la funzione dei \textbf{numeri triangolari}.
    \begin{definition}[Numeri triangolari]
        Per ogni $n \in \mathbb{N}$ in numero triangolare $T_n$ è uguale alla somma di tutti i numeri minori o uguali a n:
        \begin{equation}\label{numeri-triangolari}
            T_n: \mathbb{N} \rightarrow \mathbb{N} = fun(\mathbb{N},\mathbb{N}) \hspace{1cm} T_n = \sum_{\substack{i=0}}^n i
        \end{equation}
    \end{definition}
    \begin{enumerate}
        \item \underline{Clausola base:} $T_0 = 0$
        \item \underline{Passo induttivo:} $T_{n + 1} = T_n + (n + 1)$
    \end{enumerate}
    Per dimostrare ciò possiamo procedere per casi andando a controllare $n=1, n=2, n=3, \ldots$ e dimostrando la validità per questi casi, cosa che però non porta ad un risultato per un generico numero $n$. Per dimostrare la validità bisogna dimostrare quella che vine chiamata \textit{formula di Gauss}.
    \begin{equation}\label{formula-gauss}
        \forall n \in \mathbb{N} . \bigg(T_n = \frac{n \cdot (n + 1)}{2}\bigg)
    \end{equation}
\end{example}

\subsection{Dimostrazione induttiva di proprietà}
Per andare a dimostrare una proprietà bisogna applicare il principio di induzione sui numeri naturali, che, dato una generica proprietà $P(n)$ sui naturali dice:
\begin{definition}[\textbf{Principio di induzione}]
    Se (caso base) $P(0)$ è vera, e se (passo induttivo) per ogni $n \in \mathbb{N}$ vale che se $P(n)$ è vera allora anche $P(n+1)$ lo è, allora $P(m)$ è vera per ogni $m \in \mathbb{N}$
    \begin{equation}
        \frac{P(0) \wedge \forall n \in \mathbb{N}.(P(n) \Rightarrow P(n+1))}{\forall m \in \mathbb{N}.P(m)}
    \end{equation}
\end{definition}
\begin{proposition}[Formula di gauss]
La formula (\ref{formula-gauss}), $T_n = \frac{n \cdot (n + 1)}{2}$, è valida per ogni $n \in \mathbb{N}$
\end{proposition}
\begin{demostration}[Formula di gauss]
Per dimostrare la formula di gauss (\ref{formula-gauss}) per induzione è sufficiente dimostrare i seguenti casi:
\begin{enumerate}
    \item \underline{Caso base:} $T_0 = 0$, perché per definizione sarebbe $\frac{0 \cdot (0 + 1)}{2} = 0$, quindi la formula di gauss (\ref{formula-gauss}) per $n = 0$ è vera.
    \item \underline{Passo induttivo:} Assumiamo che l'ipotesi $T_n = \frac{n \cdot (n + 1)}{2}$ sia vera e dimostriamo che è vera per ogni $n + 1$, cioè che:\\\\
    $T_{n+1} = T_n + (n + 1)$ \hspace{.5cm} Possibile per il passo induttivo della definizione dei numeri triangolari \ref{numeri-triangolari} \\\\
    $= \frac{n \cdot (n + 1)}{2} + (n + 1)$ \hspace{.5cm} Sostituiamo a $T_n$ la clausola induttiva\\\\
    $= \frac{n \cdot (n + 1) + 2 \cdot (n + 1)}{2}$ \hspace{.5cm} Sviluppiamo l'equazione\\\\
    $= \frac{(n + 2) \cdot (n + 1)}{2}$  \hspace{.5cm} Dimostrato il caso $T_{n+1}$ la formula è dimostrata per induzione. \: \: $\blacksquare$
\end{enumerate}
\end{demostration}
\begin{example}[De Morgan]
    Facciamo un altro esempio andando a dimostrare la proprietà di \textbf{De Morgan} per induzione:
    \begin{center}
        $\overline{(A \cup B)} = \overline{A} \cap \overline{B}$
    \end{center}
    Noi dobbiamo dimostrare che questa legge può valere per più insiemi utilizzando l'induzione. Quindi dobbiamo far valere la generalizzazione:
    \begin{center}
        $DM(n) = \forall A_1, \ldots , A_n \cdot \Bigg(\overline{\Big( \bigcup\limits_{i=1}^{n} A_i \Big)} = \bigcap\limits_{i=1}^{n}\overline{A_i} \Bigg)$
    \end{center}
    Noi a questo punto dobbiamo in particolare andare a dimostrare che per ogni $n \geq 2$ vale $DM(n)$:
    \begin{center}
        $\forall n . (n \geq 2 \Longrightarrow DM(n))$
    \end{center}
    \begin{enumerate}
        \item \underline{Caso base:} $DM(2)$ diche che $\overline{(A_1 \cup A_2)} = \overline{A_1} \cap \overline{A_2}$, ed è dimostrato perché è ciò che dice la legge di De Morgan.
        \item \underline{Passo induttivo:} Assumiamo che $DM(n)$ sia vera e dimostriamo che vale anche $DM(n+1)$. Noi quindi vogliamo dimostrare che la seguente proprietà sia vera:
        \begin{center}
            $DM(n+1) = \forall \: \: A_1, ..., A_{n+1} \: \cdot \: \Bigg(\overline{\Big( \bigcup\limits_{i=1}^{n+1} A_i \Big)} = \bigcap\limits_{i=1}^{n+1}\overline{A_i} \Bigg)$
        \end{center}
        Se prendiamo $n+1$ che è un insieme di $A_1, A_2, ...A_n, A_{n+1}$ notiamo che:\\ \\
        $\overline{\Big( \bigcup\limits_{i=1}^{n+1} A_i \Big)} = \overline{\Big(A_{n+1} \cup \bigcup\limits_{i=1}^{n} A_i \Big)}$ \hspace{.5cm} Perché possiamo usare la proprietà associativa per $\cup$.\\ \\
        $= \overline{A_{n+1}} \cap \overline{\Big(\bigcup\limits_{i=1}^{n} A_i \Big)}$ \hspace{.5cm} Utilizzando De Morgan\\\\
        $= \overline{A_{n+1}} \cap \Big(\bigcap\limits_{i=1}^{n} \overline{A_i} \Big)$ \hspace{.5cm} Per ipotesi induttiva.\\\\
        $= \Big(\bigcup\limits_{i=1}^{n+1} \overline{A_i} \Big)$ \hspace{.5cm} Possiamo racchiudere tutto sotto la concatenazioni di intersezioni. \\ \\
        Vediamo quindi che anche il $DM(n+1)$ è dimostrato, quindi che tutte le sequenze $DM(n)$ sono dimostrate. \: \: $\blacksquare$
    \end{enumerate}
\end{example}
\begin{example}[Numeri di Fibonacci]
    La successione dei numeri di Fibonacci può essere definita induttivamente come:
    \begin{enumerate}
        \item Caso base: $f_1 = 1$
        \item Caso base: $f_2 = 1$
        \item Passo induttivo: $f_n = f_{n-1} + f_{n-2}$ \hspace{.5cm} se $n > 2$
    \end{enumerate}
    \begin{demostration}[Numeri di Fibonacci]
        Proviamo a dimostrare questa successione numerica. Sia $f_i$ l'i-esimo numero di Fibonacci, noi dobbiamo dimostrare per induzione che:
        \begin{equation}
            \forall n \in \mathbb{N}^{+}.\bigg(\sum_{\substack{i=1}}^n f_i^2 = f_n \cdot f_{n+1} \bigg)
        \end{equation}
        \begin{enumerate}
            \item \underline{Caso base:} Poiché noi dobbiamo dimostrare per tutti gli $n$ tale che $n \in \mathbb{N}^{+}$ per il caso base dobbiamo considerare $n = 1$, quindi il nostro caso base sarà $\sum_{i=1}^1 f_i^{2} = f_1 \cdot f_{2}$. questo caso è dimostriamo immediatamente dalla definizione di $f_i$
            \item \underline{Passo induttivo:} Assumiamo per ipotesi induttiva che valga $\sum_{i=1}^n f_i^{2} = f_n \cdot f_{n+1}$
        \end{enumerate}
        Una volta assunto per vero il passo induttivo possiamo dimostrare il caso $n+1$, ciò porta a dimostrare che:\\ \\
        $\sum_{i=1}^{n+1}\:f_i^{2} = \sum_{i=1}^{n}\:f_n^2 \: + \: f_{n+1}^2$ \hspace{.3cm} Togliamo l'ultimo termine dalla sommatoria e scriviamolo esplicitamente. \\ \\
        $= f_n^2 \: \cdot \: f_{n+1}^2 + \: f_{n+1}^2$ \hspace{.3cm} Applichiamo l'ipotesi induttiva \\ \\
        $= (f_n^2 \: + \: f_{n+1}^2) \: \cdot \: f_{n+1}$  \hspace{.3cm} Proprietà distributiva\\ \\
        $= f_{n+2}^2 \: \cdot \: f_{n+1}$  \hspace{.3cm} Sviluppiamo l'equazione \\\\
        Vediamo come $f_{n+2}^2 \: \cdot \: f_{n+1}$ sia la causala induttiva che volevamo dimostrare. Possiamo dire dunque che la formula di Fibonacci e dimostrata. $\blacksquare$
    \end{demostration}
\end{example}

\subsection{Principio di induzione forte sui naturali}
Il Principio di Induzione Forte sui naturali ci permette di rafforzare le ipotesi del passo induttivo e portare avanti la dimostrazione in modo più semplice.
\begin{definition}[Induzione Forte]
    Se per ogni $n \in \mathbb{N}$ vale che se $P(0), P(1),\ldots, P(n-1)$ sono vere allora anche $P(n)$ lo è, allora $P(m)$ è vera per ogni $m \in \mathbb{N}$.
    \begin{equation}
        \frac{\forall n . (P(0) \: \land P(1) \land ... \land P(n-1) \rightarrow P(n))}{\forall m . P(m)}
    \end{equation}
\end{definition}
\begin{example}
	Per dimostrare il:
	\begin{definition}[Teorema fondamentale dell'Aritmetica]
		Ogni intero $n$ maggiore di $1$ o è un numero primo oppure può essere scritto come prodotto di numeri primi.
	\end{definition}
	\begin{itemize}
		\item Caso base: $P(2)$: $2$ è primo
		\item Passo induttivo: dimostro $P(m)$ assumendo $P(n)$ valga $\forall n \in \mathbf{N}^+ . n<m$
		\begin{itemize}
			\item Se $m$ è primo allora $P(m)$ vale
			\item Se $m$ non è primo allora ha un fattore \textbf{non banale} $x$ ovvero $m = x \cdot y . x < m \wedge y<m$. Per induzione forte $x \cdot y$ può essere scritto come prodotto di numeri primi quindi $P(m)$ vale
		\end{itemize}
	\end{itemize}
\end{example}