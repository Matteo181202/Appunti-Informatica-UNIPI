\newpage
\section{Garbage collection}
Il sistema che si occupa di ripulire la memoria quando degli oggetti/variabili non sono più utilizzati. Gestisce quindi l'heap della memoria, che è organizzata come segue:
\begin{itemize}
	\item \textit{Static area}: ha dimensione fissa e i suoi contenuti sono determinati ed allocati a compilazione
	\item \textit{Run-time stack}: ha dimensione variabile e gestisce i sottoprogrammi
	\item \textbf{Heap}: può avere dimensione fissa o variabile e gestisce gli oggetti e le strutture dati dinamiche
\end{itemize}
Evita di saturare la memoria, evita problemi alla deallocazione (memory leak, dandling problem) e risolve problemi di sicurezza (elimina dati potenzialmente sensibili rimasti in memoria).
\subsection{Allocazione heap}
Può avvenire in due modi:
\begin{itemize}
	\item Allocazione con blocchi di memoria \textbf{fissa} predifiniti organizzati come una lista. Questo porta a frammentazione quando poi avvengono de-allocazioni e nuovi allocazioni (non c'è una sequenza di blocchi contigua abbastanza grande). Esistono due tecniche:
	\begin{itemize}
		\item \textit{First fit}: si prende il primo blocco grande abbastanza. È più veloce ma spreca più memoria.
		\item \textit{Best fit}: si prende il blocco di dimensione più piccola che sia grande abbastanza. Più lento ma più efficiente in termini di memoria.
	\end{itemize}
	\item Allocazione con blocchi di memoria di dimensione \textbf{variabile}: i blocchi vengono creati strada facendo a seconda della dimensione necessaria e uniti quando sono contigui e di nuovo liberi. Si può comunque presentare frammentazione.
\end{itemize}
\subsection{Frammentazione}
Ne esistono di due tipi:
\begin{itemize}
	\item \textbf{Interna}: quando c'è memoria in eccesso nei blocchi che vengono allocati
	\item \textbf{Esterna}: quando vengono lasciati blocchi di memoria non allocati tra blocchi allocati che sono troppo piccoli per essere utilizzati
\end{itemize}
\subsection{Algoritmi}
Il garbage collector perfetto dovrebbe avere le seguenti caratteristiche:
\begin{enumerate}
	\item Nessun impatto visibile sulle performance
	\item Opera su ogni tipo di programma e struttura dati (anche cicliche)
	\item Individua il garbage in modo efficiente e veloce
	\item Nessun overhead sulla gestione della memoria complessiva (caching e paginazione)
	\item Gestione efficiente dell'heap (nessuna frammentazione)
\end{enumerate}
\subsubsection{Reference counting}
Si tiene traccia del numero dei riferimenti ai blocchi allocati tramite un \textbf{contatore}. Viene inizializzato ad 1 e ad ogni copia del puntatore aumenta di 1 mentre ad ogni deallocazione decrementa. Quando arriva a 0 viene liberata la memoria.\\
Il problema principale che si pone con questa tecnica è il \textbf{memory leak}: ovvero quando alcuni oggetti hanno riferimenti circolari e potrebbero rimanere isolati dal resto dei puntatori ma comunque allocati. Inoltre, essendo necessario spazio per i contatori, si ha un \textbf{overhead di memoria}.
\subsubsection{Tracing}
Sfrutta un \textbf{grafo} di raggiungibilità. Ogni tanto interrompono l'esecuzione del programma e verificano quali riferimenti sono ancora attivi e quali no. Si risolve il problema del \textit{memory leak} ma si presenta un \textbf{overhead di tempo}. Esistono due principali tecniche:
\begin{definition}[mark-sweep]
	Ogni cella prevede un \textbf{bit di marcatura}. Si interrompe periodicamente il programma e si scorre il grafo: parterndo dal \textit{root set} si marcano le celle live (marking), poi tutte le celle non marcate sono deallocate e vengono resettati i bit (sweep).
	\begin{lstlisting}
		// Mark
		For each root v:
			DFS(v)
		
		function DFS(x):
			if x is a pointr into heap
				if record x is not marked
					mark x
					for each field fi of record x
						DFS(x.fi)
		
		// Sweep
		p <- first address in heap
		while p < last address in heap
			if record p is marked
				unmark p
			else let f1 be the first field in p
				p.f1 <- freelist
				freelist <- p
			p <- p + (size of record p)
	\end{lstlisting}
	Funziona sulle strutture circolari, non ha overhead di spazio ma necessita della sospensione dell'esecuzione e non interviene sul problema della frammentazione.
\end{definition}
\begin{definition}[Copy collection - Cheney algorithm]
	Suddividiamo l'heap in due parti:
	\begin{itemize}
		\item \textbf{from-space}: viene utilizzato per allocare inizialmente
		\item \textbf{to-space}
	\end{itemize}
	Ad ogni chiamata del garbage collector viene fatta una visita al grafo e tutti i riferimenti ancora attivi vengono spostati nell'altra area, che poi diventa quella attiva. Ciò che rimanene nella prima viene deallocato.\\
	Il problema principale è l'overhead di memoria in quanto usa solo metà dell'heap. Migliora il problema della frammentazione perché copia in maniera allineata.
\end{definition}
\subsubsection{Generational GC}
Si basa sull'idea \textit{"most cells that die, die young"}. Viene quindi fatta una classificazione in base alla lunghezza della vita. L'heap viene diviso in \textbf{generazioni}, dove le più "giovani" vengono visitate più frequentemente. Ad ogni iterazione se qualcosa è attivo viene spostato in una generazione successiva che viene visitata meno frequentemente. Su varie generazioni vengono utilizzati algoritmi diversi di quelli elencati precedentemente.