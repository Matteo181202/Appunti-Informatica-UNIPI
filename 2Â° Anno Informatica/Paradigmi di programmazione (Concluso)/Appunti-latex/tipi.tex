% !TeX spellcheck = it_IT
\newpage
\section{Sistemi di tipi}
\subsection{Perché?}
Dato che nel lambda calcolo i programmi e i valori sono funzioni
possiamo facilmente scrivere programmi che non sono corretti rispetto
all’uso inteso dei valori. Ad esempio:
\begin{example}[Errore tipi]
	Nella seguente espressione si può applicare $0$ a \textit{False}, ottenendo quindi un risultato che però non ha alcun senso:
	\begin{equation}
		False \: 0 = (\lambda t.\lambda f.f)(\lambda z.\lambda s.z) \rightarrow \lambda f.f
	\end{equation} 
\end{example}
\noindent Analogamente per una macchina tutto è un bit: istruzioni, dati e operazioni. 
Un esempio più pratico è il seguente:
\begin{example}
	L'istruzione nel corpo dell'\textit{if} contiene un errore di tipo (stringa divisa per un intero). Se non avessimo il controllo dei tipi l'unico modo per scoprire l'errore sarebbe eseguire numerosi test per riuscire a coprire tutte le possibilità, fino ad entrare nel corpo dell'\textit{if}. Richiederebbe tempo e risorse e non ci garantisce neanche la certezza di aver provato tutti i casi possibili.
	
	\begin{lstlisting}
			if (condizione_complicata) {
				return "hello"/10;
			}
	\end{lstlisting}
\end{example}
Se in certi linguaggi di programmazione ci troveremmo davanti ad errori di esecuzione, in altri (come ad esempio JavaScript) otterremmo un errore nel risultato in quanto l'interprete proverebbe a fare un cast manuale. \\
Concludendo, la mancanza di \textbf{type safety} aumenta il numero di bug, rendendo così un software meno funzionale e più vulnerabile.

\subsection{Cosa sono i tipi?}
I \textbf{sistemi di tipo} sono meccanismi che permettono di rilevare in anticipo errori di programmazione. 
\begin{definition}[Tipo]
	Il tipo è un \textbf{attributo} di un dato che descrive come il linguaggio di programmazione permetta di usare quel particolare dato.
\end{definition}

\noindent Un tipo serve quindi a limitare i valori che un'espressione può assumere, che operazioni possono essere effettuate sui dati e in che modo questi ultimi possono essere salvati.

\begin{definition}[Sistema dei tipi]
	Un sistema dei tipi è un metodo \textbf{sintattico}, \textbf{effettivo} per dimostrare
	l'assenza di comportamenti anomali del programma \textbf{strutturando} le
	operazioni del programma in base ai tipi di valori che calcolano.
\end{definition}
\noindent Analizziamo i tre aspetti:
\begin{itemize}
	\item \textit{Sintattico}:  l'analisi viene effettuata dal punto di vista sintattico
	\item \textit{Effettivo}: si può definire un algoritmo che effettui questa analisi
	\item \textit{Strutturale}: i tipi assegnati si ottengono in maniera \textbf{composizionale} dalle sottoespressioni.
\end{itemize}

\subsection{Come funziona?}
Un sistema dei tipi associa dei tipi ai valori calcolati. Esaminando il flusso dei valori calcolati prova a dimostrare che non avvengano errori (di tipo, non in generale)facendo un controllo, che può avvenire in due modi:
\begin{itemize}
	\item \textit{Statico}: avviene in fase di compilazione, non degradando le prestazioni
	\item  \textit{Dinamico}: avviene in fase di esecuzione e aumenta il tempo di esecuzione
\end{itemize}

\subsection{Come si progetta?}
\subsubsection{Specifiche del linguaggio}
Prendiamo come esempio il seguente \textbf{linguaggio}:\\
\begin{center}
	\begin{tabular}{|c|c|c|c|}
		\hline
		\textbf{Espressioni} & \textbf{Valori} & \textbf{Valori numerici} & \textbf{Tipi} \\
		\hline
		E::= & V::= & NV::= & T::= \\
		true & true & $0 \vert 1 \vert 2 \vert \ldots$ & Bool \\
		false & false & & Nat \\
		NV & NV & &\\
		if \textit{E} then \textit{E} else \textit{E} & &  &\\
		succ \textit{E} & & &\\
		pred \textit{E} & & &\\
		isZero \textit{E} & & &\\
		\hline
	\end{tabular}
\end{center}
\subsubsection{Regole di valutazione}
Avremo le seguenti \textbf{regole di valutazione}:\\
\begin{gather}
		if \: true \: then \: E1 \: else \: E2 \rightarrow E1 \\
		if \: false \: then \: E1 \: else \: E2 \rightarrow E2 \\
		\frac{E \rightarrow E'}{if \: E \: then \: E1 \: else \: E2 \rightarrow if \: E'  \: then \: E1 \: else \: E2 \rightarrow E1} \label{eq:if_stepfw}
\end{gather}
\begin{equation}
	\frac{E \rightarrow E'}{succ \: E \rightarrow succ \: E'} \qquad
	\frac{m = n+1}{succ \: E \rightarrow succ \: E'}
\end{equation}
\begin{equation}
	\frac{E \rightarrow E'}{pred \: E \rightarrow pred \: E'} \qquad
	\frac{ n>0,\: m=n-1}{pred \: n \rightarrow m} \qquad
	pred \: 0 \rightarrow 0
\end{equation}
\begin{equation}
	\frac{E \rightarrow E'}{isZero \: E \rightarrow isZero \: E'} \qquad isZero \: 0 \rightarrow true \qquad \frac{n>0}{isZero \: n \rightarrow false}
\end{equation}
\subsubsection{Type checking}
Il \textbf{controllo di tipo} definisce una relazione binaria $(E, T)$ che associa il tipo $T$ all'espressione $E$. Questo ha due caratteristiche principali:
\begin{itemize}
	\item Utilizza il \textit{metodo sintattico}
	\item Le regole sono definite per \textit{induzione strutturale} sul programma 
\end{itemize}
Le regole sono le seguenti:
\begin{equation}
	true:Bool \qquad false:Bool \qquad n:Nat
\end{equation}
\begin{equation}
	\frac{E:Nat}{succ \: E : Nat} \qquad \frac{E:Nat}{pred \: E : Nat} \qquad \frac{E:Nat}{isZero\: E : Bool}
\end{equation}
\begin{equation}\label{eq:type_if}
	\frac{E:Bool, E1:T,E2:T}{if \: E \: then \: E1 \: else \: E2}
\end{equation}
\subsubsection{Composizionalità}
I sistemi di tipo sono \textbf{imprecisi}: non definiscono esattamente quale tipo di valore sarà restituito da ogni programma, ma solo un'\textbf{approssimazione conservativa}.
\begin{example}
	La seguente espressione:
	\begin{equation}
		if \: E \: then \: 0 \: else \: false
	\end{equation}
	potrebbe restituire come risultato sia un \textit{Bool} che un \textit{Nat} a seconda del valore di \textit{E}. Il controllo dei tipi quindi non permetterà che possano esserci due risultati diversi, riducendo la precisione ma mantenendo la sicurezza.
\end{example}
Questo avviene proprio per garantire la \textbf{composizionalità}, infatti ad esempio la regola dell'equazione \ref{eq:type_if} necessita che $E1$ ed $E2$ abbiano lo stesso tipo.

\subsection{Dimostrazione}
La \textbf{correttezza} del sistema di tipo è espressa da due proprietà:
\begin{itemize}
	\item Progresso
	\item Conservazione
\end{itemize}

\subsubsection{Progresso}
\begin{definition}[Progresso]
	Se $E:T$ allora $E$ è un valore oppure $E \rightarrow E'$ per una qualche espressione $E'$.
\end{definition}
In pratica, un'espressione ben tipata non si blocca a run-time. Può fare sempre un passo a meno che non sia un valore.
\begin{proof}
	\label{proof_progresso}
	Utilizziamo l'induzione sulla struttura di derivazione di $E:T$.\\
	I \textit{casi base} sono i seguenti:
	\begin{itemize}
		\item $true:Bool$
		\item $false:Bool$
		\item $0 \vert 1 \vert 1 \vert \ldots : Nat$
	\end{itemize}
	I \textit{casi induttivi} sono tutti molto simili, vediamo quello per la formula \ref{eq:type_if}. \\
	Per \textit{ipotesi induttiva} abbiamo due casi:
	\begin{itemize}
		\item $E1$ è un valore. In questo caso deve essere $true$ o $false$ e le regole della semantica fanno fare un \textbf{passo} del tipo $E \rightarrow E1$ o $E \rightarrow E2$
		\item  Esiste $E4$ tale che $E1 \rightarrow E4$. In questo caso si applica la regola \ref{eq:if_stepfw} e si esegue un \textbf{passo}.
	\end{itemize}
\end{proof}

\subsubsection{Conservazione}
\begin{definition}[Conservazione]
	Se $E:T$ e $E \rightarrow E'$ allora $E':T$.
\end{definition}
In pratica i tipi sono preservati dalle regole di esecuzione.
\begin{proof}
	\label{proof_conservazione}
	Utilizziamo l'induzione come nella precedente dimostrazione.\\
	I \textit{casi base} sono immediati: $true$, $false$ e $0 \vert 1 \vert 2 \vert \ldots$ sono valori e di conseguenza non fanno nessun passo. \\
	Anche qui per i \textit{casi induttivi} vediamo quello per la formula \ref{eq:if_stepfw}. Per l'ipotesi induttiva abbiamo due casi:
	\begin{itemize}
		\item $E1$ è un valore: \begin{itemize}
			\item $true$: in questo caso $E \rightarrow E2$ e sappiamo già per ipotesi induttiva che $E2:T$ (sappiamo che il passo ha successo)
			\item $false$: in questo caso $E \rightarrow E3$ e $E3:T$
		\end{itemize}
		\item Esiste $E4$ tale che $E1 \rightarrow E4$. Questo implica:
		\begin{equation*}
			E = if \: E1 \: then \: E2 \: else \: E3 \rightarrow if \: E4 \: then \: E2 \: else \: E3
		\end{equation*}
		Dato che per ipotesi induttiva $E1:Bool$ abbiamo che $E4:Bool$ e, grazie alle derivazioni che valgono per ipotesi $E2:T$ e $E3:T$, possiamo derivare applicando la regola \ref{eq:type_if}.
	\end{itemize}
\end{proof}

\newpage
\subsection{Lambda calcolo tipato}
Nel nostro caso descriveremo il lambda calcolo tipato con valori di tipo \textit{booleano} e \textit{funzione}.
\subsubsection{Sintassi}
\begin{center}
	\begin{tabular}{|c|c|c|}
		\hline
		\textbf{Tipi} & \textbf{Linguaggio} & \textbf{Valori} \\
		\hline
		$\tau ::=$ & $e::=$ & $V::=$\\
		\hline
		$Bool$ & $x$& $fun \: x:\tau = e$\\
		$\tau \rightarrow \tau$ \footnotemark & $fun \: x: \tau = e$& $true$\\
		& $Apply(e,e)$ & $false$\\
		& $true$ & \\
		& $false$ & \\
		& if $e$ then $e$ else $e$ & \\
		\hline
	\end{tabular}
\end{center}
\footnotetext{Nota bene: l'associazione avviene a destra. Quindi $Bool \rightarrow Bool \rightarrow Bool = Bool \rightarrow (Bool \rightarrow Bool)$ }

\subsubsection{Semantica}
Di seguito le regole della semantica del linguaggio:
\begin{gather}
	Apply(fun \: x:\tau = e_1,v) \rightarrow e_1 \{x:=v\} \label{eq_briduz}\\
	\frac{e_1 \rightarrow e'}{Apply(e_1, e_2) \rightarrow Apply(e', e_2)} \\
	 \frac{e2 \rightarrow e'}{Apply(v,e2) \rightarrow Apply(v, e')}
\end{gather}
A questi corrisponde la \textit{$\beta$-riduzione} con strategia call-by-value vista nel lambda calcolo.
\begin{gather}
	\frac{e_1 \rightarrow e_4}{if \: e_1 \: then \: e_2 \: else \: e_3 \rightarrow if \: e_4 \: then \: e_2 \: else \: e_3} \\
	if \: true \: then \: e_2 \: else \: e_3 \rightarrow e_2 \\
	if \: false \: then \: e_2 \: else \: e_3 \rightarrow e_3
\end{gather}
Queste regole invece sono per le espressioni condizionali.
\subsubsection{Ambiente dei tipi}
L'ambiente dei tipi ci serve per tenere conto delle associazioni variabile-tipo in modo da poter eseguire correttamente le regole di semantica. \\
L'ambiente è una \textbf{funzione} di dominio finito e la indichiamo come:
\begin{equation*}
	\Gamma = x_1:\tau_1,x_2:\tau_2 \ldots x_k:\tau_k
\end{equation*}
Per ottenere il tipo $\tau_i$ dal valore $x_i$ si usa la notazione
\begin{equation*}
	\Gamma(i)=\tau_i
\end{equation*}
Per \textbf{estendere} l'ambiente aggiungendo associazioni  si usa la notazione
\begin{equation*}
	\Gamma,x:\tau
\end{equation*}
Per applicare l'ambiente nelle regole di semantica si usa la notazione
\begin{equation*}
	\Gamma \vdash e:\tau
\end{equation*}
Le regole sono applicate dal compilatore in fase di \textit{analisi statica}. L'ambiente dei tipi è chiamato \textbf{symbol table}.
\newpage
\subsubsection{Type checker}
\begin{equation}
	\Gamma \vdash true:Bool \qquad \Gamma \vdash false:Bool \qquad \frac{\Gamma(x) = \tau}{\Gamma \vdash x : \tau}
\end{equation}
\begin{equation}
	\frac{\Gamma \vdash e:Bool \qquad \Gamma \vdash e_1 : \tau \qquad \Gamma \vdash e_2 : \tau}{\Gamma \vdash if \: e \: then \: e_1 \: else \: e_2 : \tau}
\end{equation}
Poi abbiamo il tipo delle funzioni, ovvero $\tau_1 \rightarrow \tau_2$, ovvero prende in ingresso un argomento di tipo $\tau_1$ e restituisce un risultato di tipo $\tau_2$.
\begin{equation}
	\frac{\Gamma,x:\tau_1 \vdash e:\tau_2}{\Gamma \vdash fun \: x:\tau_1 = e:\tau_1 \rightarrow \tau_2}
\end{equation}
In pratica, estendiamo l'ambiente con il tipo dell'argomento in ingresso (che è noto), e poi troviamo il tipo dell'espressione $e$ che sarà anche il tipo del risultato della funzione. Questa estensione "annulla" tutte le dichiarazioni precedenti della stessa funzione per questo scope.\\
Infine per l'applicazione delle funzioni:
\begin{equation}
	\frac{\Gamma \vdash e_1 : \tau_1 \rightarrow \tau_2 \qquad \Gamma \vdash e_2: \tau_1}{\Gamma \vdash Apply(e_1,e_2):\tau_2}
\end{equation}
\subsubsection{Correttezza}
\begin{proof}[Dimostrazione del progresso]
	Utilizziamo l'induzione sulle derivazioni di tipo.\\
	I casi base sono identici alla dimostrazione \ref{proof_progresso}.\\
	Il caso delle variabili è banale.\\
	Il caso dell'astrazione di funzione è immediato dato che le funzioni stesse sono valori.\\
	Per il caso dell'applicazione di funzione
	\begin{equation*}
		e = Apply(e_1,e_2), \emptyset \vdash e_1:\tau_1 \rightarrow \tau_2, \emptyset \vdash \tau_1
	\end{equation*}
	Per \textit{ipotesi induttiva} abbiamo due casi:
	\begin{itemize}
		\item Le due espressioni possono fare un passo: applichiamo le regole di riduzione dell'applicazione e terminiamo
		\item Le due espressioni sono entrambi valori: dovremo avere $e_1$ nella forma del tipo $fun \: x:\tau_1 = e':\tau_1 \rightarrow \tau_2$ e possiamo applicare la regola \ref{eq_briduz} 
	\end{itemize}
\end{proof}
\begin{proof}[Dimostrazione della conservazione]
	Si dimostra per induzione strutturale sulle regole di tipo. Anche qui il caso difficile è quello per l'applicazione
	\begin{equation*}
		e = Apply(e_1,e_2)
	\end{equation*}
	Quindi vale che:
	\begin{gather*}
		\Gamma \vdash e : \tau \\
		\Gamma \vdash e_1 : \tau_1 \rightarrow \tau_2 \\
		\Gamma \vdash e_2  : \tau_2 \\
		\tau = \tau_2 \\
		e \rightarrow e'
	\end{gather*}
	Dobbiamo dimostrare che:
	\begin{equation*}
		\Gamma \vdash e':\tau_2
	\end{equation*}
	La seconda affermazione comporta che
	\begin{equation*}
		e_1 = fun \: x : \tau_1 = e_3 \qquad \Gamma,x:\tau_1 \vdash e_3:\tau_2
	\end{equation*}
	che però non è ciò che  ci serve: noi vogliamo ottenere
	\begin{equation*}
		e' = e_3 \{x:=e_2\}
	\end{equation*}
	Dobbiamo in qualche modo eseguire la sostituzione, e per questo usiamo un lemma ad hoc:
	\begin{lemma}[Substitution lemma]
		I tipi sono preservati dall'operazione di sostituzione.
		\begin{gather*}
			\Gamma,x:\tau_1 \vdash e:\tau \\
			\Gamma \vdash e_1:\tau_1 \\
			\Gamma \vdash e\{x=e_1\}:\tau
		\end{gather*}
		Per dimostrarlo si usa l'induzione sulla derivazione del primo punto.
	\end{lemma}
\end{proof}
\subsubsection{Estensioni}
Cambiamo la sintassi del linguaggio per implementare le \textit{operazioni numeriche} e le \textit{dichiarazioni locali}:
\begin{center}
	\begin{tabular}{|c|c|}
		\hline
		$e ::=$ & Espressioni\\
		\hline
		$x$ & Variabili\\
		$fun \: x:\tau = e$ & Funzioni\\
		$Apply(e,e)$ & Applicazioni \\
		$true$ & Costante true\\
		$false$ & Costante false\\
		$n$ & Costanti numeriche\\
		$e \oplus e$ & Operazioni binarie \\
		if $e$ then $e$ else $e$ & Condizionale\\
		$let \: x = e_1 \: in \: e_2 : \tau_2$ & Dichiarazioni locali \\
		\hline
	\end{tabular}
\end{center}
Aggiungiamo le seguenti regole di valutazione:
\begin{equation}
	\frac{e_1 \rightarrow e'}{let \: x = e_1 \: in \: e_2 : \tau_2 \rightarrow let \: x = e' \: in \: e_2:\tau_2}
\end{equation}
\begin{equation}
	let \: x = v \: in \: e_2:\tau_2 \rightarrow e_2 \{x=v\}:\tau_2
\end{equation}
E le seguenti regole di tipo:
\begin{equation}
	\Gamma \vdash n:Nat
\end{equation}
\begin{equation}
	\Gamma \vdash e_1: \tau_1 \qquad \Gamma \vdash e_2:\tau_2 \qquad \frac{\oplus:\tau_1 \times \tau_2 \rightarrow \tau}{\Gamma \vdash e_1 \oplus e_2 : \tau}
\end{equation}
\begin{equation}
	\frac{\Gamma \vdash e_1 : \tau_1 \qquad \Gamma,x:\tau_1 \vdash e_2:\tau_2}{let \: x = e_1 \: in \: e_2 : \tau_2}
\end{equation}
\subsubsection{Ricorsione}
Nel lambda calcolo tipato non possiamo utilizzare il combinatore $\Omega$.\\ Dobbiamo quindi introdurre un generatore \textbf{aux} che, applicato ad una funzione \textit{iE} approssima il comportamento di una ipotetica funzione (ad esempio \textit{isEven}) fino a $n$. $iE \: k$ con $k \leq n$  restituisce il valore calcolato da $isEven \: k$. Allora $aux \: iE \: k$ restituisce  una migliore approssimazione di \textit{isEven} fino a $k+2$.
\begin{lstlisting}
	let aux = fun f:Nat -> Bool =
		fun x:Nat=
			if (isZero x) then true else
			if (isZero(pred x)) then false else
				f(pred(pred x))
	aux: (Nat -> Bool) -> Nat -> Bool
\end{lstlisting}
Aggiungiamo al linguaggio un costrutto \textit{fix} tale che $isEven = fix \: aux$. Applicando \textit{fix} ad \textit{aux} si ottiene il limite delle approssimazioni, ovvero il \textbf{punto fisso}.\\
Estendiamo la sintassi aggiungendo $fix \: e$, le seguenti regole di valutazione
\begin{gather}
	\frac{e \rightarrow e'}{fix \: e \rightarrow fix \: e'}\\
	fix(fun \: f:\tau = e) \rightarrow e [f = (fix (fun \: f:\tau = e))]
\end{gather}
e le regole di tipo
\begin{equation}
	\frac{\Gamma \vdash e:\tau \rightarrow \tau}{\Gamma \vdash fix \: e: \tau}
\end{equation}
La versione semplificata della sintassi, che poi viene usata anche in Ocaml, è la seguente:
\begin{lstlisting}
	let rec x: <Type> = e in e'
\end{lstlisting}
che corrisponde a
\begin{lstlisting}
	let x = fix (fun x: <Type> = e) in e'
\end{lstlisting}

\begin{example}[Esempio ricorsione semplificata]
	\begin{lstlisting}
		let rec isEven: Nat -> Bool =
			fun x:Nat =
				if(isZero x) then true else
				if (isZero (pred x)) then false
				else isEven(pred (pred x))
		in isEven 7;
	\end{lstlisting}
\end{example}