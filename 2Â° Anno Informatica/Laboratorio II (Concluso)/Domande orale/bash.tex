% !TeX spellcheck = it_IT
\newpage
\section{Bash}

\subsection{kill}
\textbf{Invio di segnale dalla riga di comando con \textit{kill}.}\\
\begin{lstlisting}[language=BASH]
	kill -signal pid
\end{lstlisting}
Dove i parametri sono:
\begin{itemize}
	\item \textit{signal} nome o numero del segnale da inviare al processo
	\item \textit{pid} process ID a cui inviare il segnale:
	\begin{itemize}
		\item $>0$ processo con PID specifico
		\item $=0$ tutti processi nel gruppo del chiamante
		\item $<-1$ tutti i processi nel gruppo con PID$=-$PID
		\item $=-1$ tutti i processi a cui l'utente può inviare segnali
	\end{itemize}
\end{itemize}
\begin{note}
	Utilizzare il parametro \textit{-l} per vedere l'elenco dei segnali e i loro numeri.
\end{note}
Esiste anche una funzione che può essere chiamata all'interno di un programma e invia il segnale ad un singolo thread all'interno dello stesso processo.
\begin{lstlisting}[language=C]
	int pthread_kill(pthread_t thread, int sig);
\end{lstlisting}

\subsection{MAKEFILE}
\textbf{Parla del MAKEFILE e della sua compilazione.}\\
Il MAKEFILE è uno strumento per gestire la compilazione di progetti, specialmente quelli composti da più file sorgente.
\begin{lstlisting}[language=BASH]
	target: dipendenze
	comando
\end{lstlisting}
Si compone di:
\begin{itemize}
	\item \textit{target}: nome del file da creare
	\item \textit{dipendenze}: file da cui il target dipende
	\item \textit{comando}: comando da eseguire
\end{itemize}
\begin{example}[MAKEFILE]
	Dato \textit{main.c} il file sorgente, \textit{funzioni.c} il sorgente con funzioni aggiuntive e \textit{funzioni.h} il suo header.
	\begin{lstlisting}[language=BASH]
		# Compilatore
		CC=gcc
		# Opzioni di compilazione
		CFLAGS= -wall -g
		# Eseguibile
		TARGET = programma
		# File sorgenti
		SRCS = main.c funzioni.c
		# File oggetto
		OBJS = main.o funzioni.o
		# Regola principale
		all: $(TARGET)
		# Regola per creare eseguibile
		$(TARGET): $(OBJS)
		# $< significa primo file nella lista dipendenze
		# $@ significa nome del target .o
		$(CC) $(CFLAGS) -c $< -o $@
		# Pulizia
		clean:
		rm -f $(OBJS) $(TARGET)
	\end{lstlisting}
\end{example}

\subsection{Processi in background}
In Linux è possibile avviare un processo in background aggiungendo il simbolo \textit{\&} alla fine del comando.
\begin{lstlisting}[language=BASH]
	./programma &
\end{lstlisting}
Il terminale visualizza il PID del processo avviato.

\subsection{Elenco dei file}
Per elencare i file:
\begin{lstlisting}[language=BASH]
	ls -l # Dettagliato
	ls -a # Mostra anche quelli nascosti
	ls -R # Elenca ricorsivamente il contenuto delle sottodirectory
	ls -lh # Specifica anche la dimensione dei file
\end{lstlisting}
L'output di \textit{ls -l} ci specifica anche se un determinato oggetto è un file o una directory in quanto l'ultima avrà in aggiunta la lettera \textit{d} mentre il primo avrà \textit{-}.

\subsection{Manuale}
Tramite il comando
\begin{lstlisting}[language=BASH]
	man comando
\end{lstlisting}
è possibile leggere la pagina del manuale relativa ad uno specifico comando. Il risultato conterrà anche \textit{comando(n)} dove il numero tra parentesi rappresenterà il numero della pagina del manuale.