% !TeX spellcheck = it_IT
\newpage
\section{Socket}
Permettono la comunicazione tra due processi su macchine diverse. Funzionano in maniera analoga alle \textit{pipe}, quindi deve essere chiaro il \textbf{formato} con il quale vengono inviati i dati.\\
La complicazione principale delle socket è che il mezzo di trasmissione (\textbf{internet}) non è sempre disponibile e la connessione potrebbe cadere. Inoltre, a differenza delle pipe, il server interagisce con più client.

\subsection{Implementazione}
Nel corso verrà utilizzato Python perché più veloce da scrivere e comunque molto simile ai comandi in C. La differenza principale tra i due linguaggi sta nel server, che però a livello concettuale non è complesso (solo lungo).
\subsubsection{Server}
Una possibile implementazione di un \textbf{echo-server} in Python è la seguente:
\begin{lstlisting}[language=Python]
	import socket
	
	# Specifica da dove accettare le connessioni
	HOST = "127.0.0.1"  # Standard loopback interface address (localhost)
	PORT = 65432  # Port to listen on (non-privileged ports are > 1023)
	
	# Creazione del server socket
	with socket.socket(socket.AF_INET, socket.SOCK_STREAM) as s:
		s.bind((HOST, PORT)) # Accetta connessioni da HOST in riferimento alla porta PORT
		s.listen() # In ascolto
		while True:
			# In attesa di una connessione
			conn, addr = s.accept()
			# Lavoro con la connessione appena ricevuta 
			with conn:
				while True:
					data = conn.recv(64) # Leggo fino a 64 bytes
					# Se ricevo 0 bytes, termino la connessione
					if not data:
						break
					conn.sendall(data) # Altrimenti rispedisco i dati indietro
\end{lstlisting}

\subsubsection{Client}
Una possibile implementazione di un \textbf{echo-client} in Python è la seguente:
\begin{lstlisting}[language=Python]
	import sys,socket
	
	# Valori verso cui connettersi 
	HOST = "127.0.0.1" # Server hostname or IP address
	PORT = 65432 # Port used by the server

	# Creazione del socket per la connesssione al server
	with socket.socket(socket.AF_INET, socket.SOCK_STREAM) as s:
		# Mi connetto
		s.connect((host, port))
		while True:
			n = int(input("Quanti byte? "))
			if n<=0:
				break
			# Invio stringa di n a e attendo la risposta   
			msg = "a"*n
			s.sendall(msg.encode())
			data = s.recv(64)
			if len(data)==0:
				break
		# Comunico al server di chiudere la connessione
		# SHUT_RDWR indica che terminiamo sia lettura che scrittura     
		s.shutdown(socket.SHUT_RDWR)
\end{lstlisting}

\begin{observation}[Client multipli]
	Cosa succede se un altro client prova a connettersi mentre c'è già una connessione attiva?\\
	Il server stabilisce la connessione ma non risponde ad entrambi, solo ad uno, l'altro rimarrà in attesa.
\end{observation}
\begin{observation}[Violazione del protocollo]
	Cosa succede se ignoro le specifiche che mi sono dato sul formato dei dati?\\
	Si perde la sincronizzazione. Per garantirla si può ad esempio precedere i dati con la lunghezza del pacchetto.
\end{observation}
\begin{observation}[Codifica]
	In Python possiamo codificare una stringa tramite la seguente istruzione:
	\begin{lstlisting}[language=Python]
		s = "hello world"
		s.encode()
	\end{lstlisting}
	Di default utilizza la codifica \textit{UTF-8} che per i caratteri standard usa un byte mentre per quelli più complessi (e.g. €) ne usa di più. \\
	Per quanto riguarda i numeri posso scegliere il tipo di codifica tra \textit{big-endian} (!) o \textit{little-endian}:
	\begin{lstlisting}[language=Python]
		encoded = struct.pack("i", 33) # Little endian
		struct.pack("!i", 33) # Big endian
	\end{lstlisting}
	E poi per decodificarlo:
	\begin{lstlisting}[language=Python]
		struct.unpack("i", encoded)
	\end{lstlisting}
\end{observation}
\subsection{Client multipli}
Per gestire più client contemporaneamente, esistono due tipi di implementazione:
\begin{itemize}
	\item Il singolo server alterna la connessione tra i vari client
	\item Quando arriva un client, crea un \textbf{thread} dedicato a quella connessione
\end{itemize}