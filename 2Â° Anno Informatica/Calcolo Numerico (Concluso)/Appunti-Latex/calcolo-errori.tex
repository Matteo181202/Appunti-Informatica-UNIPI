% !TeX spellcheck = it_IT
\newpage
\section{Calcolo degli errori}
Supponiamo di avere una funzione \(f: [a,b] \to \mathbb{R}\) e \(f \neq 0\), per andare a calcolare questa funzione come già visto usiamo un algoritmo
che esprime tale valore come risutlato di una sequanza di operazioni aritmentiche. Questa rappresentazione come abbiamo già potuto
verificare con esempi produce degli errori di approssimazione. Questi errori possono essere suddivisini in 3 tipologie.

\begin{definition}[Errore inerente o inevitabile]
    Si dice errore inerente o inevitabile generato nel calcolo di \(f(x) \neq 0\) la quantità:
    \[\epsilon_{in} = \frac{f(\tilde{x}) - f(x)}{f(x)}\]
\end{definition}

\begin{definition}[Errore algoritmico]
    Si dice errore algoritmico generato nel calcolo di \(f(x) \neq 0\) la quantità:
    \[\epsilon_{alg} = \frac{g(\tilde{x}) - f(\tilde{x})}{f(\tilde{x})}\]
\end{definition}

\begin{definition}[Errore totale]
    Si dice errore algoritmico totale nel calcolo di \(f(x) \neq 0\) mediante l'algoritmo specificato da g la quantità:
    \[\epsilon_{tot} = \frac{g(\tilde{x}) - f(x)}{f(x)}\]
\end{definition}

\begin{observation}
    Vediamo che se \(|\epsilon_{in}|\) è grande il problema si definisce \textbf{problema mal condizionato}. Mentre se
    \(|\epsilon_{alg}\) è grande l'algoritmo si dice che \textbf{algoritmo è numericamente instabile}.
\end{observation}