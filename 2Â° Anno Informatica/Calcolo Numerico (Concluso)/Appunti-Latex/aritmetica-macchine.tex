% !TeX spellcheck = it_IT
\section{Aritmetica di Macchina}
Per una macchina la scrittura $(x + y) + z$ è diverso da $x + (y + z)$. Vediamo dunque che ci sono alcuni punti focali da considerare per far si che una macchina funzioni correttamente:

\begin{itemize}
    \item Trovare uno standard per come \textbf{memorizzare} i numeri.
    \item Trovare uno standard per come \textbf{manipolare} i numeri.
\end{itemize}

\noindent Da questi due punti possiamo ricondurci ad un solo problema, come andare a \textbf{rappresentare} i numeri.

\subsection{Teorema di rappresentazione}
\begin{theorem}
    Dato $x \in \mathbb{R}, x \neq 0$\footnote{Lo zero viene utilizzato dalla macchina per alcune operazioni come il confronto, quindi deve averlo ma lo rappresenta in un modo particolare} e una base di numerazione $B, B\in \mathbb{N}, B>1$ esistono e sono univocamente determinati:
    \begin{enumerate}
        \item Un intero $p \in \mathbb{Z}$ detto \textbf{esponente} della rappresentazione
        \item Una successione di numeri naturali $\{d_i\}_{i=1}^{+\infty}$ con $d_i \neq 0, 0 \leq d_i \leq B - 1$ e $d_i$
        non definitivamente uguali a \(B - 1\), dette cifre della rappresentazione tali per cui $x$ si scrive in modo \textbf{unico} nella seguente forma:
        \begin{equation}
        	x = sign(x)B^p \sum_{i=1}^{+\infty}d_i B^{-1}.
        \end{equation}
        dove la sommatoria viene chiamata \textbf{mantissa}
    \end{enumerate}   
\end{theorem}

\begin{example}[Esempio in base 10]
	Poniamo come numero da rappresentare $0.1$ in base $10$ è
	\begin{equation*}
		0.1 = +10^0(0.1)
	\end{equation*}
\end{example}

\noindent Andiamo ora ad analizzare il significato di questo teorema. Esso descrive quella che viene chiamata \textbf{rappresentazione
in virgola mobile}, in quanto l'esponente \(p\) on è determinato in modo da avere la parte intera nulla. Le cose da considerare
in questo teorema sono:
\begin{itemize}
    \item La condizione \(d_i \neq 0\) e \(d_i\) non definitivamente uguale a \(B - 1\) sono introdotte per garantire
    l'unicità delle rappresentazioni. Ad esempio:
    \[B = 10 \text{ abbiamo } 1 = +10^1 (1 \cdot 10^{-1}) = +10^2 (0 \cdot 10^{-1} + 1 \cdot 10^{-1})\]
    Quindi due rappresentazioni diverse per lo stesso numero, però considerando le condizioni scritte sopra la seconda non risulta
    accettabile perché la prima cifra è nulla.\\
    Questa clausola ci garantisce anche l'unicità delle rappresentazioni nei numeri \textbf{periodici}:
    \begin{equation*}
    	0.\overline{9} = 10^0 (0.99\ldots9)
    \end{equation*}
    Non è ammissibile in quanto è definitivamente uguale a $B-1$.
    \item Questa rappresentazione si estende anche all'insieme dei numeri complessi del tipo \(z = a + ib\), utilizzando una rappresentazione
    come coppie di numeri reali del tipo \((a,b)\).
\end{itemize}

Possiamo dedurre che visto che stiamo lavorano con registri di memoria di un calcolatore con memoria a numero finito, anche la quantità
di cifre rappresentabili saranno a numero finito esso viene chiamato \textbf{insieme dei numeri di macchina}.\\

Dal teorema di rappresentazione in base di un numero reale può avvenire assegnando delle posizioni di memoria per il segno, 
per l'esponente e per le cifre della rappresentazione.

\begin{definition}[Insieme dei numeri di macchina]
    Si definisce l'insieme dei numeri di macchina in rappresentazione floating point con t cifre, base B e range \(-m, M\) l'insieme dei numeri reali.
    \[\mathbb{F}(B, t, m, M) = \{0\} \cup \{s \in \mathbb{R} : x = \pm B^p \sum_{i=1}^{t} d_i B^{-1}, d_1 \neq 0, -m \leq p \leq M\}\]
\end{definition}

\noindent Si osserva in questa definizione che:
\begin{itemize}
    \item L'insieme \(\mathbb{F}\) ha cardinalità \textbf{finita}: \(N = 2B^{t-1}(B-1)(M + m + 1) + 1\).
    \item L'insieme dei numeri di macchina \(\mathbb{F}(B, t, m, M)\) è \textbf{simmetrico} rispetto all'origine.
    \item Possiamo definire \(\Omega = B^M \sum_{i=1}^{t} (B-1)b^{-1}\) come il \textbf{più grande} numero macchina e \(\omega = +B^{-m} B^{-1}\) come
    invece il \textbf{più piccolo} positivo.
    \item Posto un \(x = B^P \sum_{i=1}^{t}d_i B^{-1}\) possiamo definire il suo \textbf{successivo} numero di macchina come \(y = B^p(\sum_{i=1}^{t-1}d_i B^{-1} + (d_t + 1)B^{-t})\). \\
    Da qui vediamo che la distanza \(y - x = B^p - t\) porta
    i numeri ad essere \textbf{non equidistanti} fra di loro, quindi la distanza aumenta con l'avvicinarci a \(\Omega\).\\
    Questo ci fa comodo perché ci interessa l'\textbf{errore relativo}, quindi su numeri piccoli ci serve un errore piccolo mentre su numeri grandi posso fare errori grandi.
\end{itemize}

\begin{example}
    Facciamo ora un esempio in cui andiamo a rappresentare il numero successivo di \(x = B^p \sum_{t}^{i=1} d_i B^{-1}\). Esso
    si può scrivere come \(y = B^p \bigg( \sum_{i=1}^{t-1} d_i B^{-1} + (d_t + 1)B^{-t} \bigg)\).\\
    Mentre si può scrivere la distanza fra questi due valori come \(y - x = B^p - t\).
\end{example}

\noindent E' stato fissato uno standard IEEE 754 negli anni 70/80 che dice che, visto ci sono macchine che hanno metodi di rappresentazione
diversi, bisogna fissare un standard, ovvero \(B = 2\) con registri a 32 o 64 bit.\\\\
Questa rappresentazione ha uno svantaggio che può sembrare minimo ma non lo è, lo 0 si rappresenta due volte con \(-0, +0\). Per ovviare
a questo problema si è andato ad abbandonare questa rappresentazione in esponenti ma si rappresentato i numeri nel seguente modo:
\(p_1 2^0 + p_1 2^1 + \dots + p_11 s^10 \) che rappresentano numeri da 0 a \(2^11 - 1\) quindi 2047 numeri, mentre lo 0 si può scrivere come:
\begin{itemize}
    \item O tenendo tutti i valori a 0
    \item Oppure tendendo tutti i valori a 1
\end{itemize}

\hspace{-15pt}In entrambi i casi abbiamo un range di valori che va da \([-1022, 1024]\). A questo punto ho \(2^{P-1022}\) numeri che la 
macchina rappresenta come \(\pm 2^{P - 1022}(0.1d_1 \dots d_{52})\).\\
Impostando questo standard abbiamo \(\Omega = 2^{1024}(01 \dots 1)_2\) e \(\omega = 2^{-1022} (101)_2\).\\

\begin{observation}
    Quando \(p=0\) abbiamo i numeri che si trovano nella porzione della retta dei numeri che è compresa fra \(-\omega\) \(\omega\)
    e possiamo qui avere anche tutti 0 e quindi si introduce il caso dei numeri denormalizzati.
\end{observation}

\hspace{-15pt}Se abbiamo l'esponente uguale a tutti 1, la convenzione è che tutte le cifre della mantissa sono tutti uguali a 0/1
questo numero indica il \(\pm \infty\) altrimenti sta a significare NaN (not a number). Questi valori ci permettono di gestire
forme indeterminate.

\subsection{Errore di rappresentazione}
Quando si va a rappresentare un numero reale non nullo \(x \in \mathbb{R}\) e con \(x \neq 0\) si può andare a commettere degli 
errori di rappresentazione detto anche \textbf{errore relativo di approssimazione}, e di definisce come, prendendo un \(\tilde{x} \in \mathbb{F}(B, t, m, M)\)
\[\epsilon_x = \frac{\tilde{x} - x}{x} = \frac{\eta x}{x}, x \neq 0\]

\hspace{-15pt}Definiamo \(|\epsilon_x| = \big | \frac{\tilde{x} - x}{x}\big | \leq \frac{B^{P-t}}{| x|} \leq \frac{B^{P - t}}{B^{P - 1}} = B^{1 - t} = u\)
la \(u\) è definita come \textbf{precisione di macchina}.\\

\hspace{-15pt}Andiamo inoltre a definire le condizioni di underflow e overflow. Dato un \(x \in \mathbb{R}, x \neq 0\) abbiamo che:
\begin{enumerate}
    \item Se \(|x| < \omega\) o \(|x| > \Omega\) overflow. In questo caso si va ad associare il \(+\infty\).
    \item Se invece \(\omega \leq |x| \leq \Omega\) abbiamo underflow. In questo caso allora prendiamo una \(x = B^p \sum_{i = 1}^{\infty}d_i B^{-1}
    \longrightarrow B^p \sum_{i=1}^{t}d_i B^{-1} = \tilde{x}\) che è una approssimazione 
\end{enumerate}

\subsection{Operazioni di macchina}
Consideriamo ora due numeri \(x, y \in \mathbb{F}\) e chiediamoci perché la macchina non possa fare l'operazione \(x + y\). La risposta
è che i risultati da questa operazione di ritornano fra i numeri di macchina. Per ovviare a questo problema dovremo usare le Operazioni 
di macchina che si identificano come \(\oplus \ominus \otimes \oslash\).\\
Nel nostro caso l'addizione di macchina \(x \oplus y = troncamento(x + y) = (x+ y)(1 + \epsilon_1)\) con \(|\epsilon_1| \leq u\) con \(e_1\)
detto errore locale dell'operazione.

\begin{example}
    Supponiamo di dover calcolare in macchina la funzione \(f(x) = \frac{x - 1}{x}\). In macchina questa funzione corrisponderebbe a
    \(g(\tilde{x}) = (\tilde{x} \ominus 1) \oslash \tilde{x} \). Abbiamo quindi:
    \[g(\tilde{x}) = \frac{(x (1 + \epsilon_x) - 1)(1 + \epsilon_1)}{x(1 + \epsilon_x)} \cdot (1 + \epsilon_1) = 
    \frac{(x (1 + \epsilon_x) - 1)(1 + \epsilon_1 + \epsilon_2)}{x(1 + \epsilon_x)} =
    \frac{(x (1 + \epsilon_x) - 1)(1 + \epsilon_1 + \epsilon_2 - \epsilon_x)}{x}\]
    \[g(\tilde{x}) = (\tilde{x} \oplus 1) \oslash \tilde{x} = \frac{(x (1 + \epsilon_x) - 1)(1 + \epsilon_1 + \epsilon_2 - \epsilon_x)}{x}\]
    \[\frac{g(\tilde{x} - f(x))}{f(x)} = \frac{((x-1)/x) + (\epsilon_1 + \epsilon_2)((x-1)/x) + \epsilon_2 / x - ((x-1)/ x)}{(x-1)/x} = \epsilon_1 + \epsilon_2 - \frac{\epsilon_x}{x - 1}\]
\end{example}

\begin{example}
    Supponiamo ora di calcolare la funzione \(f(x) = \frac{x - 1}{x}\) in un altro modo, \(g_2(\tilde{x}) = \frac{g_2(\tilde{x}) - f(x)}{f(x)}\) ed 
    andiamo a fare l'analisi dell'errore.
    \[\frac{g_1(\tilde{x}) - f(x)}{f(x)} \doteq \epsilon_1 + \epsilon_2 + \frac{\epsilon_1}{(x-1)} \text{ Questo è il risultato di un analisi al primo ordine}\]
    \[g_2(\tilde{x}) = 1 \ominus \frac{1}{\tilde{x}}(1 + \delta_1) = 1 \ominus \frac{1}{x}(1 + \delta_1)(1 - \epsilon_x) = [1 -\frac{1}{x} (1 - \delta_1)(1-\epsilon_1)](1 + \delta_2)\]
    \[\doteq (1 + \delta_1) - \frac{1}{x}(1 + \delta_1 + \delta_2 + \epsilon_x) \doteq (1 - \frac{1}{x}) + \delta_2(1 - \frac{1}{x}) - \frac{\delta_1}{x} + \frac{\epsilon_x}{x}
    \doteq \delta_2 - \frac{\delta_1}{x - 1} + \frac{\epsilon_x}{x-1}\]
    \[\frac{g_2(\tilde{x}) - f(x)}{f(x)} = \delta_2 - \frac{\delta_1}{x - 1} + \frac{\epsilon_x}{x-1}\]
    Questo è il risultato finale dove \(\delta_2 - \frac{\delta_1}{x - 1}\) viene definita come parte stabilità mentre \(\delta_2 - \frac{\delta_1}{x - 1}\) viene
    chiamato condizionamento,il risultato finale viene definito invece numero stabile.
\end{example}