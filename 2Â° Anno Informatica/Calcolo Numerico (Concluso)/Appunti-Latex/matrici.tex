% !TeX spellcheck = it_IT
\newpage
\section{Matrici}
%TODO recuper

\subsection{Norme vettoriali}
Sono uno strumento che ci permette di definire una distanza tra due vettori.
\begin{definition}[Norma vettoriale]
	È una funzione del tipo $f:\mathbb{R}^n \to \mathbb{R}$ che deve soddisfare tre proprietà:
	\begin{enumerate}
		\item $f(x)\geq 0 \wedge f(x)=0 \Leftrightarrow x=0$
		\item $f(\alpha x)=\lvert \alpha \rvert f(x) \quad \forall \alpha \in \mathbb{R} \quad \forall x \in \mathbb{R}^n$
		\item \textbf{Disuguaglianza triangolare}: $f(x+y)\leq f(x)+f(y) \quad \forall x,y \in \mathbb{R}^n$
	\end{enumerate}
	e la indichiamo come
	\begin{equation*}
		f(x) = \lVert x \rVert
	\end{equation*}
\end{definition}

Detto questo possiamo definire una distanza come:
\begin{equation}
	dist(x,y)=\lVert  x-y \rVert
\end{equation}
Le tre proprietà ci danno alcune informazioni sulla distanza:
\begin{enumerate}
	\item La distanza deve essere non negativa e valere $0$ solo se i due vettori coincidono 
	\item La distanza tra $x$ e $y$ deve essere uguale a quella tra $y$ e $x$
	\item $\lVert x-y \rVert = \lVert(x-a)+(a-y)\rVert \leq \lVert x-a \rVert + \lVert a-y \rVert$
\end{enumerate}

\begin{definition}[Distanza euclidea - Norma 2]
	\begin{equation}
		f(x) = \sqrt{x_1 ^ 2 + x_2 ^ 2 + \ldots + x_n ^ 2} > \lVert  x \rVert_2
	\end{equation}
\end{definition}
\begin{definition}[Norma infinito]
	\begin{equation}
		f(x)=max\lvert x_1 \rvert = \lVert x \rVert_{\infty}
	\end{equation}
\end{definition}
\begin{definition}[Norma 1]
	\begin{equation}
		f(x)=\sum_{i=1}^{n}\lvert x_i \rvert = \lVert x\rVert_1
	\end{equation}
\end{definition}

\begin{example}
	Prendiamo due vettori
	\begin{equation*}
		\begin{Bmatrix}
			1 \\
			1
		\end{Bmatrix} \quad
		\begin{Bmatrix}
			2 \\
			-2
		\end{Bmatrix}
	\end{equation*}
	e calcoliamo le varie norme:
	\begin{equation*}
		\lVert x \rVert_2 = \sqrt{2} \quad \lVert x \rVert_\infty = 1 \quad \lVert x \rVert_1 = 2
	\end{equation*}
	%TODO Finisci
\end{example}

\begin{definition}[Norma matriciale]
		È una funzione del tipo $f:\mathbb{R}^{n\times n} \to \mathbb{R}$ che deve soddisfare tre proprietà:
	\begin{enumerate}
		\item $f(A)\geq 0 \wedge f(A)=0 \Leftrightarrow A=0$
		\item $f(\alpha A)=\lvert \alpha \rvert f(x) \quad \forall \alpha \in \mathbb{R} \quad \forall A \in \mathbb{R}^{n\times n}$
		\item \textbf{Disuguaglianza triangolare}: $f(A+B)\leq f(A)+f(B) \quad \forall A,B \in \mathbb{R}^{n\times n}$
		\item $f(A\cdot B) \leq f(A) f(B)$
	\end{enumerate}
	e la indichiamo come
	\begin{equation*}
		f(A) = \lVert A \rVert
	\end{equation*}
\end{definition}