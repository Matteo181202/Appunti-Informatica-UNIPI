% !TeX spellcheck = it_IT
\section{Assembler}
\subsection{Numeri frazionari}
Supponiamo di avere il numero $2.375_{10}$. Per convertirlo in base 2, prima convertiamo la parte intera ($2=10$) e poi la parte frazionaria:
\begin{equation*}
	\begin{split}
		2.375_{10} = 10.011 \\
		0.375 \cdot 2 = 0.75 \\
		0.75 \cdot 2 = 1.5 \\
		0.5 \cdot 2 = 1 \\
	\end{split}
\end{equation*}
Per fare la conversione inversa:
\begin{equation*}
	10.011 = 1*2^1 + 1 * 2^{-2} + 1 * 2^{-3}=2+\frac{1}{4} + \frac{1}{8} = 2.375
\end{equation*}
\subsubsection{Virgola fissa}
\subsubsection{Virgola mobile}
Lo standard IEEE 754 divide i numeri in due categorie:
\begin{itemize}
	\item \textbf{Singola precisione}
	\item \textbf{Doppia precisione}
\end{itemize}

\subsubsection{Estensioni Floating-Point}
