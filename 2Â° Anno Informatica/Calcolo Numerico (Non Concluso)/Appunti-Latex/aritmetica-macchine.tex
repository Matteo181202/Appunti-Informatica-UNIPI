\section{Aritmetica di Macchina}
Per una macchina la scrittura $(x + y) + z \neq x + (y + z)$. Vediamo dunque che ci sono alcuni punti focali da considerare
per far si che una macchina funzioni correttamente:

\begin{itemize}
    \item Trovare uno standard per come memorizzare i numeri.
    \item Trovare uno standard per come manipolare i numeri.
\end{itemize}

\hspace{-15pt}Da questi due punti possiamo ricondurci ad un solo problema, come andare a rappresentare i numeri.

\subsection{Teorema di rapresentazione}
\begin{theorem}
    Dato $x \in \mathbb{R}, x \neq 0$ esistono e sono univocamente determinati.
    \begin{enumerate}
        \item un intero $p \in \mathbb{Z}$ detto esponente della rappresentazione.
        \item una successione di numeri naturiali $\{d_i\}_{i\geq 1}$ con $d_i \neq 0, 0 \leq d_i \leq B - 1$ e $d_i$
        non definitvamente uguali a \(B - 1\), dette cire della rappresentazione; tali per cui si ha
    \end{enumerate}
    \begin{equation}
        x = sign(x)B^p \sum_{i=1}^{+\infty}d_i B^{-1}.
    \end{equation}
\end{theorem}

\hspace{-15pt}Andiamo ora ad analizzare il significato di questo teorema. Esso descrive quella che viene chiamata rappresentazione
in virgola mobile, in quanto l'esponente \(p\) on è determinato in modo da avere la parte intera nulla. Le cose da considerare
in questo teorema sono:
\begin{itemize}
    \item La condizione \(d_i \neq 0\) e \(d_i\) non definitivamente uguale a \(B - 1\) sono introdotte per garantire
    l'unicità delle rappresentazioni. Ad esempio:
    \[B = 10 \text{ abbiamo } 1 = +10^1 (1 \cdot 10^{-1}) = +10^2 (0 \cdot 10^{-1} + 1 \cdot 10^{-1})\]
    Quindi due rappresentazioni diverse per lo stesso numero, però considerando le condizioni scritte sopra la seconda non risulta
    accettabile perché la prima cifra è nulla.
    \item Il caso \(x=0\) non ammette rapresentazione normalizzata. Questa casistica viene trattata dalla macchina in un modo
    particolare, per questo abbiamo la condizione \(x\neq \).
    \item Questa rapresentazione si estende anche all'insieme dei numeri complessi del tipo \(z = a + ib\), utilizzando una rapresentazione
    come coppie di numeri reali del tipo \((a,b)\).
\end{itemize}

Possiamo dedurre che visto che stiamo lavornano con registri di meoria di un calcolatore con memoria a numero finito, anche la quantità
di cifre rapresentabili saranno a numero finito esso vinene chiamato \textbf{insieme dei numeri di macchina}.\\

Dal teorema di rapresentazione in base di un numero reale può avvenire assegnando delle posizioni di meoria per il segno, 
per l'esponente e per le cifre della rappresentazione.

\begin{definition}[Insieme die numeri di macchina]
    Si definisce l'insieme dei nuermi di macchina in rappresentazione floating point con t cifre, base B e range \(-m, M\) l'insieme dei numeri reali.
    \[\mathbb{F}(B, t, m, M) = \{0\} \cup \{s \in \mathbb{R} \::\: x = sign(x) B^p \sum_{i=1}^{t} d_i B^{-1}, 0\leq d_i \leq B - 1, d_1 \neq 0, -m \leq p \leq M\}\]
\end{definition}

\hspace{-15pt}Si osserva in questa definizione che:
\begin{itemize}
    \item L'insieme \(\mathbb{F}\) ha cardinalità finita \(N = 2B^{t-1}(B-1)(M + m + 1) + 1\).
    \item L'insieme dei numeri di macchina \(\mathbb{F}(B, t, m, M)\) è simmetrico rispetto all'origine.
    \item Possiamo definire \(\Omega = B^M (B-1) \sum_{i=1}^{t} b^{-1}\) come il più grande numero macchina e \(\omega = B^{-m} B^{-1}\) come
    invece il più piccolo.
    \item Posto un \(x = B^P \sum_{i=1}^{t}d_i B^{-1}\) possiamo definire il suo successiovo numero di macchina come
    \(y = B^p(\sum_{i=1}^{t-1}d_i B^{-1} + (d_t + 1)B^{-t})\). Da qui vediamo che la distanza \(y - x = B^p - t\) porta
    i numeri ad essere non equispaziali fra di loro, quindi la distanza aumento con l'avicinarci a \(\Omega\)
\end{itemize}

\begin{example}
    Facciamo ora un esempio in cui andiamo a rappresentare il numero successivo di \(x = B^p \sum_{t}^{i=1} d_i B^{-1}\). Esso
    si può scrivere come \(y = B^p \bigg( \sum_{i=1}^{t-1} d_i B^{-1} + (d_t + 1)B^{-t} \bigg)\).\\
    Mentre si può scrivere la distanza fra questi due valori come \(y - x = B^p - t\).
\end{example}

\hspace{-15pt}E' stato fissato uno standard IEEE 754 fra gli anni 70/80, questo standard dice che, visto ci sono macchine che hanno metodi di rappresentazione
diversi bisogna fissare un standard, esso appunto dice che \(B = 2\) ed i registri sono a 32 o 64 bit.\\\\
Questa rapresentazione ha uno svantaggio che può sembrare minimo ma non lo è, lo 0 si rappresenta due volte con \(-0, +0\). Per ovviare
a questo problema si è andato ad abbandonare questa rappresentazione in esponenti ma si rapprensentato i numeri nel seguente modo:
\(p_1 2^0 + p_1 2^1 + \dots + p_11 s^10 \) che rappresentano numeri da 0 a \(2^11 - 1\) quindi 2047 numeri, mentre lo 0 si può scrivere come:
\begin{itemize}
    \item O tenendo tutti i valori a 0
    \item Oppure tendendo tutti i valori a 1
\end{itemize}

\hspace{-15pt}In entrambi i casi abbiamo un range di valori che va da \([-1022, 1024]\). A questo punto ho \(2^{P-1022}\) numeri che la 
macchina rappresneta come \(\pm 2^{P - 1022}(0.1d_1 \dots d_{52})\).\\
Impostando questo standard abbiamo \(\Omega = 2^{1024}(01 \dots 1)_2\) e \(\omega = 2^{-1022} (101)_2\).\\

\begin{observation}
    Quando \(p=0\) abbiamo i numeri che si trovano nella porzione della retta dei numeri che è compresa fra \(-\omega\) \(\omega\)
    e possiamo qui avere anche tuti 0 e quindi si introduce il caso dei numeri denormalizzati.
\end{observation}

\hspace{-15pt}Se abbiamo l'esponante uguale a tutti 1, la convensione è che tutte le cifre della mantissa sono tutti uguali a 0/1
questo numero indica il \(\pm \infty\) altrimenti sta a signifiare NaN (not a number). Questi valori ci permettono di gestire
forme indeterminate.

\subsection{Errore di rappresentazione}
Quando si va a rappresentare un numero reale non nullo \(x \in \mathbb{R}\) e con \(x \neq 0\) si può andare a commettere degli 
errori di rappresentazione detto anche \textbf{errore relativo di approssimazione}, e di definisce come, prendendo un \(\tilde{x} \in \mathbb{F}(B, t, m, M)\)
\[\epsilon_x = \frac{\tilde{x} - x}{x} = \frac{\eta x}{x}, x \neq 0\]

\hspace{-15pt}Definiamo \(|\epsilon_x| = \big | \frac{\tilde{x} - x}{x}\big | \leq \frac{B^{P-t}}{| x|} \leq \frac{B^{P - t}}{B^{P - 1}} = B^{1 - t} = u\)
la \(u\) è definita come \textbf{precisione di macchina}.\\

\hspace{-15pt}Andiamo inoltre a definire le conidizioni di underflow e overflow. Dato un \(x \in \mathbb{R}, x \neq 0\) abbiamo che:
\begin{enumerate}
    \item Se \(|x| < \omega\) o \(|x| > \Omega\) overflow. In questo caso si va ad associare il \(+\infty\).
    \item Se invece \(\omega \leq |x| \leq \Omega\) abbiamo underflow. In questo caso allora prendiamo una \(x = B^p \sum_{i = 1}^{\infty}d_i B^{-1}
    \longrightarrow B^p \sum_{i=1}^{t}d_i B^{-1} = \tilde{x}\) che è una approssimazione 
\end{enumerate}