% !TeX spellcheck = it_IT
\newpage
\section{Test statistici}
Un test statistico è una procedura per verificare ipotesi su uno o più parametri incogniti della distribuzione di probabilità con cui vogliamo descrivere un esperimento ripetuto di cui conosciamo gli esiti.

\subsection{Formulazione}
\subsubsection{Ipotesi}
\begin{definition}[Ipotesi statistica]
	Un'affermazione sul parametro $\theta \in \Theta \subseteq \mathbb{R}$ che governa la legge di un campione statistico $X_1, \ldots, X_n$. Data una partizione $\Theta = \Theta_0 \cup \Theta_1$  in due sottoinsiemi disgiunti, l'ipotesi \textbf{nulla} $H_0$ è l'affermazione logica $\theta \in \Theta_0$, mentre quella \textbf{alternativa} è l'affermazione logica $\theta \in \Theta_1$. I due insiemi sono rispettivamente dei parametri dell'ipotesi nulla e alternativa.
\end{definition}

\begin{definition}[Test statistico]
	È una procedura per decidere se accettare o rifiutare l'ipotesi nulla $H_0$ a partire dai valori assunti dal campione:
	\begin{itemize}
		\item Si \textbf{accetta} se i valori assunti sono con essa compatibili (esito \textbf{negativo})
		\item Si \textbf{rifiuta}, in favore di $H_1$, se con un alto grado di fiducia, i valori non sono compatibili (esito \textbf{positivo})
	\end{itemize}
\end{definition}
\subsubsection{Regione critica}
Fissata l'ipotesi è necessario determinare un insieme di risultati che portano a rifiutare l'ipotesi nulla.
\begin{definition}[Regione critica]
	La regione critica per l'ipotesi nulla $H_0$ è un evento $C \subset \Omega$. Il suo complementare $A = \Omega \setminus C$ è quella di accettazione.\\
	Data una realizzazione $(x_1, \ldots, x_n)$ del campione $(X_1, \ldots, X_n)$, l'ipotesi nulla viene:
	\begin{itemize}
		\item \textbf{Respinta} se si verifica l'evento $C$, ovvero se $X_1(\omega), \ldots, X_n(\omega)$ assume valore $(x_1, \ldots, x_n)$ per un qualche $\omega \in C$
		\item \textbf{Accettata} se non si verifica l'evento $C$, ovvero se all'interno della regione critica $\omega \in C$, $X_1(\omega), \ldots, X_n(\omega)$ non coincide mai con $(x_1, \ldots, x_n)$
	\end{itemize}
\end{definition}

\subsubsection{Valutazione del test}
Il risultato del test è soggetto a due tipi di errore:
\begin{itemize}
	\item \textbf{Prima specie}: consiste nel rifiutare l'ipotesi nulla quando questa è soddisfatta (falso positivo). Se $\theta \in \Theta_0$ la probabilità di commettere questo tipo di errore è $\mathbb{P}_\theta(C)$
	\item \textbf{Seconda specie}: consiste nell'accettare l'ipotesi nulla quando questa non è soddisfatta (falso negativo). Se $\theta \in \Theta_1$ la probabilità di commettere questo tipo di errore è $\mathbb{P}_\theta(A)$
\end{itemize}

\begin{definition}[Livello del test]
	Dato $0 < \alpha < 1$ si dice che il test è a livello se
	\begin{equation}
		\mathbb{P}_\theta(X) \leq \alpha \quad\quad \forall \theta \in \Theta_0
	\end{equation}
	Fissare un livello significa fissare un limite superiore per la probabilità dell'errore di prima specie, scegliendo opportunamente la regione critica.
\end{definition}

\begin{definition}[Potenza del test]
	Si chiama potenza del test la funzione
	\begin{equation}
		\Theta_1 \ni \theta \mapsto \mathbb{P}_\theta(C) \in [0,1]
	\end{equation}
	Rappresenta la probabilità di rifiutare correttamente l'ipotesi nulla, quindi di accorgersi che non è soddisfatta.
\end{definition}

\begin{observation}
	Il test ideale ha livello basso (bassa probabilità dell'errore di prima specie) e potenza alta (bassa probabilità dell'errore di seconda specie). I due fattori sono però in contrapposizione.
\end{observation}

Per ovviare alla dipendenza del risultato del test dal livello, si usa spesso il p-value, che dipende dalla realizzazione $(x_1, \ldots, x_n)$ del campione $(X_1, \ldots, X_n)$. È la probabilità di ottenere dati più estremi rispetto all'ipotesi nulla di quelli già osservati.

\begin{definition}[p-value]
	Data una famiglia di regioni critiche $\{C(\alpha)\}_{\alpha \in (0,1)}$ tali che il test con regione critica $C(\alpha)$ abbia livello $\alpha$. Data una realizzazione $(x_1, \ldots, x_n)$ del campione $(X_1, \ldots, X_n)$, il p-value è il numero $\bar{\alpha}=\bar{\alpha}(x_1, \ldots, x_n)$ tale che:
	\begin{itemize}
		\item se $\alpha < \bar{\alpha}$ l'ipotesi viene accettata dal test 
		\item se $\alpha > \bar{\alpha}$ l'ipotesi viene rifiutata dal test
	\end{itemize}
\end{definition}

\subsection{Tipologie di test}
\subsubsection{Z-Test}
È il test sulla \textbf{media} di un campione \textbf{Gaussiano} con varianza nota.

\paragraph{Ipotesi}
Dato un campione $X_1, \ldots, X_n$ di legge $N(m, \sigma^2)$ con $\sigma > 0$ nota, vogliamo effettuare un test per decidere se la media $m$ coincide o meno con un valore $m_0$. Scegliamo quindi $m = \theta \in \Theta = \mathbb{R}$ e $\Theta_0 = \{m_0\}$ $\Theta_1 = \mathbb{R} \setminus \{m_0\}$